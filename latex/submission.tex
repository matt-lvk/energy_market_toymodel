\documentclass[11pt]{article}

    \usepackage[breakable]{tcolorbox}
    \usepackage{parskip} % Stop auto-indenting (to mimic markdown behaviour)
    

    % Basic figure setup, for now with no caption control since it's done
    % automatically by Pandoc (which extracts ![](path) syntax from Markdown).
    \usepackage{graphicx}
    % Keep aspect ratio if custom image width or height is specified
    \setkeys{Gin}{keepaspectratio}
    % Maintain compatibility with old templates. Remove in nbconvert 6.0
    \let\Oldincludegraphics\includegraphics
    % Ensure that by default, figures have no caption (until we provide a
    % proper Figure object with a Caption API and a way to capture that
    % in the conversion process - todo).
    \usepackage{caption}
    \DeclareCaptionFormat{nocaption}{}
    \captionsetup{format=nocaption,aboveskip=0pt,belowskip=0pt}

    \usepackage{float}
    \floatplacement{figure}{H} % forces figures to be placed at the correct location
    \usepackage{xcolor} % Allow colors to be defined
    \usepackage{enumerate} % Needed for markdown enumerations to work
    \usepackage{geometry} % Used to adjust the document margins
    \usepackage{amsmath} % Equations
    \usepackage{amssymb} % Equations
    \usepackage{textcomp} % defines textquotesingle
    % Hack from http://tex.stackexchange.com/a/47451/13684:
    \AtBeginDocument{%
        \def\PYZsq{\textquotesingle}% Upright quotes in Pygmentized code
    }
    \usepackage{upquote} % Upright quotes for verbatim code
    \usepackage{eurosym} % defines \euro

    \usepackage{iftex}
    \ifPDFTeX
        \usepackage[T1]{fontenc}
        \IfFileExists{alphabeta.sty}{
              \usepackage{alphabeta}
          }{
              \usepackage[mathletters]{ucs}
              \usepackage[utf8x]{inputenc}
          }
    \else
        \usepackage{fontspec}
        \usepackage{unicode-math}
    \fi

    \usepackage{fancyvrb} % verbatim replacement that allows latex
    \usepackage{grffile} % extends the file name processing of package graphics
                         % to support a larger range
    \makeatletter % fix for old versions of grffile with XeLaTeX
    \@ifpackagelater{grffile}{2019/11/01}
    {
      % Do nothing on new versions
    }
    {
      \def\Gread@@xetex#1{%
        \IfFileExists{"\Gin@base".bb}%
        {\Gread@eps{\Gin@base.bb}}%
        {\Gread@@xetex@aux#1}%
      }
    }
    \makeatother
    \usepackage[Export]{adjustbox} % Used to constrain images to a maximum size
    \adjustboxset{max size={0.9\linewidth}{0.9\paperheight}}

    % The hyperref package gives us a pdf with properly built
    % internal navigation ('pdf bookmarks' for the table of contents,
    % internal cross-reference links, web links for URLs, etc.)
    \usepackage{hyperref}
    % The default LaTeX title has an obnoxious amount of whitespace. By default,
    % titling removes some of it. It also provides customization options.
    \usepackage{titling}
    \usepackage{longtable} % longtable support required by pandoc >1.10
    \usepackage{booktabs}  % table support for pandoc > 1.12.2
    \usepackage{array}     % table support for pandoc >= 2.11.3
    \usepackage{calc}      % table minipage width calculation for pandoc >= 2.11.1
    \usepackage[inline]{enumitem} % IRkernel/repr support (it uses the enumerate* environment)
    \usepackage[normalem]{ulem} % ulem is needed to support strikethroughs (\sout)
                                % normalem makes italics be italics, not underlines
    \usepackage{soul}      % strikethrough (\st) support for pandoc >= 3.0.0
    \usepackage{mathrsfs}
    

    
    % Colors for the hyperref package
    \definecolor{urlcolor}{rgb}{0,.145,.698}
    \definecolor{linkcolor}{rgb}{.71,0.21,0.01}
    \definecolor{citecolor}{rgb}{.12,.54,.11}

    % ANSI colors
    \definecolor{ansi-black}{HTML}{3E424D}
    \definecolor{ansi-black-intense}{HTML}{282C36}
    \definecolor{ansi-red}{HTML}{E75C58}
    \definecolor{ansi-red-intense}{HTML}{B22B31}
    \definecolor{ansi-green}{HTML}{00A250}
    \definecolor{ansi-green-intense}{HTML}{007427}
    \definecolor{ansi-yellow}{HTML}{DDB62B}
    \definecolor{ansi-yellow-intense}{HTML}{B27D12}
    \definecolor{ansi-blue}{HTML}{208FFB}
    \definecolor{ansi-blue-intense}{HTML}{0065CA}
    \definecolor{ansi-magenta}{HTML}{D160C4}
    \definecolor{ansi-magenta-intense}{HTML}{A03196}
    \definecolor{ansi-cyan}{HTML}{60C6C8}
    \definecolor{ansi-cyan-intense}{HTML}{258F8F}
    \definecolor{ansi-white}{HTML}{C5C1B4}
    \definecolor{ansi-white-intense}{HTML}{A1A6B2}
    \definecolor{ansi-default-inverse-fg}{HTML}{FFFFFF}
    \definecolor{ansi-default-inverse-bg}{HTML}{000000}

    % common color for the border for error outputs.
    \definecolor{outerrorbackground}{HTML}{FFDFDF}

    % commands and environments needed by pandoc snippets
    % extracted from the output of `pandoc -s`
    \providecommand{\tightlist}{%
      \setlength{\itemsep}{0pt}\setlength{\parskip}{0pt}}
    \DefineVerbatimEnvironment{Highlighting}{Verbatim}{commandchars=\\\{\}}
    % Add ',fontsize=\small' for more characters per line
    \newenvironment{Shaded}{}{}
    \newcommand{\KeywordTok}[1]{\textcolor[rgb]{0.00,0.44,0.13}{\textbf{{#1}}}}
    \newcommand{\DataTypeTok}[1]{\textcolor[rgb]{0.56,0.13,0.00}{{#1}}}
    \newcommand{\DecValTok}[1]{\textcolor[rgb]{0.25,0.63,0.44}{{#1}}}
    \newcommand{\BaseNTok}[1]{\textcolor[rgb]{0.25,0.63,0.44}{{#1}}}
    \newcommand{\FloatTok}[1]{\textcolor[rgb]{0.25,0.63,0.44}{{#1}}}
    \newcommand{\CharTok}[1]{\textcolor[rgb]{0.25,0.44,0.63}{{#1}}}
    \newcommand{\StringTok}[1]{\textcolor[rgb]{0.25,0.44,0.63}{{#1}}}
    \newcommand{\CommentTok}[1]{\textcolor[rgb]{0.38,0.63,0.69}{\textit{{#1}}}}
    \newcommand{\OtherTok}[1]{\textcolor[rgb]{0.00,0.44,0.13}{{#1}}}
    \newcommand{\AlertTok}[1]{\textcolor[rgb]{1.00,0.00,0.00}{\textbf{{#1}}}}
    \newcommand{\FunctionTok}[1]{\textcolor[rgb]{0.02,0.16,0.49}{{#1}}}
    \newcommand{\RegionMarkerTok}[1]{{#1}}
    \newcommand{\ErrorTok}[1]{\textcolor[rgb]{1.00,0.00,0.00}{\textbf{{#1}}}}
    \newcommand{\NormalTok}[1]{{#1}}

    % Additional commands for more recent versions of Pandoc
    \newcommand{\ConstantTok}[1]{\textcolor[rgb]{0.53,0.00,0.00}{{#1}}}
    \newcommand{\SpecialCharTok}[1]{\textcolor[rgb]{0.25,0.44,0.63}{{#1}}}
    \newcommand{\VerbatimStringTok}[1]{\textcolor[rgb]{0.25,0.44,0.63}{{#1}}}
    \newcommand{\SpecialStringTok}[1]{\textcolor[rgb]{0.73,0.40,0.53}{{#1}}}
    \newcommand{\ImportTok}[1]{{#1}}
    \newcommand{\DocumentationTok}[1]{\textcolor[rgb]{0.73,0.13,0.13}{\textit{{#1}}}}
    \newcommand{\AnnotationTok}[1]{\textcolor[rgb]{0.38,0.63,0.69}{\textbf{\textit{{#1}}}}}
    \newcommand{\CommentVarTok}[1]{\textcolor[rgb]{0.38,0.63,0.69}{\textbf{\textit{{#1}}}}}
    \newcommand{\VariableTok}[1]{\textcolor[rgb]{0.10,0.09,0.49}{{#1}}}
    \newcommand{\ControlFlowTok}[1]{\textcolor[rgb]{0.00,0.44,0.13}{\textbf{{#1}}}}
    \newcommand{\OperatorTok}[1]{\textcolor[rgb]{0.40,0.40,0.40}{{#1}}}
    \newcommand{\BuiltInTok}[1]{{#1}}
    \newcommand{\ExtensionTok}[1]{{#1}}
    \newcommand{\PreprocessorTok}[1]{\textcolor[rgb]{0.74,0.48,0.00}{{#1}}}
    \newcommand{\AttributeTok}[1]{\textcolor[rgb]{0.49,0.56,0.16}{{#1}}}
    \newcommand{\InformationTok}[1]{\textcolor[rgb]{0.38,0.63,0.69}{\textbf{\textit{{#1}}}}}
    \newcommand{\WarningTok}[1]{\textcolor[rgb]{0.38,0.63,0.69}{\textbf{\textit{{#1}}}}}
    \makeatletter
    \newsavebox\pandoc@box
    \newcommand*\pandocbounded[1]{%
      \sbox\pandoc@box{#1}%
      % scaling factors for width and height
      \Gscale@div\@tempa\textheight{\dimexpr\ht\pandoc@box+\dp\pandoc@box\relax}%
      \Gscale@div\@tempb\linewidth{\wd\pandoc@box}%
      % select the smaller of both
      \ifdim\@tempb\p@<\@tempa\p@
        \let\@tempa\@tempb
      \fi
      % scaling accordingly (\@tempa < 1)
      \ifdim\@tempa\p@<\p@
        \scalebox{\@tempa}{\usebox\pandoc@box}%
      % scaling not needed, use as it is
      \else
        \usebox{\pandoc@box}%
      \fi
    }
    \makeatother

    % Define a nice break command that doesn't care if a line doesn't already
    % exist.
    \def\br{\hspace*{\fill} \\* }
    % Math Jax compatibility definitions
    \def\gt{>}
    \def\lt{<}
    \let\Oldtex\TeX
    \let\Oldlatex\LaTeX
    \renewcommand{\TeX}{\textrm{\Oldtex}}
    \renewcommand{\LaTeX}{\textrm{\Oldlatex}}
    % Document parameters
    % Document title
    \title{submission}
    
    
    
    
    
    
    
% Pygments definitions
\makeatletter
\def\PY@reset{\let\PY@it=\relax \let\PY@bf=\relax%
    \let\PY@ul=\relax \let\PY@tc=\relax%
    \let\PY@bc=\relax \let\PY@ff=\relax}
\def\PY@tok#1{\csname PY@tok@#1\endcsname}
\def\PY@toks#1+{\ifx\relax#1\empty\else%
    \PY@tok{#1}\expandafter\PY@toks\fi}
\def\PY@do#1{\PY@bc{\PY@tc{\PY@ul{%
    \PY@it{\PY@bf{\PY@ff{#1}}}}}}}
\def\PY#1#2{\PY@reset\PY@toks#1+\relax+\PY@do{#2}}

\@namedef{PY@tok@w}{\def\PY@tc##1{\textcolor[rgb]{0.73,0.73,0.73}{##1}}}
\@namedef{PY@tok@c}{\let\PY@it=\textit\def\PY@tc##1{\textcolor[rgb]{0.24,0.48,0.48}{##1}}}
\@namedef{PY@tok@cp}{\def\PY@tc##1{\textcolor[rgb]{0.61,0.40,0.00}{##1}}}
\@namedef{PY@tok@k}{\let\PY@bf=\textbf\def\PY@tc##1{\textcolor[rgb]{0.00,0.50,0.00}{##1}}}
\@namedef{PY@tok@kp}{\def\PY@tc##1{\textcolor[rgb]{0.00,0.50,0.00}{##1}}}
\@namedef{PY@tok@kt}{\def\PY@tc##1{\textcolor[rgb]{0.69,0.00,0.25}{##1}}}
\@namedef{PY@tok@o}{\def\PY@tc##1{\textcolor[rgb]{0.40,0.40,0.40}{##1}}}
\@namedef{PY@tok@ow}{\let\PY@bf=\textbf\def\PY@tc##1{\textcolor[rgb]{0.67,0.13,1.00}{##1}}}
\@namedef{PY@tok@nb}{\def\PY@tc##1{\textcolor[rgb]{0.00,0.50,0.00}{##1}}}
\@namedef{PY@tok@nf}{\def\PY@tc##1{\textcolor[rgb]{0.00,0.00,1.00}{##1}}}
\@namedef{PY@tok@nc}{\let\PY@bf=\textbf\def\PY@tc##1{\textcolor[rgb]{0.00,0.00,1.00}{##1}}}
\@namedef{PY@tok@nn}{\let\PY@bf=\textbf\def\PY@tc##1{\textcolor[rgb]{0.00,0.00,1.00}{##1}}}
\@namedef{PY@tok@ne}{\let\PY@bf=\textbf\def\PY@tc##1{\textcolor[rgb]{0.80,0.25,0.22}{##1}}}
\@namedef{PY@tok@nv}{\def\PY@tc##1{\textcolor[rgb]{0.10,0.09,0.49}{##1}}}
\@namedef{PY@tok@no}{\def\PY@tc##1{\textcolor[rgb]{0.53,0.00,0.00}{##1}}}
\@namedef{PY@tok@nl}{\def\PY@tc##1{\textcolor[rgb]{0.46,0.46,0.00}{##1}}}
\@namedef{PY@tok@ni}{\let\PY@bf=\textbf\def\PY@tc##1{\textcolor[rgb]{0.44,0.44,0.44}{##1}}}
\@namedef{PY@tok@na}{\def\PY@tc##1{\textcolor[rgb]{0.41,0.47,0.13}{##1}}}
\@namedef{PY@tok@nt}{\let\PY@bf=\textbf\def\PY@tc##1{\textcolor[rgb]{0.00,0.50,0.00}{##1}}}
\@namedef{PY@tok@nd}{\def\PY@tc##1{\textcolor[rgb]{0.67,0.13,1.00}{##1}}}
\@namedef{PY@tok@s}{\def\PY@tc##1{\textcolor[rgb]{0.73,0.13,0.13}{##1}}}
\@namedef{PY@tok@sd}{\let\PY@it=\textit\def\PY@tc##1{\textcolor[rgb]{0.73,0.13,0.13}{##1}}}
\@namedef{PY@tok@si}{\let\PY@bf=\textbf\def\PY@tc##1{\textcolor[rgb]{0.64,0.35,0.47}{##1}}}
\@namedef{PY@tok@se}{\let\PY@bf=\textbf\def\PY@tc##1{\textcolor[rgb]{0.67,0.36,0.12}{##1}}}
\@namedef{PY@tok@sr}{\def\PY@tc##1{\textcolor[rgb]{0.64,0.35,0.47}{##1}}}
\@namedef{PY@tok@ss}{\def\PY@tc##1{\textcolor[rgb]{0.10,0.09,0.49}{##1}}}
\@namedef{PY@tok@sx}{\def\PY@tc##1{\textcolor[rgb]{0.00,0.50,0.00}{##1}}}
\@namedef{PY@tok@m}{\def\PY@tc##1{\textcolor[rgb]{0.40,0.40,0.40}{##1}}}
\@namedef{PY@tok@gh}{\let\PY@bf=\textbf\def\PY@tc##1{\textcolor[rgb]{0.00,0.00,0.50}{##1}}}
\@namedef{PY@tok@gu}{\let\PY@bf=\textbf\def\PY@tc##1{\textcolor[rgb]{0.50,0.00,0.50}{##1}}}
\@namedef{PY@tok@gd}{\def\PY@tc##1{\textcolor[rgb]{0.63,0.00,0.00}{##1}}}
\@namedef{PY@tok@gi}{\def\PY@tc##1{\textcolor[rgb]{0.00,0.52,0.00}{##1}}}
\@namedef{PY@tok@gr}{\def\PY@tc##1{\textcolor[rgb]{0.89,0.00,0.00}{##1}}}
\@namedef{PY@tok@ge}{\let\PY@it=\textit}
\@namedef{PY@tok@gs}{\let\PY@bf=\textbf}
\@namedef{PY@tok@ges}{\let\PY@bf=\textbf\let\PY@it=\textit}
\@namedef{PY@tok@gp}{\let\PY@bf=\textbf\def\PY@tc##1{\textcolor[rgb]{0.00,0.00,0.50}{##1}}}
\@namedef{PY@tok@go}{\def\PY@tc##1{\textcolor[rgb]{0.44,0.44,0.44}{##1}}}
\@namedef{PY@tok@gt}{\def\PY@tc##1{\textcolor[rgb]{0.00,0.27,0.87}{##1}}}
\@namedef{PY@tok@err}{\def\PY@bc##1{{\setlength{\fboxsep}{\string -\fboxrule}\fcolorbox[rgb]{1.00,0.00,0.00}{1,1,1}{\strut ##1}}}}
\@namedef{PY@tok@kc}{\let\PY@bf=\textbf\def\PY@tc##1{\textcolor[rgb]{0.00,0.50,0.00}{##1}}}
\@namedef{PY@tok@kd}{\let\PY@bf=\textbf\def\PY@tc##1{\textcolor[rgb]{0.00,0.50,0.00}{##1}}}
\@namedef{PY@tok@kn}{\let\PY@bf=\textbf\def\PY@tc##1{\textcolor[rgb]{0.00,0.50,0.00}{##1}}}
\@namedef{PY@tok@kr}{\let\PY@bf=\textbf\def\PY@tc##1{\textcolor[rgb]{0.00,0.50,0.00}{##1}}}
\@namedef{PY@tok@bp}{\def\PY@tc##1{\textcolor[rgb]{0.00,0.50,0.00}{##1}}}
\@namedef{PY@tok@fm}{\def\PY@tc##1{\textcolor[rgb]{0.00,0.00,1.00}{##1}}}
\@namedef{PY@tok@vc}{\def\PY@tc##1{\textcolor[rgb]{0.10,0.09,0.49}{##1}}}
\@namedef{PY@tok@vg}{\def\PY@tc##1{\textcolor[rgb]{0.10,0.09,0.49}{##1}}}
\@namedef{PY@tok@vi}{\def\PY@tc##1{\textcolor[rgb]{0.10,0.09,0.49}{##1}}}
\@namedef{PY@tok@vm}{\def\PY@tc##1{\textcolor[rgb]{0.10,0.09,0.49}{##1}}}
\@namedef{PY@tok@sa}{\def\PY@tc##1{\textcolor[rgb]{0.73,0.13,0.13}{##1}}}
\@namedef{PY@tok@sb}{\def\PY@tc##1{\textcolor[rgb]{0.73,0.13,0.13}{##1}}}
\@namedef{PY@tok@sc}{\def\PY@tc##1{\textcolor[rgb]{0.73,0.13,0.13}{##1}}}
\@namedef{PY@tok@dl}{\def\PY@tc##1{\textcolor[rgb]{0.73,0.13,0.13}{##1}}}
\@namedef{PY@tok@s2}{\def\PY@tc##1{\textcolor[rgb]{0.73,0.13,0.13}{##1}}}
\@namedef{PY@tok@sh}{\def\PY@tc##1{\textcolor[rgb]{0.73,0.13,0.13}{##1}}}
\@namedef{PY@tok@s1}{\def\PY@tc##1{\textcolor[rgb]{0.73,0.13,0.13}{##1}}}
\@namedef{PY@tok@mb}{\def\PY@tc##1{\textcolor[rgb]{0.40,0.40,0.40}{##1}}}
\@namedef{PY@tok@mf}{\def\PY@tc##1{\textcolor[rgb]{0.40,0.40,0.40}{##1}}}
\@namedef{PY@tok@mh}{\def\PY@tc##1{\textcolor[rgb]{0.40,0.40,0.40}{##1}}}
\@namedef{PY@tok@mi}{\def\PY@tc##1{\textcolor[rgb]{0.40,0.40,0.40}{##1}}}
\@namedef{PY@tok@il}{\def\PY@tc##1{\textcolor[rgb]{0.40,0.40,0.40}{##1}}}
\@namedef{PY@tok@mo}{\def\PY@tc##1{\textcolor[rgb]{0.40,0.40,0.40}{##1}}}
\@namedef{PY@tok@ch}{\let\PY@it=\textit\def\PY@tc##1{\textcolor[rgb]{0.24,0.48,0.48}{##1}}}
\@namedef{PY@tok@cm}{\let\PY@it=\textit\def\PY@tc##1{\textcolor[rgb]{0.24,0.48,0.48}{##1}}}
\@namedef{PY@tok@cpf}{\let\PY@it=\textit\def\PY@tc##1{\textcolor[rgb]{0.24,0.48,0.48}{##1}}}
\@namedef{PY@tok@c1}{\let\PY@it=\textit\def\PY@tc##1{\textcolor[rgb]{0.24,0.48,0.48}{##1}}}
\@namedef{PY@tok@cs}{\let\PY@it=\textit\def\PY@tc##1{\textcolor[rgb]{0.24,0.48,0.48}{##1}}}

\def\PYZbs{\char`\\}
\def\PYZus{\char`\_}
\def\PYZob{\char`\{}
\def\PYZcb{\char`\}}
\def\PYZca{\char`\^}
\def\PYZam{\char`\&}
\def\PYZlt{\char`\<}
\def\PYZgt{\char`\>}
\def\PYZsh{\char`\#}
\def\PYZpc{\char`\%}
\def\PYZdl{\char`\$}
\def\PYZhy{\char`\-}
\def\PYZsq{\char`\'}
\def\PYZdq{\char`\"}
\def\PYZti{\char`\~}
% for compatibility with earlier versions
\def\PYZat{@}
\def\PYZlb{[}
\def\PYZrb{]}
\makeatother


    % For linebreaks inside Verbatim environment from package fancyvrb.
    \makeatletter
        \newbox\Wrappedcontinuationbox
        \newbox\Wrappedvisiblespacebox
        \newcommand*\Wrappedvisiblespace {\textcolor{red}{\textvisiblespace}}
        \newcommand*\Wrappedcontinuationsymbol {\textcolor{red}{\llap{\tiny$\m@th\hookrightarrow$}}}
        \newcommand*\Wrappedcontinuationindent {3ex }
        \newcommand*\Wrappedafterbreak {\kern\Wrappedcontinuationindent\copy\Wrappedcontinuationbox}
        % Take advantage of the already applied Pygments mark-up to insert
        % potential linebreaks for TeX processing.
        %        {, <, #, %, $, ' and ": go to next line.
        %        _, }, ^, &, >, - and ~: stay at end of broken line.
        % Use of \textquotesingle for straight quote.
        \newcommand*\Wrappedbreaksatspecials {%
            \def\PYGZus{\discretionary{\char`\_}{\Wrappedafterbreak}{\char`\_}}%
            \def\PYGZob{\discretionary{}{\Wrappedafterbreak\char`\{}{\char`\{}}%
            \def\PYGZcb{\discretionary{\char`\}}{\Wrappedafterbreak}{\char`\}}}%
            \def\PYGZca{\discretionary{\char`\^}{\Wrappedafterbreak}{\char`\^}}%
            \def\PYGZam{\discretionary{\char`\&}{\Wrappedafterbreak}{\char`\&}}%
            \def\PYGZlt{\discretionary{}{\Wrappedafterbreak\char`\<}{\char`\<}}%
            \def\PYGZgt{\discretionary{\char`\>}{\Wrappedafterbreak}{\char`\>}}%
            \def\PYGZsh{\discretionary{}{\Wrappedafterbreak\char`\#}{\char`\#}}%
            \def\PYGZpc{\discretionary{}{\Wrappedafterbreak\char`\%}{\char`\%}}%
            \def\PYGZdl{\discretionary{}{\Wrappedafterbreak\char`\$}{\char`\$}}%
            \def\PYGZhy{\discretionary{\char`\-}{\Wrappedafterbreak}{\char`\-}}%
            \def\PYGZsq{\discretionary{}{\Wrappedafterbreak\textquotesingle}{\textquotesingle}}%
            \def\PYGZdq{\discretionary{}{\Wrappedafterbreak\char`\"}{\char`\"}}%
            \def\PYGZti{\discretionary{\char`\~}{\Wrappedafterbreak}{\char`\~}}%
        }
        % Some characters . , ; ? ! / are not pygmentized.
        % This macro makes them "active" and they will insert potential linebreaks
        \newcommand*\Wrappedbreaksatpunct {%
            \lccode`\~`\.\lowercase{\def~}{\discretionary{\hbox{\char`\.}}{\Wrappedafterbreak}{\hbox{\char`\.}}}%
            \lccode`\~`\,\lowercase{\def~}{\discretionary{\hbox{\char`\,}}{\Wrappedafterbreak}{\hbox{\char`\,}}}%
            \lccode`\~`\;\lowercase{\def~}{\discretionary{\hbox{\char`\;}}{\Wrappedafterbreak}{\hbox{\char`\;}}}%
            \lccode`\~`\:\lowercase{\def~}{\discretionary{\hbox{\char`\:}}{\Wrappedafterbreak}{\hbox{\char`\:}}}%
            \lccode`\~`\?\lowercase{\def~}{\discretionary{\hbox{\char`\?}}{\Wrappedafterbreak}{\hbox{\char`\?}}}%
            \lccode`\~`\!\lowercase{\def~}{\discretionary{\hbox{\char`\!}}{\Wrappedafterbreak}{\hbox{\char`\!}}}%
            \lccode`\~`\/\lowercase{\def~}{\discretionary{\hbox{\char`\/}}{\Wrappedafterbreak}{\hbox{\char`\/}}}%
            \catcode`\.\active
            \catcode`\,\active
            \catcode`\;\active
            \catcode`\:\active
            \catcode`\?\active
            \catcode`\!\active
            \catcode`\/\active
            \lccode`\~`\~
        }
    \makeatother

    \let\OriginalVerbatim=\Verbatim
    \makeatletter
    \renewcommand{\Verbatim}[1][1]{%
        %\parskip\z@skip
        \sbox\Wrappedcontinuationbox {\Wrappedcontinuationsymbol}%
        \sbox\Wrappedvisiblespacebox {\FV@SetupFont\Wrappedvisiblespace}%
        \def\FancyVerbFormatLine ##1{\hsize\linewidth
            \vtop{\raggedright\hyphenpenalty\z@\exhyphenpenalty\z@
                \doublehyphendemerits\z@\finalhyphendemerits\z@
                \strut ##1\strut}%
        }%
        % If the linebreak is at a space, the latter will be displayed as visible
        % space at end of first line, and a continuation symbol starts next line.
        % Stretch/shrink are however usually zero for typewriter font.
        \def\FV@Space {%
            \nobreak\hskip\z@ plus\fontdimen3\font minus\fontdimen4\font
            \discretionary{\copy\Wrappedvisiblespacebox}{\Wrappedafterbreak}
            {\kern\fontdimen2\font}%
        }%

        % Allow breaks at special characters using \PYG... macros.
        \Wrappedbreaksatspecials
        % Breaks at punctuation characters . , ; ? ! and / need catcode=\active
        \OriginalVerbatim[#1,codes*=\Wrappedbreaksatpunct]%
    }
    \makeatother

    % Exact colors from NB
    \definecolor{incolor}{HTML}{303F9F}
    \definecolor{outcolor}{HTML}{D84315}
    \definecolor{cellborder}{HTML}{CFCFCF}
    \definecolor{cellbackground}{HTML}{F7F7F7}

    % prompt
    \makeatletter
    \newcommand{\boxspacing}{\kern\kvtcb@left@rule\kern\kvtcb@boxsep}
    \makeatother
    \newcommand{\prompt}[4]{
        {\ttfamily\llap{{\color{#2}[#3]:\hspace{3pt}#4}}\vspace{-\baselineskip}}
    }
    

    
    % Prevent overflowing lines due to hard-to-break entities
    \sloppy
    % Setup hyperref package
    \hypersetup{
      breaklinks=true,  % so long urls are correctly broken across lines
      colorlinks=true,
      urlcolor=urlcolor,
      linkcolor=linkcolor,
      citecolor=citecolor,
      }
    % Slightly bigger margins than the latex defaults
    
    \geometry{verbose,tmargin=1in,bmargin=1in,lmargin=1in,rmargin=1in}
    
    

\begin{document}
    
    \maketitle
    
    

    
    \textbf{Matthew Liew}

{Entrix: Case Study Challenge}

\textbf{Table of Contents}:

\begin{itemize}
\tightlist
\item
  \hyperref[q1]{Question 1}

  \begin{enumerate}
  \def\labelenumi{\arabic{enumi}.}
  \tightlist
  \item
    \hyperref[discovery]{Data Discovery}
  \item
    \hyperref[arima]{ARIMA}
  \item
    \hyperref[prophet]{Prophet}
  \item
    \hyperref[xgboost]{XGBoost}
  \item
    \hyperref[backtesting_forecast]{Backtesting: Forecast Models}
  \item
    \hyperref[q1_conclusion]{Q1 Analysis Conclusion}
  \end{enumerate}
\item
  \hyperref[q2]{Question 2}

  \begin{enumerate}
  \def\labelenumi{\arabic{enumi}.}
  \tightlist
  \item
    \hyperref[problem]{Problem Formulation}
  \item
    \hyperref[opt_implementation]{Implementation: CVXPY}
  \item
    \hyperref[backtest_opt]{Backtesting: Optimization}
  \item
    \hyperref[q2_con]{Q2 Conclusion}
  \end{enumerate}
\item
  \hyperref[6]{Question 3}

  \begin{enumerate}
  \def\labelenumi{\arabic{enumi}.}
  \tightlist
  \item
    \hyperref[setup]{Setup}
  \item
    \hyperref[training]{Training Strategy}
  \item
    \hyperref[q3_con]{Q3 Conclusion}
  \end{enumerate}
\end{itemize}

    \begin{tcolorbox}[breakable, size=fbox, boxrule=1pt, pad at break*=1mm,colback=cellbackground, colframe=cellborder]
\prompt{In}{incolor}{1}{\boxspacing}
\begin{Verbatim}[commandchars=\\\{\}]
\PY{c+c1}{\PYZsh{} load necessary packages}

\PY{k+kn}{import}\PY{+w}{ }\PY{n+nn}{warnings}
\PY{k+kn}{from}\PY{+w}{ }\PY{n+nn}{datetime}\PY{+w}{ }\PY{k+kn}{import} \PY{n}{datetime}\PY{p}{,} \PY{n}{timedelta}

\PY{c+c1}{\PYZsh{} Data manipulation}
\PY{k+kn}{import}\PY{+w}{ }\PY{n+nn}{numpy}\PY{+w}{ }\PY{k}{as}\PY{+w}{ }\PY{n+nn}{np}
\PY{k+kn}{import}\PY{+w}{ }\PY{n+nn}{pandas}\PY{+w}{ }\PY{k}{as}\PY{+w}{ }\PY{n+nn}{pd}

\PY{c+c1}{\PYZsh{} Visualization}
\PY{k+kn}{import}\PY{+w}{ }\PY{n+nn}{matplotlib}\PY{n+nn}{.}\PY{n+nn}{pyplot}\PY{+w}{ }\PY{k}{as}\PY{+w}{ }\PY{n+nn}{plt}
\PY{k+kn}{import}\PY{+w}{ }\PY{n+nn}{seaborn}\PY{+w}{ }\PY{k}{as}\PY{+w}{ }\PY{n+nn}{sns}
\PY{k+kn}{import}\PY{+w}{ }\PY{n+nn}{plotly}\PY{n+nn}{.}\PY{n+nn}{express}\PY{+w}{ }\PY{k}{as}\PY{+w}{ }\PY{n+nn}{px}

\PY{c+c1}{\PYZsh{} Time series and stats}
\PY{k+kn}{from}\PY{+w}{ }\PY{n+nn}{darts}\PY{+w}{ }\PY{k+kn}{import} \PY{n}{TimeSeries}
\PY{k+kn}{from}\PY{+w}{ }\PY{n+nn}{darts}\PY{n+nn}{.}\PY{n+nn}{utils}\PY{n+nn}{.}\PY{n+nn}{statistics}\PY{+w}{ }\PY{k+kn}{import} \PY{p}{(}
    \PY{n}{check\PYZus{}seasonality}\PY{p}{,} \PY{n}{plot\PYZus{}acf}\PY{p}{,} \PY{n}{plot\PYZus{}pacf}\PY{p}{,}
    \PY{n}{remove\PYZus{}seasonality}\PY{p}{,} \PY{n}{remove\PYZus{}trend}\PY{p}{,}
    \PY{n}{stationarity\PYZus{}test\PYZus{}adf}\PY{p}{,} \PY{n}{extract\PYZus{}trend\PYZus{}and\PYZus{}seasonality}
\PY{p}{)}
\PY{k+kn}{from}\PY{+w}{ }\PY{n+nn}{darts}\PY{n+nn}{.}\PY{n+nn}{utils}\PY{n+nn}{.}\PY{n+nn}{model\PYZus{}selection}\PY{+w}{ }\PY{k+kn}{import} \PY{n}{train\PYZus{}test\PYZus{}split}
\PY{k+kn}{from}\PY{+w}{ }\PY{n+nn}{statsmodels}\PY{n+nn}{.}\PY{n+nn}{tsa}\PY{n+nn}{.}\PY{n+nn}{seasonal}\PY{+w}{ }\PY{k+kn}{import} \PY{n}{seasonal\PYZus{}decompose}

\PY{c+c1}{\PYZsh{} Machine learning \PYZam{} metrics}
\PY{k+kn}{from}\PY{+w}{ }\PY{n+nn}{sklearn}\PY{n+nn}{.}\PY{n+nn}{metrics}\PY{+w}{ }\PY{k+kn}{import} \PY{n}{mean\PYZus{}squared\PYZus{}error}\PY{p}{,} \PY{n}{mean\PYZus{}absolute\PYZus{}error}\PY{p}{,} \PY{n}{mean\PYZus{}absolute\PYZus{}percentage\PYZus{}error}
\PY{k+kn}{from}\PY{+w}{ }\PY{n+nn}{sklearn}\PY{n+nn}{.}\PY{n+nn}{decomposition}\PY{+w}{ }\PY{k+kn}{import} \PY{n}{PCA}
\PY{k+kn}{from}\PY{+w}{ }\PY{n+nn}{sklearn}\PY{n+nn}{.}\PY{n+nn}{preprocessing}\PY{+w}{ }\PY{k+kn}{import} \PY{n}{StandardScaler}
\PY{k+kn}{import}\PY{+w}{ }\PY{n+nn}{xgboost}\PY{+w}{ }\PY{k}{as}\PY{+w}{ }\PY{n+nn}{xgb}
\PY{k+kn}{from}\PY{+w}{ }\PY{n+nn}{xgboost}\PY{+w}{ }\PY{k+kn}{import} \PY{n}{plot\PYZus{}importance}\PY{p}{,} \PY{n}{plot\PYZus{}tree}
\PY{k+kn}{from}\PY{+w}{ }\PY{n+nn}{prophet}\PY{+w}{ }\PY{k+kn}{import} \PY{n}{Prophet}
\PY{k+kn}{import}\PY{+w}{ }\PY{n+nn}{cvxpy}\PY{+w}{ }\PY{k}{as}\PY{+w}{ }\PY{n+nn}{cp}

\PY{c+c1}{\PYZsh{} External data sources}
\PY{k+kn}{from}\PY{+w}{ }\PY{n+nn}{meteostat}\PY{+w}{ }\PY{k+kn}{import} \PY{n}{Point}\PY{p}{,} \PY{n}{Hourly}
\PY{k+kn}{import}\PY{+w}{ }\PY{n+nn}{holidays}
\PY{k+kn}{import}\PY{+w}{ }\PY{n+nn}{yfinance}\PY{+w}{ }\PY{k}{as}\PY{+w}{ }\PY{n+nn}{yf}

\PY{c+c1}{\PYZsh{} Custom utilities and wrappers}
\PY{k+kn}{import}\PY{+w}{ }\PY{n+nn}{data\PYZus{}utils}\PY{+w}{ }\PY{k}{as}\PY{+w}{ }\PY{n+nn}{du}
\PY{k+kn}{import}\PY{+w}{ }\PY{n+nn}{plotters}
\PY{k+kn}{from}\PY{+w}{ }\PY{n+nn}{forecast\PYZus{}utils}\PY{+w}{ }\PY{k+kn}{import} \PY{n}{ARIMAWrapper}\PY{p}{,} \PY{n}{XGBWrapper}\PY{p}{,} \PY{n}{ProphetWrapper}
\PY{k+kn}{from}\PY{+w}{ }\PY{n+nn}{backtesting}\PY{+w}{ }\PY{k+kn}{import} \PY{n}{BacktestXGB}\PY{p}{,} \PY{n}{BacktestProphet}\PY{p}{,} \PY{n}{BacktestARIMA}


\PY{c+c1}{\PYZsh{} To keep track of the RSME and MAE metrics}
\PY{n}{forecast\PYZus{}metric\PYZus{}long} \PY{o}{=} \PY{p}{\PYZob{}}\PY{p}{\PYZcb{}}
\PY{n}{forecast\PYZus{}metric\PYZus{}short} \PY{o}{=} \PY{p}{\PYZob{}}\PY{p}{\PYZcb{}}
\PY{n}{warnings}\PY{o}{.}\PY{n}{simplefilter}\PY{p}{(}\PY{n}{action}\PY{o}{=}\PY{l+s+s1}{\PYZsq{}}\PY{l+s+s1}{ignore}\PY{l+s+s1}{\PYZsq{}}\PY{p}{,} \PY{n}{category}\PY{o}{=}\PY{n+ne}{FutureWarning}\PY{p}{)}
\end{Verbatim}
\end{tcolorbox}

    Question 1

    Data Discovery

    \begin{tcolorbox}[breakable, size=fbox, boxrule=1pt, pad at break*=1mm,colback=cellbackground, colframe=cellborder]
\prompt{In}{incolor}{2}{\boxspacing}
\begin{Verbatim}[commandchars=\\\{\}]
\PY{c+c1}{\PYZsh{} load data set}

\PY{n}{day\PYZus{}price\PYZus{}df} \PY{o}{=} \PY{n}{pd}\PY{o}{.}\PY{n}{read\PYZus{}csv}\PY{p}{(}\PY{l+s+s1}{\PYZsq{}}\PY{l+s+s1}{./data/Day\PYZhy{}ahead\PYZus{}Prices\PYZus{}60min.csv}\PY{l+s+s1}{\PYZsq{}}\PY{p}{)}

\PY{c+c1}{\PYZsh{} split the date column into start and end time, we will use start time only}
\PY{n}{day\PYZus{}price\PYZus{}df}\PY{p}{[}\PY{p}{[}\PY{l+s+s1}{\PYZsq{}}\PY{l+s+s1}{start\PYZus{}time}\PY{l+s+s1}{\PYZsq{}}\PY{p}{,} \PY{l+s+s1}{\PYZsq{}}\PY{l+s+s1}{end\PYZus{}time}\PY{l+s+s1}{\PYZsq{}}\PY{p}{]}\PY{p}{]} \PY{o}{=} \PY{n}{day\PYZus{}price\PYZus{}df}\PY{p}{[}\PY{l+s+s1}{\PYZsq{}}\PY{l+s+s1}{MTU (CET/CEST)}\PY{l+s+s1}{\PYZsq{}}\PY{p}{]}\PYZbs{}
                                                \PY{o}{.}\PY{n}{str}\PY{o}{.}\PY{n}{split}\PY{p}{(}\PY{l+s+s1}{\PYZsq{}}\PY{l+s+s1}{ \PYZhy{} }\PY{l+s+s1}{\PYZsq{}}\PY{p}{,} \PY{n}{expand}\PY{o}{=}\PY{k+kc}{True}\PY{p}{)}

\PY{c+c1}{\PYZsh{} drops unnecessary columns}
\PY{n}{day\PYZus{}price\PYZus{}df}\PY{o}{.}\PY{n}{drop}\PY{p}{(}\PY{n}{columns}\PY{o}{=}\PY{p}{[}\PY{l+s+s1}{\PYZsq{}}\PY{l+s+s1}{MTU (CET/CEST)}\PY{l+s+s1}{\PYZsq{}}\PY{p}{,} \PY{l+s+s1}{\PYZsq{}}\PY{l+s+s1}{BZN|DE\PYZhy{}LU}\PY{l+s+s1}{\PYZsq{}}\PY{p}{,} \PY{l+s+s1}{\PYZsq{}}\PY{l+s+s1}{Currency}\PY{l+s+s1}{\PYZsq{}}\PY{p}{,} \PY{l+s+s1}{\PYZsq{}}\PY{l+s+s1}{end\PYZus{}time}\PY{l+s+s1}{\PYZsq{}}\PY{p}{]}
                \PY{p}{,} \PY{n}{inplace}\PY{o}{=}\PY{k+kc}{True}\PY{p}{)} \PY{c+c1}{\PYZsh{} assumes all currency same in same region}

\PY{n}{day\PYZus{}price\PYZus{}df}\PY{o}{.}\PY{n}{rename}\PY{p}{(}\PY{n}{columns}\PY{o}{=}\PY{p}{\PYZob{}}
    \PY{l+s+s1}{\PYZsq{}}\PY{l+s+s1}{Day\PYZhy{}ahead Price [EUR/MWh]}\PY{l+s+s1}{\PYZsq{}}\PY{p}{:} \PY{l+s+s1}{\PYZsq{}}\PY{l+s+s1}{price}\PY{l+s+s1}{\PYZsq{}}
\PY{p}{\PYZcb{}}\PY{p}{,} \PY{n}{inplace}\PY{o}{=}\PY{k+kc}{True}\PY{p}{)}

\PY{n}{day\PYZus{}price\PYZus{}df}\PY{p}{[}\PY{l+s+s1}{\PYZsq{}}\PY{l+s+s1}{start\PYZus{}time}\PY{l+s+s1}{\PYZsq{}}\PY{p}{]} \PY{o}{=} \PY{n}{pd}\PY{o}{.}\PY{n}{to\PYZus{}datetime}\PY{p}{(}\PY{n}{day\PYZus{}price\PYZus{}df}\PY{p}{[}\PY{l+s+s1}{\PYZsq{}}\PY{l+s+s1}{start\PYZus{}time}\PY{l+s+s1}{\PYZsq{}}\PY{p}{]}\PY{p}{,}
                                            \PY{n+nb}{format}\PY{o}{=}\PY{l+s+s1}{\PYZsq{}}\PY{l+s+si}{\PYZpc{}d}\PY{l+s+s1}{.}\PY{l+s+s1}{\PYZpc{}}\PY{l+s+s1}{m.}\PY{l+s+s1}{\PYZpc{}}\PY{l+s+s1}{Y }\PY{l+s+s1}{\PYZpc{}}\PY{l+s+s1}{H:}\PY{l+s+s1}{\PYZpc{}}\PY{l+s+s1}{M}\PY{l+s+s1}{\PYZsq{}}\PY{p}{)}
\PY{n}{day\PYZus{}price\PYZus{}ts} \PY{o}{=} \PY{n}{day\PYZus{}price\PYZus{}df}\PY{o}{.}\PY{n}{set\PYZus{}index}\PY{p}{(}\PY{l+s+s1}{\PYZsq{}}\PY{l+s+s1}{start\PYZus{}time}\PY{l+s+s1}{\PYZsq{}}\PY{p}{)} 

\PY{n}{px}\PY{o}{.}\PY{n}{line}\PY{p}{(}\PY{n}{day\PYZus{}price\PYZus{}ts}\PY{p}{,} \PY{n}{title}\PY{o}{=}\PY{l+s+s1}{\PYZsq{}}\PY{l+s+s1}{Day\PYZhy{}Ahead Price Time Series (EUR/MWh) \PYZhy{} hourly}\PY{l+s+s1}{\PYZsq{}}\PY{p}{)}
\end{Verbatim}
\end{tcolorbox}

    
    
    \#

Data Cleaning

In this section, we'll focus on data cleaning, specifically checking for
any NaN values. If any are found, we'll examine the corresponding dates
and determine whether we can fill the missing value using interpolation,
back fill or front fill.

    \begin{tcolorbox}[breakable, size=fbox, boxrule=1pt, pad at break*=1mm,colback=cellbackground, colframe=cellborder]
\prompt{In}{incolor}{3}{\boxspacing}
\begin{Verbatim}[commandchars=\\\{\}]
\PY{n}{idx\PYZus{}nan} \PY{o}{=} \PY{n}{day\PYZus{}price\PYZus{}df}\PY{p}{[}\PY{l+s+s1}{\PYZsq{}}\PY{l+s+s1}{price}\PY{l+s+s1}{\PYZsq{}}\PY{p}{]}\PY{o}{.}\PY{n}{isna}\PY{p}{(}\PY{p}{)}
\PY{n+nb}{print}\PY{p}{(}\PY{l+s+sa}{f}\PY{l+s+s2}{\PYZdq{}}\PY{l+s+s2}{There exists }\PY{l+s+si}{\PYZob{}}\PY{n}{idx\PYZus{}nan}\PY{o}{.}\PY{n}{sum}\PY{p}{(}\PY{p}{)}\PY{l+s+si}{\PYZcb{}}\PY{l+s+s2}{ missing values}\PY{l+s+s2}{\PYZdq{}}\PY{p}{)}

\PY{c+c1}{\PYZsh{} prints missing value with 5 rows before and after}
\PY{n}{day\PYZus{}price\PYZus{}df}\PY{p}{[}\PY{n}{idx\PYZus{}nan}\PY{p}{]}
\PY{n}{missing\PYZus{}row} \PY{o}{=} \PY{n}{du}\PY{o}{.}\PY{n}{show\PYZus{}nrows\PYZus{}around\PYZus{}target}\PY{p}{(}\PY{n}{day\PYZus{}price\PYZus{}df}\PY{p}{,} \PY{l+s+s1}{\PYZsq{}}\PY{l+s+s1}{price}\PY{l+s+s1}{\PYZsq{}}\PY{p}{)}
\PY{n}{missing\PYZus{}row} \PY{o}{=} \PY{n}{pd}\PY{o}{.}\PY{n}{DataFrame}\PY{p}{(}\PY{n}{missing\PYZus{}row}\PY{p}{)}\PY{o}{.}\PY{n}{set\PYZus{}index}\PY{p}{(}\PY{l+s+s1}{\PYZsq{}}\PY{l+s+s1}{start\PYZus{}time}\PY{l+s+s1}{\PYZsq{}}\PY{p}{)}
\PY{n}{px}\PY{o}{.}\PY{n}{line}\PY{p}{(}\PY{n}{missing\PYZus{}row}\PY{p}{,} \PY{n}{title}\PY{o}{=}\PY{l+s+s2}{\PYZdq{}}\PY{l+s+s2}{Shows missing value(s)}\PY{l+s+s2}{\PYZdq{}}\PY{p}{)}
\end{Verbatim}
\end{tcolorbox}

    \begin{Verbatim}[commandchars=\\\{\}]
There exists 1 missing values
    \end{Verbatim}

    
    
    We identified one missing value occurring at 2AM CET. Given the time,
it's unlikely to be an outlier or exhibit extreme variation, so it's
reasonable to interpolate this value.

    \begin{tcolorbox}[breakable, size=fbox, boxrule=1pt, pad at break*=1mm,colback=cellbackground, colframe=cellborder]
\prompt{In}{incolor}{ }{\boxspacing}
\begin{Verbatim}[commandchars=\\\{\}]
\PY{n}{day\PYZus{}price\PYZus{}ts}\PY{p}{[}\PY{l+s+s1}{\PYZsq{}}\PY{l+s+s1}{price\PYZus{}interpolated}\PY{l+s+s1}{\PYZsq{}}\PY{p}{]} \PY{o}{=} \PY{n}{day\PYZus{}price\PYZus{}ts}\PY{p}{[}\PY{l+s+s1}{\PYZsq{}}\PY{l+s+s1}{price}\PY{l+s+s1}{\PYZsq{}}\PY{p}{]}\PY{o}{.}\PY{n}{interpolate}\PY{p}{(}\PY{n}{method}\PY{o}{=}\PY{l+s+s1}{\PYZsq{}}\PY{l+s+s1}{time}\PY{l+s+s1}{\PYZsq{}}\PY{p}{)}

\PY{c+c1}{\PYZsh{} Analysis}
\PY{n}{start\PYZus{}date} \PY{o}{=} \PY{n}{datetime}\PY{p}{(}\PY{l+m+mi}{2022}\PY{p}{,} \PY{l+m+mi}{3}\PY{p}{,} \PY{l+m+mi}{26}\PY{p}{,} \PY{l+m+mi}{12}\PY{p}{)}
\PY{n}{end\PYZus{}date} \PY{o}{=} \PY{n}{datetime}\PY{p}{(}\PY{l+m+mi}{2022}\PY{p}{,} \PY{l+m+mi}{3}\PY{p}{,} \PY{l+m+mi}{27}\PY{p}{,} \PY{l+m+mi}{12}\PY{p}{)}
\PY{n}{filtered\PYZus{}data} \PY{o}{=} \PY{n}{day\PYZus{}price\PYZus{}ts}\PY{p}{[}\PY{p}{[}\PY{l+s+s1}{\PYZsq{}}\PY{l+s+s1}{price}\PY{l+s+s1}{\PYZsq{}}\PY{p}{,} \PY{l+s+s1}{\PYZsq{}}\PY{l+s+s1}{price\PYZus{}interpolated}\PY{l+s+s1}{\PYZsq{}}\PY{p}{]}\PY{p}{]}\PY{o}{.}\PY{n}{loc}\PY{p}{[}
    \PY{p}{(}\PY{n}{day\PYZus{}price\PYZus{}ts}\PY{o}{.}\PY{n}{index} \PY{o}{\PYZgt{}}\PY{o}{=} \PY{n}{start\PYZus{}date}\PY{p}{)} \PY{o}{\PYZam{}} \PY{p}{(}\PY{n}{day\PYZus{}price\PYZus{}ts}\PY{o}{.}\PY{n}{index} \PY{o}{\PYZlt{}} \PY{n}{end\PYZus{}date}\PY{p}{)}
\PY{p}{]}
\PY{n}{plt}\PY{o}{.}\PY{n}{figure}\PY{p}{(}\PY{n}{figsize}\PY{o}{=}\PY{p}{(}\PY{l+m+mi}{12}\PY{p}{,} \PY{l+m+mi}{6}\PY{p}{)}\PY{p}{)}
\PY{n}{plt}\PY{o}{.}\PY{n}{plot}\PY{p}{(}\PY{n}{filtered\PYZus{}data}\PY{o}{.}\PY{n}{index}\PY{p}{,} \PY{n}{filtered\PYZus{}data}\PY{p}{[}\PY{l+s+s1}{\PYZsq{}}\PY{l+s+s1}{price}\PY{l+s+s1}{\PYZsq{}}\PY{p}{]}\PY{p}{,} \PY{l+s+s1}{\PYZsq{}}\PY{l+s+s1}{b\PYZhy{}o}\PY{l+s+s1}{\PYZsq{}}\PY{p}{,} \PY{n}{label}\PY{o}{=}\PY{l+s+s1}{\PYZsq{}}\PY{l+s+s1}{Original DAA Price}\PY{l+s+s1}{\PYZsq{}}\PY{p}{,} \PY{n}{markersize}\PY{o}{=}\PY{l+m+mi}{4}\PY{p}{)}
\PY{n}{plt}\PY{o}{.}\PY{n}{plot}\PY{p}{(}\PY{n}{filtered\PYZus{}data}\PY{o}{.}\PY{n}{index}\PY{p}{,} \PY{n}{filtered\PYZus{}data}\PY{p}{[}\PY{l+s+s1}{\PYZsq{}}\PY{l+s+s1}{price\PYZus{}interpolated}\PY{l+s+s1}{\PYZsq{}}\PY{p}{]}\PY{p}{,} \PY{l+s+s1}{\PYZsq{}}\PY{l+s+s1}{r\PYZhy{}}\PY{l+s+s1}{\PYZsq{}}\PY{p}{,} \PY{n}{label}\PY{o}{=}\PY{l+s+s1}{\PYZsq{}}\PY{l+s+s1}{Interpolated DAA Price}\PY{l+s+s1}{\PYZsq{}}\PY{p}{)}
\PY{n}{plt}\PY{o}{.}\PY{n}{title}\PY{p}{(}\PY{l+s+s1}{\PYZsq{}}\PY{l+s+s1}{Original vs Interpolated DAA Price}\PY{l+s+s1}{\PYZsq{}}\PY{p}{)}
\PY{n}{plt}\PY{o}{.}\PY{n}{xlabel}\PY{p}{(}\PY{l+s+s1}{\PYZsq{}}\PY{l+s+s1}{Time}\PY{l+s+s1}{\PYZsq{}}\PY{p}{)}
\PY{n}{plt}\PY{o}{.}\PY{n}{ylabel}\PY{p}{(}\PY{l+s+s1}{\PYZsq{}}\PY{l+s+s1}{DAA Price (EUR/MWh)}\PY{l+s+s1}{\PYZsq{}}\PY{p}{)}
\PY{n}{plt}\PY{o}{.}\PY{n}{legend}\PY{p}{(}\PY{p}{)}
\PY{n}{plt}\PY{o}{.}\PY{n}{grid}\PY{p}{(}\PY{k+kc}{True}\PY{p}{,} \PY{n}{linestyle}\PY{o}{=}\PY{l+s+s1}{\PYZsq{}}\PY{l+s+s1}{\PYZhy{}\PYZhy{}}\PY{l+s+s1}{\PYZsq{}}\PY{p}{,} \PY{n}{alpha}\PY{o}{=}\PY{l+m+mf}{0.7}\PY{p}{)}

\PY{n}{vertical\PYZus{}line\PYZus{}time} \PY{o}{=} \PY{n}{datetime}\PY{p}{(}\PY{l+m+mi}{2022}\PY{p}{,} \PY{l+m+mi}{3}\PY{p}{,} \PY{l+m+mi}{27}\PY{p}{,} \PY{l+m+mi}{2}\PY{p}{)}  \PY{c+c1}{\PYZsh{} vertical line for missing value}
\PY{n}{plt}\PY{o}{.}\PY{n}{axvline}\PY{p}{(}\PY{n}{x}\PY{o}{=}\PY{n}{vertical\PYZus{}line\PYZus{}time}\PY{p}{,} \PY{n}{color}\PY{o}{=}\PY{l+s+s1}{\PYZsq{}}\PY{l+s+s1}{green}\PY{l+s+s1}{\PYZsq{}}\PY{p}{,} \PY{n}{linestyle}\PY{o}{=}\PY{l+s+s1}{\PYZsq{}}\PY{l+s+s1}{\PYZhy{}\PYZhy{}}\PY{l+s+s1}{\PYZsq{}}\PY{p}{,} \PY{n}{label}\PY{o}{=}\PY{l+s+s1}{\PYZsq{}}\PY{l+s+s1}{02:00}\PY{l+s+s1}{\PYZsq{}}\PY{p}{)}

\PY{n}{plt}\PY{o}{.}\PY{n}{gcf}\PY{p}{(}\PY{p}{)}\PY{o}{.}\PY{n}{autofmt\PYZus{}xdate}\PY{p}{(}\PY{p}{)}
\PY{n}{plt}\PY{o}{.}\PY{n}{tight\PYZus{}layout}\PY{p}{(}\PY{p}{)}
\PY{n}{plt}\PY{o}{.}\PY{n}{show}\PY{p}{(}\PY{p}{)}
\end{Verbatim}
\end{tcolorbox}

    \begin{center}
    \adjustimage{max size={0.9\linewidth}{0.9\paperheight}}{submission_files/submission_8_0.png}
    \end{center}
    { \hspace*{\fill} \\}
    
    \begin{tcolorbox}[breakable, size=fbox, boxrule=1pt, pad at break*=1mm,colback=cellbackground, colframe=cellborder]
\prompt{In}{incolor}{5}{\boxspacing}
\begin{Verbatim}[commandchars=\\\{\}]
\PY{c+c1}{\PYZsh{} replaces interpolated price with original price}
\PY{n}{day\PYZus{}price\PYZus{}ts}\PY{p}{[}\PY{l+s+s1}{\PYZsq{}}\PY{l+s+s1}{price}\PY{l+s+s1}{\PYZsq{}}\PY{p}{]} \PY{o}{=} \PY{n}{day\PYZus{}price\PYZus{}ts}\PY{p}{[}\PY{l+s+s1}{\PYZsq{}}\PY{l+s+s1}{price\PYZus{}interpolated}\PY{l+s+s1}{\PYZsq{}}\PY{p}{]}
\PY{n}{day\PYZus{}price\PYZus{}ts}\PY{o}{.}\PY{n}{drop}\PY{p}{(}\PY{n}{columns}\PY{o}{=}\PY{p}{[}\PY{l+s+s1}{\PYZsq{}}\PY{l+s+s1}{price\PYZus{}interpolated}\PY{l+s+s1}{\PYZsq{}}\PY{p}{]}\PY{p}{,} \PY{n}{inplace}\PY{o}{=}\PY{k+kc}{True}\PY{p}{)}
\PY{n}{day\PYZus{}price\PYZus{}ts}\PY{o}{.}\PY{n}{head}\PY{p}{(}\PY{p}{)}
\end{Verbatim}
\end{tcolorbox}

            \begin{tcolorbox}[breakable, size=fbox, boxrule=.5pt, pad at break*=1mm, opacityfill=0]
\prompt{Out}{outcolor}{5}{\boxspacing}
\begin{Verbatim}[commandchars=\\\{\}]
                     price
start\_time
2022-01-01 00:00:00  50.05
2022-01-01 01:00:00  41.33
2022-01-01 02:00:00  43.22
2022-01-01 03:00:00  45.46
2022-01-01 04:00:00  37.67
\end{Verbatim}
\end{tcolorbox}
        
    \#

Preliminary Analysis

Now that we have a complete time series dataset, we can perform a
preliminary analysis to inspect the time series data

    \begin{tcolorbox}[breakable, size=fbox, boxrule=1pt, pad at break*=1mm,colback=cellbackground, colframe=cellborder]
\prompt{In}{incolor}{6}{\boxspacing}
\begin{Verbatim}[commandchars=\\\{\}]
\PY{c+c1}{\PYZsh{} Seasonal decomposition}

\PY{n}{seasonal\PYZus{}decompose}\PY{p}{(}\PY{n}{day\PYZus{}price\PYZus{}ts}\PY{p}{[}\PY{l+s+s1}{\PYZsq{}}\PY{l+s+s1}{price}\PY{l+s+s1}{\PYZsq{}}\PY{p}{]}\PY{p}{)}\PY{o}{.}\PY{n}{plot}\PY{p}{(}\PY{p}{)}
\PY{n}{plt}\PY{o}{.}\PY{n}{show}\PY{p}{(}\PY{p}{)}
\end{Verbatim}
\end{tcolorbox}

    \begin{center}
    \adjustimage{max size={0.9\linewidth}{0.9\paperheight}}{submission_files/submission_11_0.png}
    \end{center}
    { \hspace*{\fill} \\}
    
    The seasonal decomposition of the price time series reveals a clear
upward trend with a noticeable disturbance around early March 2022, this
is likely caused by \textbf{Russia invasion of Ukraine}. Furthermore,
towards the end, there appears to be slight upward trend.

The seasonal component shows regular periodic behavior, likely due to
daily or hourly cycles.

The residuals remain relatively stable throughout most of the period,
there is increased variability and notable outliers around the same
March period, aligning with the observed spike.

This suggests that perhaps one may use ARIMA or ETS to model the time
series.

    ARIMA

ARIMA(p,d,q) is a combinaation of Autogressive model AR(p), Moving
Average model MA(q) with differenciating parameter d where
\((p,d,q) \in \mathbb{N} \times \mathbb{N} \times \mathbb{N}\). Given a
time series \(Y_t\) for all \(t\in(t,\dots,T)\), the models can be
defined as follows

\begin{itemize}
\tightlist
\item
  Autoregressive, AR(p) \[ 
    Y_t = c + \sum_{i=1}^{p} \phi_i Y_{t-i} + \epsilon_t 
    \],
\end{itemize}

where \(c, \phi \in \mathbb{R}\) and \(\epsilon_t\) is residual
distributed by \(N(0, \sigma^2)\)

\begin{itemize}
\tightlist
\item
  Moving Average: MA(q)
\end{itemize}

\[ Y_t = c + \epsilon_t + \sum_{i=1}^{q} \theta_i \epsilon_{t-i} \]

\begin{itemize}
\tightlist
\item
  Differencing term
\end{itemize}

\[ Y_t^d := (1-B)^d Y_t,\]

where \(B\) is backward operator, i.e.~\(BY_t = Y_{t-1}\).

Thereforem, ARIMA(p,d,q) is modeled as

\[
    Y_t^d = c + \sum_{t=1}^{p} \phi_i Y_{t-i}^d + \sum_{t=1}^{q} \theta_i \epsilon_{t-i} + \epsilon_t,
\] where c is some constant and \(\phi_t\) and \(\theta_t\) are
parameters to be calculated.

{\textbf{Assumptions}}: - \(Y_t^d\) is stationary or has been
differenced to achieve stationarity, i.e.~constant mean, variance, and
autocorrelation, - Linear relationship, - \(\epsilon_t\) is residual
distributed by \(\mathcal{N}(0, \sigma^2)\).

\textbf{Pros}

\begin{itemize}
\tightlist
\item
  ARIMA models most of time series data with trends, cycles,
  seasonality, and beyond as long as it's stationary,
\item
  Handles missing values,
\item
  Has good statistical metrics (i.e.~coefficients, std, p-values, AIC)
  for analysis and forecast purposes.
\end{itemize}

\textbf{Cons}

\begin{itemize}
\tightlist
\item
  Need time series to be stationary - if not unable to handle longer
  forecast,
\item
  Cannot hand nonlinear models.
\end{itemize}

    To begin analsysis, we will split that data. First, we will attempt
approx 70/30 split of the time series to evaluate its performance. Then,
we will shorten the test window, as one can see later, ARIMA is unable
to handle long term forecast, but does well in a shorter term e.g.~24 H.

    \begin{tcolorbox}[breakable, size=fbox, boxrule=1pt, pad at break*=1mm,colback=cellbackground, colframe=cellborder]
\prompt{In}{incolor}{7}{\boxspacing}
\begin{Verbatim}[commandchars=\\\{\}]
\PY{n}{split\PYZus{}date} \PY{o}{=} \PY{n}{datetime}\PY{p}{(}\PY{l+m+mi}{2022}\PY{p}{,} \PY{l+m+mi}{6}\PY{p}{,} \PY{l+m+mi}{1}\PY{p}{)}
\PY{n}{arima\PYZus{}obj} \PY{o}{=} \PY{n}{ARIMAWrapper}\PY{p}{(}\PY{n}{day\PYZus{}price\PYZus{}ts}\PY{p}{,} \PY{n}{split\PYZus{}date}\PY{p}{)}

\PY{c+c1}{\PYZsh{} First, we check the stationary test and plot the ACF and PACF for the full time series}
\PY{n}{arima\PYZus{}obj}\PY{o}{.}\PY{n}{plot\PYZus{}acf\PYZus{}pacf}\PY{p}{(}\PY{p}{)} 
\end{Verbatim}
\end{tcolorbox}

    \begin{Verbatim}[commandchars=\\\{\}]
Stationarity test for whole series with 0 lag:
Reject the null hypothesis. The series is likely stationary.

Train-Test split at 2022-06-01 00:00:00
Stationarity test for train series with 0 lag:
Reject the null hypothesis. The series is likely stationary.

    \end{Verbatim}

    \begin{center}
    \adjustimage{max size={0.9\linewidth}{0.9\paperheight}}{submission_files/submission_15_1.png}
    \end{center}
    { \hspace*{\fill} \\}
    
    \subsubsection{Note:}\label{note}

we find the best (p,d,q) parameters for ARIMA using the training data.
We do this by traversing (p,d,q) with different integers and get the one
with the smalles AIC value.

    \begin{tcolorbox}[breakable, size=fbox, boxrule=1pt, pad at break*=1mm,colback=cellbackground, colframe=cellborder]
\prompt{In}{incolor}{8}{\boxspacing}
\begin{Verbatim}[commandchars=\\\{\}]
\PY{c+c1}{\PYZsh{} \PYZpc{}\PYZpc{}capture }
\PY{c+c1}{\PYZsh{} p,d,q = arima\PYZus{}obj.get\PYZus{}best\PYZus{}ARIMA\PYZus{}params()}
\PY{c+c1}{\PYZsh{} print(f\PYZdq{}Best params obtain: ARIMA(\PYZob{}p\PYZcb{}, \PYZob{}d\PYZcb{}, \PYZob{}q\PYZcb{})\PYZdq{})}

\PY{c+c1}{\PYZsh{} \PYZsh{} Best params obtain: ARIMA(5, 1, 3)}
\end{Verbatim}
\end{tcolorbox}

    \begin{tcolorbox}[breakable, size=fbox, boxrule=1pt, pad at break*=1mm,colback=cellbackground, colframe=cellborder]
\prompt{In}{incolor}{9}{\boxspacing}
\begin{Verbatim}[commandchars=\\\{\}]
\PY{n}{arima\PYZus{}obj}\PY{o}{.}\PY{n}{run\PYZus{}ARIMA}\PY{p}{(}\PY{l+m+mi}{5}\PY{p}{,} \PY{l+m+mi}{1}\PY{p}{,} \PY{l+m+mi}{3}\PY{p}{)}
\PY{n}{arima\PYZus{}obj}\PY{o}{.}\PY{n}{get\PYZus{}forecast}\PY{p}{(}\PY{p}{)}
\PY{n}{arima\PYZus{}obj}\PY{o}{.}\PY{n}{plot\PYZus{}forecast}\PY{p}{(}\PY{p}{)}
\PY{n}{forecast\PYZus{}metric\PYZus{}long}\PY{p}{[}\PY{l+s+s1}{\PYZsq{}}\PY{l+s+s1}{ARIMA}\PY{l+s+s1}{\PYZsq{}}\PY{p}{]} \PY{o}{=} \PY{n+nb}{list}\PY{p}{(}\PY{n}{arima\PYZus{}obj}\PY{o}{.}\PY{n}{get\PYZus{}rmse\PYZus{}mae}\PY{p}{(}\PY{p}{)}\PY{p}{)}
\end{Verbatim}
\end{tcolorbox}

    \begin{Verbatim}[commandchars=\\\{\}]
--------------------------------
Running ARIMA(5, 1, 3) on train set
    \end{Verbatim}

    \begin{Verbatim}[commandchars=\\\{\}]
/Users/matthewliew/projects/entrix\_assessment/.venv/lib/python3.12/site-
packages/statsmodels/tsa/base/tsa\_model.py:473: ValueWarning:

No frequency information was provided, so inferred frequency h will be used.

/Users/matthewliew/projects/entrix\_assessment/.venv/lib/python3.12/site-
packages/statsmodels/tsa/base/tsa\_model.py:473: ValueWarning:

No frequency information was provided, so inferred frequency h will be used.

/Users/matthewliew/projects/entrix\_assessment/.venv/lib/python3.12/site-
packages/statsmodels/tsa/base/tsa\_model.py:473: ValueWarning:

No frequency information was provided, so inferred frequency h will be used.

/Users/matthewliew/projects/entrix\_assessment/.venv/lib/python3.12/site-
packages/statsmodels/tsa/statespace/sarimax.py:978: UserWarning:

Non-invertible starting MA parameters found. Using zeros as starting parameters.

/Users/matthewliew/projects/entrix\_assessment/.venv/lib/python3.12/site-
packages/statsmodels/base/model.py:607: ConvergenceWarning:

Maximum Likelihood optimization failed to converge. Check mle\_retvals

/Users/matthewliew/projects/entrix\_assessment/.venv/lib/python3.12/site-
packages/statsmodels/tsa/statespace/sarimax.py:978: UserWarning:

Non-invertible starting MA parameters found. Using zeros as starting parameters.

    \end{Verbatim}

    \begin{Verbatim}[commandchars=\\\{\}]
                               SARIMAX Results
==============================================================================
Dep. Variable:                  price   No. Observations:                 3625
Model:                 ARIMA(5, 1, 3)   Log Likelihood              -15734.424
Date:                Mon, 19 May 2025   AIC                          31486.848
Time:                        00:01:41   BIC                          31542.606
Sample:                    01-01-2022   HQIC                         31506.712
                         - 06-01-2022
Covariance Type:                  opg
==============================================================================
                 coef    std err          z      P>|z|      [0.025      0.975]
------------------------------------------------------------------------------
ar.L1          1.0839      0.088     12.258      0.000       0.911       1.257
ar.L2          0.3083      0.180      1.715      0.086      -0.044       0.661
ar.L3         -1.1098      0.145     -7.635      0.000      -1.395      -0.825
ar.L4          0.4294      0.048      8.860      0.000       0.334       0.524
ar.L5         -0.1406      0.017     -8.129      0.000      -0.175      -0.107
ma.L1         -0.7353      0.088     -8.360      0.000      -0.908      -0.563
ma.L2         -0.7103      0.151     -4.708      0.000      -1.006      -0.415
ma.L3          0.9569      0.086     11.182      0.000       0.789       1.125
sigma2       346.0710      4.652     74.399      0.000     336.954     355.188
================================================================================
===
Ljung-Box (L1) (Q):                   1.48   Jarque-Bera (JB):
4347.15
Prob(Q):                              0.22   Prob(JB):
0.00
Heteroskedasticity (H):               0.63   Skew:
0.24
Prob(H) (two-sided):                  0.00   Kurtosis:
8.34
================================================================================
===

Warnings:
[1] Covariance matrix calculated using the outer product of gradients (complex-
step).
--------------------------------
Forecasting 719 steps ahead
    \end{Verbatim}

    \begin{Verbatim}[commandchars=\\\{\}]
/Users/matthewliew/projects/entrix\_assessment/.venv/lib/python3.12/site-
packages/statsmodels/base/model.py:607: ConvergenceWarning:

Maximum Likelihood optimization failed to converge. Check mle\_retvals

    \end{Verbatim}

    
    
    \begin{Verbatim}[commandchars=\\\{\}]
RSME for ARIMA(5, 1, 3): 80.867
MAE for ARIMA(5, 1, 3): 65.13
    \end{Verbatim}

    \subsubsection{Note}\label{note}

From the analysis above, it is clear that ARIMA is not able to forecast
too far in advance. Therefore, a smart thing to do is to shorten the
forecast period as well as the start datetime of the training set.

In the following, we only want to forecast \textbf{24 H} on the day of
30. June 2022 and we take the training from 60 days prior for training.

    \begin{tcolorbox}[breakable, size=fbox, boxrule=1pt, pad at break*=1mm,colback=cellbackground, colframe=cellborder]
\prompt{In}{incolor}{10}{\boxspacing}
\begin{Verbatim}[commandchars=\\\{\}]
\PY{n}{short\PYZus{}split\PYZus{}date} \PY{o}{=} \PY{n}{datetime}\PY{p}{(}\PY{l+m+mi}{2022}\PY{p}{,} \PY{l+m+mi}{6}\PY{p}{,} \PY{l+m+mi}{29}\PY{p}{,} \PY{l+m+mi}{23}\PY{p}{)} 
\PY{n}{start\PYZus{}slice} \PY{o}{=} \PY{n}{split\PYZus{}date} \PY{o}{\PYZhy{}} \PY{n}{timedelta}\PY{p}{(}\PY{n}{days}\PY{o}{=}\PY{l+m+mi}{60}\PY{p}{)} 

\PY{n}{short\PYZus{}arima} \PY{o}{=} \PY{n}{ARIMAWrapper}\PY{p}{(}\PY{n}{day\PYZus{}price\PYZus{}ts}\PY{p}{,} \PY{n}{short\PYZus{}split\PYZus{}date}\PY{p}{,} \PY{n}{start\PYZus{}date\PYZus{}slice}\PY{o}{=}\PY{n}{start\PYZus{}slice}\PY{p}{)}
\PY{n}{short\PYZus{}arima}\PY{o}{.}\PY{n}{plot\PYZus{}acf\PYZus{}pacf}\PY{p}{(}\PY{p}{)}
\end{Verbatim}
\end{tcolorbox}

    \begin{Verbatim}[commandchars=\\\{\}]
Stationarity test for whole series with 0 lag:
Reject the null hypothesis. The series is likely stationary.

Train-Test split at 2022-06-29 23:00:00
Stationarity test for train series with 0 lag:
Reject the null hypothesis. The series is likely stationary.

    \end{Verbatim}

    \begin{center}
    \adjustimage{max size={0.9\linewidth}{0.9\paperheight}}{submission_files/submission_20_1.png}
    \end{center}
    { \hspace*{\fill} \\}
    
    \begin{tcolorbox}[breakable, size=fbox, boxrule=1pt, pad at break*=1mm,colback=cellbackground, colframe=cellborder]
\prompt{In}{incolor}{11}{\boxspacing}
\begin{Verbatim}[commandchars=\\\{\}]
\PY{c+c1}{\PYZsh{} check stationary test for one lag}

\PY{n}{ts\PYZus{}short} \PY{o}{=} \PY{n}{TimeSeries}\PY{o}{.}\PY{n}{from\PYZus{}dataframe}\PY{p}{(}\PY{n}{short\PYZus{}arima}\PY{o}{.}\PY{n}{ts}\PY{p}{)}
\PY{n}{ts\PYZus{}train\PYZus{}short} \PY{o}{=} \PY{n}{TimeSeries}\PY{o}{.}\PY{n}{from\PYZus{}dataframe}\PY{p}{(}\PY{n}{short\PYZus{}arima}\PY{o}{.}\PY{n}{ts\PYZus{}train}\PY{p}{)}

\PY{n+nb}{print}\PY{p}{(}\PY{n}{stationarity\PYZus{}test\PYZus{}adf}\PY{p}{(}\PY{n}{ts\PYZus{}short}\PY{p}{,} \PY{l+m+mi}{1}\PY{p}{)}\PY{p}{)}
\PY{n+nb}{print}\PY{p}{(}\PY{l+s+s2}{\PYZdq{}}\PY{l+s+s2}{\PYZhy{}\PYZhy{}}\PY{l+s+s2}{\PYZdq{}}\PY{o}{*}\PY{l+m+mi}{20}\PY{p}{)}
\PY{n+nb}{print}\PY{p}{(}\PY{n}{stationarity\PYZus{}test\PYZus{}adf}\PY{p}{(}\PY{n}{ts\PYZus{}train\PYZus{}short}\PY{p}{,} \PY{l+m+mi}{1}\PY{p}{)}\PY{p}{)}

\PY{c+c1}{\PYZsh{} So we should take d = 1}
\end{Verbatim}
\end{tcolorbox}

    \begin{Verbatim}[commandchars=\\\{\}]
(np.float64(-13.969798268748555), np.float64(4.3660887578948263e-26), 1, 2158,
\{'1\%': np.float64(-3.4333838718977896), '5\%': np.float64(-2.8628802551435264),
'10\%': np.float64(-2.567483485481099)\}, np.float64(18841.985495389694))
----------------------------------------
(np.float64(-14.02407015178803), np.float64(3.529014542476664e-26), 1, 2134,
\{'1\%': np.float64(-3.433418033468612), '5\%': np.float64(-2.862895338734693),
'10\%': np.float64(-2.567491516545175)\}, np.float64(18586.923040349237))
    \end{Verbatim}

    \begin{tcolorbox}[breakable, size=fbox, boxrule=1pt, pad at break*=1mm,colback=cellbackground, colframe=cellborder]
\prompt{In}{incolor}{12}{\boxspacing}
\begin{Verbatim}[commandchars=\\\{\}]
\PY{c+c1}{\PYZsh{} \PYZpc{}\PYZpc{}capture }
\PY{c+c1}{\PYZsh{} p,d,q = short\PYZus{}arima.get\PYZus{}best\PYZus{}ARIMA\PYZus{}params()}

\PY{c+c1}{\PYZsh{} print(f\PYZdq{}Best params obtain: ARIMA(\PYZob{}p\PYZcb{}, \PYZob{}d\PYZcb{}, \PYZob{}q\PYZcb{})\PYZdq{})}

\PY{c+c1}{\PYZsh{} \PYZsh{} Best params obtain: ARIMA(3, 1, 4)}
\end{Verbatim}
\end{tcolorbox}

    \begin{tcolorbox}[breakable, size=fbox, boxrule=1pt, pad at break*=1mm,colback=cellbackground, colframe=cellborder]
\prompt{In}{incolor}{13}{\boxspacing}
\begin{Verbatim}[commandchars=\\\{\}]
\PY{n}{short\PYZus{}arima}\PY{o}{.}\PY{n}{run\PYZus{}ARIMA}\PY{p}{(}\PY{l+m+mi}{3}\PY{p}{,} \PY{l+m+mi}{1}\PY{p}{,} \PY{l+m+mi}{4}\PY{p}{)}
\PY{n}{short\PYZus{}arima}\PY{o}{.}\PY{n}{get\PYZus{}forecast}\PY{p}{(}\PY{p}{)}
\PY{n}{short\PYZus{}arima}\PY{o}{.}\PY{n}{plot\PYZus{}forecast}\PY{p}{(}\PY{p}{)}
\PY{n}{forecast\PYZus{}metric\PYZus{}short}\PY{p}{[}\PY{l+s+s1}{\PYZsq{}}\PY{l+s+s1}{ARIMA}\PY{l+s+s1}{\PYZsq{}}\PY{p}{]} \PY{o}{=} \PY{n+nb}{list}\PY{p}{(}\PY{n}{short\PYZus{}arima}\PY{o}{.}\PY{n}{get\PYZus{}rmse\PYZus{}mae}\PY{p}{(}\PY{p}{)}\PY{p}{)}
\end{Verbatim}
\end{tcolorbox}

    \begin{Verbatim}[commandchars=\\\{\}]
--------------------------------
Running ARIMA(3, 1, 4) on train set
    \end{Verbatim}

    \begin{Verbatim}[commandchars=\\\{\}]
/Users/matthewliew/projects/entrix\_assessment/.venv/lib/python3.12/site-
packages/statsmodels/tsa/base/tsa\_model.py:473: ValueWarning:

No frequency information was provided, so inferred frequency h will be used.

/Users/matthewliew/projects/entrix\_assessment/.venv/lib/python3.12/site-
packages/statsmodels/tsa/base/tsa\_model.py:473: ValueWarning:

No frequency information was provided, so inferred frequency h will be used.

/Users/matthewliew/projects/entrix\_assessment/.venv/lib/python3.12/site-
packages/statsmodels/tsa/base/tsa\_model.py:473: ValueWarning:

No frequency information was provided, so inferred frequency h will be used.

/Users/matthewliew/projects/entrix\_assessment/.venv/lib/python3.12/site-
packages/statsmodels/base/model.py:607: ConvergenceWarning:

Maximum Likelihood optimization failed to converge. Check mle\_retvals

    \end{Verbatim}

    \begin{Verbatim}[commandchars=\\\{\}]
                               SARIMAX Results
==============================================================================
Dep. Variable:                  price   No. Observations:                 2136
Model:                 ARIMA(3, 1, 4)   Log Likelihood               -9030.666
Date:                Mon, 19 May 2025   AIC                          18077.333
Time:                        00:01:43   BIC                          18122.662
Sample:                    04-02-2022   HQIC                         18093.922
                         - 06-29-2022
Covariance Type:                  opg
==============================================================================
                 coef    std err          z      P>|z|      [0.025      0.975]
------------------------------------------------------------------------------
ar.L1          0.7415      0.044     16.724      0.000       0.655       0.828
ar.L2          0.7156      0.077      9.321      0.000       0.565       0.866
ar.L3         -0.9905      0.044    -22.344      0.000      -1.077      -0.904
ma.L1         -0.4522      0.047     -9.711      0.000      -0.544      -0.361
ma.L2         -0.9206      0.068    -13.585      0.000      -1.053      -0.788
ma.L3          0.7721      0.030     26.109      0.000       0.714       0.830
ma.L4          0.2691      0.017     15.945      0.000       0.236       0.302
sigma2       275.0794      4.665     58.971      0.000     265.937     284.222
================================================================================
===
Ljung-Box (L1) (Q):                   0.40   Jarque-Bera (JB):
7667.44
Prob(Q):                              0.53   Prob(JB):
0.00
Heteroskedasticity (H):               1.77   Skew:
0.01
Prob(H) (two-sided):                  0.00   Kurtosis:
12.28
================================================================================
===

Warnings:
[1] Covariance matrix calculated using the outer product of gradients (complex-
step).
--------------------------------
Forecasting 24 steps ahead
    \end{Verbatim}

    \begin{Verbatim}[commandchars=\\\{\}]
/Users/matthewliew/projects/entrix\_assessment/.venv/lib/python3.12/site-
packages/statsmodels/base/model.py:607: ConvergenceWarning:

Maximum Likelihood optimization failed to converge. Check mle\_retvals

    \end{Verbatim}

    
    
    \begin{Verbatim}[commandchars=\\\{\}]
RSME for ARIMA(3, 1, 4): 68.445
MAE for ARIMA(3, 1, 4): 57.871
    \end{Verbatim}

    \paragraph{Backtesting (not implemented, no
time)}\label{backtesting-not-implemented-no-time}

    Prophet

A forcasting tool by Facebook:

\[
Y_t = g_t + s_t + h_t + \epsilon_t,
\]

where - \(Y_t\) is the predicted value, - \(g_t\) trend function, -
\(s_t\) weekly/year seasonality, - \(h_t\) represents the effects of
holidays or any other irregular days, - \(\epsilon_t\) represents the
error term.

\textbf{Pros}

\begin{itemize}
\tightlist
\item
  Easy ``out of the box'' solution,
\item
  Detects trend automatically,
\item
  Can handle missing data and outliers.
\end{itemize}

\textbf{Cons} - Black box, - Difficulty to tune hyperparameters.

    \begin{tcolorbox}[breakable, size=fbox, boxrule=1pt, pad at break*=1mm,colback=cellbackground, colframe=cellborder]
\prompt{In}{incolor}{14}{\boxspacing}
\begin{Verbatim}[commandchars=\\\{\}]
\PY{c+c1}{\PYZsh{} Reuse the data and split\PYZus{}date from ARIMA for longer window forecast}

\PY{n}{prophet\PYZus{}obj} \PY{o}{=} \PY{n}{ProphetWrapper}\PY{p}{(}
                \PY{n}{day\PYZus{}price\PYZus{}ts}\PY{p}{,}
                \PY{n}{split\PYZus{}date}\PY{p}{,}
                \PY{p}{)}
\PY{n}{prophet\PYZus{}obj}\PY{o}{.}\PY{n}{run\PYZus{}prophet}\PY{p}{(}\PY{p}{)}
\PY{n}{prophet\PYZus{}forecast} \PY{o}{=} \PY{n}{prophet\PYZus{}obj}\PY{o}{.}\PY{n}{get\PYZus{}forecast}\PY{p}{(}\PY{p}{)}
\end{Verbatim}
\end{tcolorbox}

    \begin{Verbatim}[commandchars=\\\{\}]
00:01:44 - cmdstanpy - INFO - Chain [1] start processing
    \end{Verbatim}

    \begin{Verbatim}[commandchars=\\\{\}]
Train-Test split at 2022-06-01 00:00:00
    \end{Verbatim}

    \begin{Verbatim}[commandchars=\\\{\}]
00:01:45 - cmdstanpy - INFO - Chain [1] done processing
    \end{Verbatim}

    \begin{Verbatim}[commandchars=\\\{\}]
Prophet model fitted. Use get\_forecast() to get forecast.
    \end{Verbatim}

    \begin{tcolorbox}[breakable, size=fbox, boxrule=1pt, pad at break*=1mm,colback=cellbackground, colframe=cellborder]
\prompt{In}{incolor}{15}{\boxspacing}
\begin{Verbatim}[commandchars=\\\{\}]
\PY{n}{prophet\PYZus{}obj}\PY{o}{.}\PY{n}{prophet\PYZus{}model}\PY{o}{.}\PY{n}{plot\PYZus{}components}\PY{p}{(}\PY{n}{prophet\PYZus{}forecast}\PY{p}{)}
\PY{n}{plt}\PY{o}{.}\PY{n}{show}\PY{p}{(}\PY{p}{)}
\PY{n}{prophet\PYZus{}obj}\PY{o}{.}\PY{n}{plot\PYZus{}forecast}\PY{p}{(}\PY{p}{)}
\end{Verbatim}
\end{tcolorbox}

    \begin{center}
    \adjustimage{max size={0.9\linewidth}{0.9\paperheight}}{submission_files/submission_27_0.png}
    \end{center}
    { \hspace*{\fill} \\}
    
    
    
    \begin{tcolorbox}[breakable, size=fbox, boxrule=1pt, pad at break*=1mm,colback=cellbackground, colframe=cellborder]
\prompt{In}{incolor}{16}{\boxspacing}
\begin{Verbatim}[commandchars=\\\{\}]
\PY{n}{forecast\PYZus{}metric\PYZus{}long}\PY{p}{[}\PY{l+s+s1}{\PYZsq{}}\PY{l+s+s1}{Prophet}\PY{l+s+s1}{\PYZsq{}}\PY{p}{]} \PY{o}{=} \PY{n+nb}{list}\PY{p}{(}\PY{n}{prophet\PYZus{}obj}\PY{o}{.}\PY{n}{get\PYZus{}rmse\PYZus{}mae}\PY{p}{(}\PY{p}{)}\PY{p}{)}
\end{Verbatim}
\end{tcolorbox}

    \begin{Verbatim}[commandchars=\\\{\}]
RSME for Prophet: 139.455
MAE for Prophet: 114.839
    \end{Verbatim}

    \paragraph{\texorpdfstring{\textbf{Shorter training window and forecast
only 24
H}}{Shorter training window and forecast only 24 H}}\label{shorter-training-window-and-forecast-only-24-h}

Like ARIMA, we also observed that the Prophet model does not do so well
in a longer term in our data possibily due to non-starionarity.
Therefore, we can inspect its efficacy in the short time window: 24H

    \begin{tcolorbox}[breakable, size=fbox, boxrule=1pt, pad at break*=1mm,colback=cellbackground, colframe=cellborder]
\prompt{In}{incolor}{17}{\boxspacing}
\begin{Verbatim}[commandchars=\\\{\}]
\PY{n}{short\PYZus{}date\PYZus{}24H} \PY{o}{=} \PY{n}{datetime}\PY{p}{(}\PY{l+m+mi}{2022}\PY{p}{,} \PY{l+m+mi}{6}\PY{p}{,} \PY{l+m+mi}{29}\PY{p}{,} \PY{l+m+mi}{23}\PY{p}{)}
\PY{n}{start\PYZus{}slice} \PY{o}{=} \PY{n}{split\PYZus{}date} \PY{o}{\PYZhy{}} \PY{n}{timedelta}\PY{p}{(}\PY{n}{days}\PY{o}{=}\PY{l+m+mi}{60}\PY{p}{)} 

\PY{n}{short\PYZus{}prophet\PYZus{}obj} \PY{o}{=} \PY{n}{ProphetWrapper}\PY{p}{(}
                \PY{n}{day\PYZus{}price\PYZus{}ts}\PY{p}{,}
                \PY{n}{short\PYZus{}split\PYZus{}date}\PY{p}{,}
                \PY{n}{start\PYZus{}date\PYZus{}slice}\PY{o}{=}\PY{n}{start\PYZus{}slice}
                \PY{p}{)}
\PY{n}{short\PYZus{}prophet\PYZus{}obj}\PY{o}{.}\PY{n}{run\PYZus{}prophet}\PY{p}{(}\PY{p}{)}
\PY{n}{short\PYZus{}prophet\PYZus{}forecast} \PY{o}{=} \PY{n}{short\PYZus{}prophet\PYZus{}obj}\PY{o}{.}\PY{n}{get\PYZus{}forecast}\PY{p}{(}\PY{p}{)}
\PY{n}{short\PYZus{}prophet\PYZus{}obj}\PY{o}{.}\PY{n}{prophet\PYZus{}model}\PY{o}{.}\PY{n}{plot\PYZus{}components}\PY{p}{(}\PY{n}{prophet\PYZus{}forecast}\PY{p}{)}
\PY{n}{plt}\PY{o}{.}\PY{n}{show}\PY{p}{(}\PY{p}{)}
\PY{n}{short\PYZus{}prophet\PYZus{}obj}\PY{o}{.}\PY{n}{plot\PYZus{}forecast}\PY{p}{(}\PY{p}{)}
\end{Verbatim}
\end{tcolorbox}

    \begin{Verbatim}[commandchars=\\\{\}]
00:01:45 - cmdstanpy - INFO - Chain [1] start processing
00:01:45 - cmdstanpy - INFO - Chain [1] done processing
    \end{Verbatim}

    \begin{Verbatim}[commandchars=\\\{\}]
Train-Test split at 2022-06-29 23:00:00
Prophet model fitted. Use get\_forecast() to get forecast.
    \end{Verbatim}

    \begin{center}
    \adjustimage{max size={0.9\linewidth}{0.9\paperheight}}{submission_files/submission_30_2.png}
    \end{center}
    { \hspace*{\fill} \\}
    
    
    
    \begin{tcolorbox}[breakable, size=fbox, boxrule=1pt, pad at break*=1mm,colback=cellbackground, colframe=cellborder]
\prompt{In}{incolor}{18}{\boxspacing}
\begin{Verbatim}[commandchars=\\\{\}]
\PY{n+nb}{print}\PY{p}{(}\PY{l+s+s2}{\PYZdq{}}\PY{l+s+s2}{RSME and MAE for short term Prophet:}\PY{l+s+s2}{\PYZdq{}}\PY{p}{)}
\PY{n}{forecast\PYZus{}metric\PYZus{}short}\PY{p}{[}\PY{l+s+s1}{\PYZsq{}}\PY{l+s+s1}{Prophet}\PY{l+s+s1}{\PYZsq{}}\PY{p}{]} \PY{o}{=} \PY{n+nb}{list}\PY{p}{(}\PY{n}{short\PYZus{}prophet\PYZus{}obj}\PY{o}{.}\PY{n}{get\PYZus{}rmse\PYZus{}mae}\PY{p}{(}\PY{p}{)}\PY{p}{)}
\end{Verbatim}
\end{tcolorbox}

    \begin{Verbatim}[commandchars=\\\{\}]
RSME and MAE for short term Prophet:
RSME for Prophet: 39.688
MAE for Prophet: 30.529
    \end{Verbatim}

    XGBoost

``Extreme Gradient Boosting'' (XGBoost) is used for supervised learning
problems, where we use the training data (with multiple features) to
predict a target variable, here the DAA Price. The model uses gradient
boosting to build ensemble of decision trees where at each tree aims to
minimize the error made by previous tree.

{\textbf{Assumptions}} As the model is nonparametric, it can capture
complex relationship in the data. Therefore the assumptions here are
made as we explains features we are going to use.

\textbf{Pros} - Better performance as we will see later than ARIMA and
Prophet, - It is easy to adjust the features to be selected and XGBoost
package will generate which feature is the most important, explaining a
certain aspect of the modeling.

**Cons* - overfitting could happen, especially with high n-estimators.
However, since for this usecase, this is not much of a concern, - the
trees generated is hugh, it is not easy to have a complete picture of
the forecasting decision compared to, for e.g., ARIMA where you can see
the coefficient, - estimators might not be the same for each run unless
seed is set.

    \subsection{\texorpdfstring{{\textbf{Feature
Selection}}}{Feature Selection}}\label{feature-selection}

We loaded an hourly time-series with DAA price per MWh. We start feature
engineering by separating the the datetime into its atomic metrics,
i.e.~month of the year, week of the year, day of the week, and the hour
of day. On top of that, we will also extract boolean features such as
is\_weekend, is\_holiday.

We will do this by calling \texttt{add\_date\_features()} function.

    \begin{tcolorbox}[breakable, size=fbox, boxrule=1pt, pad at break*=1mm,colback=cellbackground, colframe=cellborder]
\prompt{In}{incolor}{19}{\boxspacing}
\begin{Verbatim}[commandchars=\\\{\}]
\PY{n}{xgb\PYZus{}day\PYZus{}price\PYZus{}ts} \PY{o}{=} \PY{n}{du}\PY{o}{.}\PY{n}{add\PYZus{}date\PYZus{}features}\PY{p}{(}\PY{n}{day\PYZus{}price\PYZus{}ts}\PY{o}{.}\PY{n}{copy}\PY{p}{(}\PY{p}{)}\PY{p}{)}

\PY{n}{xgb\PYZus{}day\PYZus{}price\PYZus{}ts}\PY{o}{.}\PY{n}{head}\PY{p}{(}\PY{p}{)}
\end{Verbatim}
\end{tcolorbox}

            \begin{tcolorbox}[breakable, size=fbox, boxrule=.5pt, pad at break*=1mm, opacityfill=0]
\prompt{Out}{outcolor}{19}{\boxspacing}
\begin{Verbatim}[commandchars=\\\{\}]
                     price  is\_weekend            datetime  hour  dayofweek  \textbackslash{}
start\_time
2022-01-01 00:00:00  50.05        True 2022-01-01 00:00:00     0          5
2022-01-01 01:00:00  41.33        True 2022-01-01 01:00:00     1          5
2022-01-01 02:00:00  43.22        True 2022-01-01 02:00:00     2          5
2022-01-01 03:00:00  45.46        True 2022-01-01 03:00:00     3          5
2022-01-01 04:00:00  37.67        True 2022-01-01 04:00:00     4          5

                     quarter  year  dayofyear  dayofmonth  month  weekofyear  \textbackslash{}
start\_time
2022-01-01 00:00:00        1  2022          1           1      1          52
2022-01-01 01:00:00        1  2022          1           1      1          52
2022-01-01 02:00:00        1  2022          1           1      1          52
2022-01-01 03:00:00        1  2022          1           1      1          52
2022-01-01 04:00:00        1  2022          1           1      1          52

                     is\_holiday
start\_time
2022-01-01 00:00:00       False
2022-01-01 01:00:00       False
2022-01-01 02:00:00       False
2022-01-01 03:00:00       False
2022-01-01 04:00:00       False
\end{Verbatim}
\end{tcolorbox}
        
    \begin{tcolorbox}[breakable, size=fbox, boxrule=1pt, pad at break*=1mm,colback=cellbackground, colframe=cellborder]
\prompt{In}{incolor}{20}{\boxspacing}
\begin{Verbatim}[commandchars=\\\{\}]
\PY{n}{sns}\PY{o}{.}\PY{n}{pairplot}\PY{p}{(}\PY{n}{xgb\PYZus{}day\PYZus{}price\PYZus{}ts}\PY{p}{,}
            \PY{n}{hue}\PY{o}{=}\PY{l+s+s1}{\PYZsq{}}\PY{l+s+s1}{hour}\PY{l+s+s1}{\PYZsq{}}\PY{p}{,}
            \PY{n}{x\PYZus{}vars}\PY{o}{=}\PY{p}{[}\PY{l+s+s1}{\PYZsq{}}\PY{l+s+s1}{dayofweek}\PY{l+s+s1}{\PYZsq{}}\PY{p}{,}\PY{l+s+s1}{\PYZsq{}}\PY{l+s+s1}{weekofyear}\PY{l+s+s1}{\PYZsq{}}\PY{p}{,} \PY{l+s+s1}{\PYZsq{}}\PY{l+s+s1}{is\PYZus{}weekend}\PY{l+s+s1}{\PYZsq{}}\PY{p}{,} \PY{l+s+s1}{\PYZsq{}}\PY{l+s+s1}{month}\PY{l+s+s1}{\PYZsq{}}\PY{p}{]}\PY{p}{,}
            \PY{n}{y\PYZus{}vars}\PY{o}{=}\PY{l+s+s1}{\PYZsq{}}\PY{l+s+s1}{price}\PY{l+s+s1}{\PYZsq{}}\PY{p}{,}
            \PY{n}{height}\PY{o}{=}\PY{l+m+mi}{5}\PY{p}{,}
            \PY{n}{plot\PYZus{}kws}\PY{o}{=}\PY{p}{\PYZob{}}\PY{l+s+s1}{\PYZsq{}}\PY{l+s+s1}{alpha}\PY{l+s+s1}{\PYZsq{}}\PY{p}{:}\PY{l+m+mf}{0.15}\PY{p}{,} \PY{l+s+s1}{\PYZsq{}}\PY{l+s+s1}{linewidth}\PY{l+s+s1}{\PYZsq{}}\PY{p}{:}\PY{l+m+mi}{0}\PY{p}{\PYZcb{}}
            \PY{p}{)}
\PY{n}{plt}\PY{o}{.}\PY{n}{suptitle}\PY{p}{(}\PY{l+s+s1}{\PYZsq{}}\PY{l+s+s1}{Day\PYZhy{}ahead Price [EUR/MWh]}\PY{l+s+s1}{\PYZsq{}}\PY{p}{,} \PY{n}{fontsize}\PY{o}{=}\PY{l+m+mi}{16}\PY{p}{)}
\PY{n}{plt}\PY{o}{.}\PY{n}{show}\PY{p}{(}\PY{p}{)}
\end{Verbatim}
\end{tcolorbox}

    \begin{center}
    \adjustimage{max size={0.9\linewidth}{0.9\paperheight}}{submission_files/submission_35_0.png}
    \end{center}
    { \hspace*{\fill} \\}
    
    \begin{tcolorbox}[breakable, size=fbox, boxrule=1pt, pad at break*=1mm,colback=cellbackground, colframe=cellborder]
\prompt{In}{incolor}{21}{\boxspacing}
\begin{Verbatim}[commandchars=\\\{\}]
\PY{n}{plotters}\PY{o}{.}\PY{n}{plot\PYZus{}violin\PYZus{}ts}\PY{p}{(}\PY{n}{xgb\PYZus{}day\PYZus{}price\PYZus{}ts}\PY{p}{,} \PY{l+s+s1}{\PYZsq{}}\PY{l+s+s1}{dayofweek}\PY{l+s+s1}{\PYZsq{}}\PY{p}{,} \PY{l+s+s1}{\PYZsq{}}\PY{l+s+s1}{price}\PY{l+s+s1}{\PYZsq{}}\PY{p}{,} \PY{l+s+s1}{\PYZsq{}}\PY{l+s+s1}{Day\PYZhy{}ahead Price [EUR/MWh] per day of week}\PY{l+s+s1}{\PYZsq{}}\PY{p}{)}
\end{Verbatim}
\end{tcolorbox}

    \begin{center}
    \adjustimage{max size={0.9\linewidth}{0.9\paperheight}}{submission_files/submission_36_0.png}
    \end{center}
    { \hspace*{\fill} \\}
    
    \begin{tcolorbox}[breakable, size=fbox, boxrule=1pt, pad at break*=1mm,colback=cellbackground, colframe=cellborder]
\prompt{In}{incolor}{22}{\boxspacing}
\begin{Verbatim}[commandchars=\\\{\}]
\PY{n}{plotters}\PY{o}{.}\PY{n}{plot\PYZus{}violin\PYZus{}ts}\PY{p}{(}\PY{n}{xgb\PYZus{}day\PYZus{}price\PYZus{}ts}\PY{p}{,} \PY{l+s+s1}{\PYZsq{}}\PY{l+s+s1}{month}\PY{l+s+s1}{\PYZsq{}}\PY{p}{,} \PY{l+s+s1}{\PYZsq{}}\PY{l+s+s1}{price}\PY{l+s+s1}{\PYZsq{}}\PY{p}{,} \PY{l+s+s1}{\PYZsq{}}\PY{l+s+s1}{Day\PYZhy{}ahead Price [EUR/MWh] per month of year}\PY{l+s+s1}{\PYZsq{}}\PY{p}{)}
\end{Verbatim}
\end{tcolorbox}

    \begin{center}
    \adjustimage{max size={0.9\linewidth}{0.9\paperheight}}{submission_files/submission_37_0.png}
    \end{center}
    { \hspace*{\fill} \\}
    
    \begin{tcolorbox}[breakable, size=fbox, boxrule=1pt, pad at break*=1mm,colback=cellbackground, colframe=cellborder]
\prompt{In}{incolor}{23}{\boxspacing}
\begin{Verbatim}[commandchars=\\\{\}]
\PY{n}{plotters}\PY{o}{.}\PY{n}{plot\PYZus{}violin\PYZus{}ts}\PY{p}{(}\PY{n}{xgb\PYZus{}day\PYZus{}price\PYZus{}ts}\PY{p}{,} \PY{l+s+s1}{\PYZsq{}}\PY{l+s+s1}{weekofyear}\PY{l+s+s1}{\PYZsq{}}\PY{p}{,} \PY{l+s+s1}{\PYZsq{}}\PY{l+s+s1}{price}\PY{l+s+s1}{\PYZsq{}}\PY{p}{,} \PY{l+s+s1}{\PYZsq{}}\PY{l+s+s1}{Day\PYZhy{}ahead Price [EUR/MWh] per month of year}\PY{l+s+s1}{\PYZsq{}}\PY{p}{,} \PY{k+kc}{False}\PY{p}{)}
\end{Verbatim}
\end{tcolorbox}

    \begin{center}
    \adjustimage{max size={0.9\linewidth}{0.9\paperheight}}{submission_files/submission_38_0.png}
    \end{center}
    { \hspace*{\fill} \\}
    
    \subsubsection{\texorpdfstring{{\textbf{Weather
Data}}}{Weather Data}}\label{weather-data}

Weather data is incredible useful to us because from the demand side.
For example, temperature and sunlight time might directly affect the
expected demand of the electricity, thus pushing up the DAA price. On
supply side, depending on how the electricity is priced, a predicted
high wind and sunny days might mean the production of electricity is
expected to be high, thus increasing the expected supply of electricity
decreasing the DAA price.

Of course, some of these weather data might actually be captured by the
time information such as month of the year already. However, given the
situation of increasingly unpredictable weather pattern due to climate
change, feature engineering on weather data is getting increasingly
important.

    \begin{tcolorbox}[breakable, size=fbox, boxrule=1pt, pad at break*=1mm,colback=cellbackground, colframe=cellborder]
\prompt{In}{incolor}{24}{\boxspacing}
\begin{Verbatim}[commandchars=\\\{\}]
\PY{n}{locations} \PY{o}{=} \PY{p}{\PYZob{}}
    \PY{l+s+s1}{\PYZsq{}}\PY{l+s+s1}{LU}\PY{l+s+s1}{\PYZsq{}}\PY{p}{:} \PY{p}{(}\PY{l+m+mf}{49.815273}\PY{p}{,} \PY{l+m+mf}{6.129583}\PY{p}{)}\PY{p}{,}  \PY{c+c1}{\PYZsh{} Luxembourg City coordinates}
    \PY{l+s+s1}{\PYZsq{}}\PY{l+s+s1}{DE}\PY{l+s+s1}{\PYZsq{}}\PY{p}{:} \PY{p}{(}\PY{l+m+mf}{52.520008}\PY{p}{,} \PY{l+m+mf}{13.404954}\PY{p}{)}  \PY{c+c1}{\PYZsh{} Berlin coordinates (as an example for Germany)}
\PY{p}{\PYZcb{}}

\PY{c+c1}{\PYZsh{} add mean weather data that is available from meteostat package}
\PY{n}{weather\PYZus{}data} \PY{o}{=} \PY{n}{du}\PY{o}{.}\PY{n}{add\PYZus{}locations\PYZus{}weather}\PY{p}{(}\PY{n}{xgb\PYZus{}day\PYZus{}price\PYZus{}ts}\PY{p}{,} \PY{n}{locations}\PY{p}{)} 

\PY{n}{xgb\PYZus{}merged\PYZus{}data} \PY{o}{=} \PY{n}{xgb\PYZus{}day\PYZus{}price\PYZus{}ts}\PY{o}{.}\PY{n}{merge}\PY{p}{(}
    \PY{n}{weather\PYZus{}data}\PY{p}{[}\PY{p}{[}\PY{l+s+s1}{\PYZsq{}}\PY{l+s+s1}{temp}\PY{l+s+s1}{\PYZsq{}}\PY{p}{,} \PY{l+s+s1}{\PYZsq{}}\PY{l+s+s1}{tsun}\PY{l+s+s1}{\PYZsq{}}\PY{p}{]}\PY{p}{]}\PY{p}{,} \PY{c+c1}{\PYZsh{} get only temperature and sunshine data}
    \PY{n}{left\PYZus{}index}\PY{o}{=}\PY{k+kc}{True}\PY{p}{,}
    \PY{n}{right\PYZus{}index}\PY{o}{=}\PY{k+kc}{True}\PY{p}{,}
    \PY{n}{how}\PY{o}{=}\PY{l+s+s1}{\PYZsq{}}\PY{l+s+s1}{left}\PY{l+s+s1}{\PYZsq{}}
    \PY{p}{)}

\PY{n}{xgb\PYZus{}merged\PYZus{}data}\PY{o}{.}\PY{n}{head}\PY{p}{(}\PY{p}{)}
\end{Verbatim}
\end{tcolorbox}

            \begin{tcolorbox}[breakable, size=fbox, boxrule=.5pt, pad at break*=1mm, opacityfill=0]
\prompt{Out}{outcolor}{24}{\boxspacing}
\begin{Verbatim}[commandchars=\\\{\}]
                     price  is\_weekend            datetime  hour  dayofweek  \textbackslash{}
start\_time
2022-01-01 00:00:00  50.05        True 2022-01-01 00:00:00     0          5
2022-01-01 01:00:00  41.33        True 2022-01-01 01:00:00     1          5
2022-01-01 02:00:00  43.22        True 2022-01-01 02:00:00     2          5
2022-01-01 03:00:00  45.46        True 2022-01-01 03:00:00     3          5
2022-01-01 04:00:00  37.67        True 2022-01-01 04:00:00     4          5

                     quarter  year  dayofyear  dayofmonth  month  weekofyear  \textbackslash{}
start\_time
2022-01-01 00:00:00        1  2022          1           1      1          52
2022-01-01 01:00:00        1  2022          1           1      1          52
2022-01-01 02:00:00        1  2022          1           1      1          52
2022-01-01 03:00:00        1  2022          1           1      1          52
2022-01-01 04:00:00        1  2022          1           1      1          52

                     is\_holiday  temp  tsun
start\_time
2022-01-01 00:00:00       False  11.9   0.0
2022-01-01 01:00:00       False  11.9   0.0
2022-01-01 02:00:00       False  11.9   0.0
2022-01-01 03:00:00       False  11.9   0.0
2022-01-01 04:00:00       False  11.8   0.0
\end{Verbatim}
\end{tcolorbox}
        
    \subsubsection{\texorpdfstring{{\textbf{How to capture economic
sentiment?}}}{How to capture economic sentiment?}}\label{how-to-capture-economic-sentiment}

As we can see in March 2022, there was a spike in DAA Price. At that
time, it was the invasion of Ukraine by Russia, a major exporter of
energy to Europe esp.~Germany. To capture this, we will use the
Volatility Index (VIX) as well as the german stock market index (DAX).

VIX measures the volatilty of options market. It's often referred to as
the ``fear index'' because it tends to spike during times of financial
stress or market uncertainty. While DAX is a stock market index that
represents 30 of the largest and most liquid publicly traded companies
in Germany, listed in the Frankfurt Stock Exchange.

Moreover, we would also include the currency exchange rate between EUR
and USD, the latter which most denominated currencies for trades.

Note here the data available to us is daily. Therefore, we will use the
closing price of previous day as part of our features.

    \begin{tcolorbox}[breakable, size=fbox, boxrule=1pt, pad at break*=1mm,colback=cellbackground, colframe=cellborder]
\prompt{In}{incolor}{25}{\boxspacing}
\begin{Verbatim}[commandchars=\\\{\}]
\PY{n}{indices} \PY{o}{=} \PY{p}{[}\PY{l+s+s1}{\PYZsq{}}\PY{l+s+s1}{\PYZca{}VIX}\PY{l+s+s1}{\PYZsq{}}\PY{p}{,} \PY{l+s+s1}{\PYZsq{}}\PY{l+s+s1}{\PYZca{}GDAXI}\PY{l+s+s1}{\PYZsq{}}\PY{p}{,} \PY{l+s+s1}{\PYZsq{}}\PY{l+s+s1}{EURUSD=X}\PY{l+s+s1}{\PYZsq{}}\PY{p}{]}

\PY{n}{vix\PYZus{}close} \PY{o}{=} \PY{n}{yf}\PY{o}{.}\PY{n}{download}\PY{p}{(}\PY{l+s+s1}{\PYZsq{}}\PY{l+s+s1}{\PYZca{}VIX}\PY{l+s+s1}{\PYZsq{}}\PY{p}{,}
                        \PY{n}{start}\PY{o}{=}\PY{n}{xgb\PYZus{}merged\PYZus{}data}\PY{o}{.}\PY{n}{index}\PY{o}{.}\PY{n}{min}\PY{p}{(}\PY{p}{)} \PY{o}{\PYZhy{}} \PY{n}{timedelta}\PY{p}{(}\PY{n}{days}\PY{o}{=}\PY{l+m+mi}{1}\PY{p}{)}\PY{p}{,}
                        \PY{n}{end}\PY{o}{=}\PY{n}{xgb\PYZus{}merged\PYZus{}data}\PY{o}{.}\PY{n}{index}\PY{o}{.}\PY{n}{max}\PY{p}{(}\PY{p}{)}
                        \PY{p}{)}\PY{p}{[}\PY{l+s+s1}{\PYZsq{}}\PY{l+s+s1}{Close}\PY{l+s+s1}{\PYZsq{}}\PY{p}{]}
\PY{n}{dax\PYZus{}close} \PY{o}{=} \PY{n}{yf}\PY{o}{.}\PY{n}{download}\PY{p}{(}\PY{l+s+s1}{\PYZsq{}}\PY{l+s+s1}{\PYZca{}GDAXI}\PY{l+s+s1}{\PYZsq{}}\PY{p}{,}
                        \PY{n}{start}\PY{o}{=}\PY{n}{xgb\PYZus{}merged\PYZus{}data}\PY{o}{.}\PY{n}{index}\PY{o}{.}\PY{n}{min}\PY{p}{(}\PY{p}{)} \PY{o}{\PYZhy{}} \PY{n}{timedelta}\PY{p}{(}\PY{n}{days}\PY{o}{=}\PY{l+m+mi}{1}\PY{p}{)}\PY{p}{,}
                        \PY{n}{end}\PY{o}{=}\PY{n}{xgb\PYZus{}merged\PYZus{}data}\PY{o}{.}\PY{n}{index}\PY{o}{.}\PY{n}{max}\PY{p}{(}\PY{p}{)}
                        \PY{p}{)}\PY{p}{[}\PY{l+s+s1}{\PYZsq{}}\PY{l+s+s1}{Close}\PY{l+s+s1}{\PYZsq{}}\PY{p}{]}
\PY{n}{eur\PYZus{}usd\PYZus{}close} \PY{o}{=} \PY{n}{yf}\PY{o}{.}\PY{n}{download}\PY{p}{(}\PY{l+s+s2}{\PYZdq{}}\PY{l+s+s2}{EURUSD=X}\PY{l+s+s2}{\PYZdq{}}\PY{p}{,}
                            \PY{n}{start}\PY{o}{=}\PY{n}{xgb\PYZus{}merged\PYZus{}data}\PY{o}{.}\PY{n}{index}\PY{o}{.}\PY{n}{min}\PY{p}{(}\PY{p}{)} \PY{o}{\PYZhy{}} \PY{n}{timedelta}\PY{p}{(}\PY{n}{days}\PY{o}{=}\PY{l+m+mi}{1}\PY{p}{)}\PY{p}{,}
                            \PY{n}{end}\PY{o}{=}\PY{n}{xgb\PYZus{}merged\PYZus{}data}\PY{o}{.}\PY{n}{index}\PY{o}{.}\PY{n}{max}\PY{p}{(}\PY{p}{)}
                            \PY{p}{)}\PY{p}{[}\PY{l+s+s1}{\PYZsq{}}\PY{l+s+s1}{Close}\PY{l+s+s1}{\PYZsq{}}\PY{p}{]}
\end{Verbatim}
\end{tcolorbox}

    \begin{Verbatim}[commandchars=\\\{\}]
YF.download() has changed argument auto\_adjust default to True
    \end{Verbatim}

    \begin{Verbatim}[commandchars=\\\{\}]
[*********************100\%***********************]  1 of 1 completed
[*********************100\%***********************]  1 of 1 completed
[*********************100\%***********************]  1 of 1 completed
    \end{Verbatim}

    \begin{tcolorbox}[breakable, size=fbox, boxrule=1pt, pad at break*=1mm,colback=cellbackground, colframe=cellborder]
\prompt{In}{incolor}{26}{\boxspacing}
\begin{Verbatim}[commandchars=\\\{\}]
\PY{c+c1}{\PYZsh{} Shift the indices by 1 day}
\PY{n}{vix\PYZus{}close}\PY{p}{[}\PY{l+s+s1}{\PYZsq{}}\PY{l+s+s1}{\PYZca{}VIX}\PY{l+s+s1}{\PYZsq{}}\PY{p}{]} \PY{o}{=} \PY{n}{vix\PYZus{}close}\PY{p}{[}\PY{l+s+s1}{\PYZsq{}}\PY{l+s+s1}{\PYZca{}VIX}\PY{l+s+s1}{\PYZsq{}}\PY{p}{]}\PY{o}{.}\PY{n}{shift}\PY{p}{(}\PY{l+m+mi}{1}\PY{p}{)}
\PY{n}{dax\PYZus{}close}\PY{p}{[}\PY{l+s+s1}{\PYZsq{}}\PY{l+s+s1}{\PYZca{}GDAXI}\PY{l+s+s1}{\PYZsq{}}\PY{p}{]} \PY{o}{=} \PY{n}{dax\PYZus{}close}\PY{p}{[}\PY{l+s+s1}{\PYZsq{}}\PY{l+s+s1}{\PYZca{}GDAXI}\PY{l+s+s1}{\PYZsq{}}\PY{p}{]}\PY{o}{.}\PY{n}{shift}\PY{p}{(}\PY{l+m+mi}{1}\PY{p}{)}
\PY{n}{eur\PYZus{}usd\PYZus{}close}\PY{p}{[}\PY{l+s+s1}{\PYZsq{}}\PY{l+s+s1}{EURUSD=X}\PY{l+s+s1}{\PYZsq{}}\PY{p}{]} \PY{o}{=} \PY{n}{eur\PYZus{}usd\PYZus{}close}\PY{p}{[}\PY{l+s+s1}{\PYZsq{}}\PY{l+s+s1}{EURUSD=X}\PY{l+s+s1}{\PYZsq{}}\PY{p}{]}\PY{o}{.}\PY{n}{shift}\PY{p}{(}\PY{l+m+mi}{1}\PY{p}{)}

\PY{c+c1}{\PYZsh{} drop the irrevelant date}
\PY{n}{vix\PYZus{}close} \PY{o}{=} \PY{n}{vix\PYZus{}close}\PY{o}{.}\PY{n}{loc}\PY{p}{[}\PY{n}{vix\PYZus{}close}\PY{o}{.}\PY{n}{index} \PY{o}{\PYZgt{}}\PY{o}{=} \PY{n}{datetime}\PY{p}{(}\PY{l+m+mi}{2022}\PY{p}{,} \PY{l+m+mi}{1}\PY{p}{,} \PY{l+m+mi}{1}\PY{p}{)}\PY{p}{]}
\PY{n}{dax\PYZus{}close} \PY{o}{=} \PY{n}{dax\PYZus{}close}\PY{o}{.}\PY{n}{loc}\PY{p}{[}\PY{n}{dax\PYZus{}close}\PY{o}{.}\PY{n}{index} \PY{o}{\PYZgt{}}\PY{o}{=} \PY{n}{datetime}\PY{p}{(}\PY{l+m+mi}{2022}\PY{p}{,} \PY{l+m+mi}{1}\PY{p}{,} \PY{l+m+mi}{1}\PY{p}{)}\PY{p}{]}
\PY{n}{eur\PYZus{}usd\PYZus{}close} \PY{o}{=} \PY{n}{eur\PYZus{}usd\PYZus{}close}\PY{o}{.}\PY{n}{loc}\PY{p}{[}\PY{n}{eur\PYZus{}usd\PYZus{}close}\PY{o}{.}\PY{n}{index} \PY{o}{\PYZgt{}}\PY{o}{=} \PY{n}{datetime}\PY{p}{(}\PY{l+m+mi}{2022}\PY{p}{,} \PY{l+m+mi}{1}\PY{p}{,} \PY{l+m+mi}{1}\PY{p}{)}\PY{p}{]}

\PY{n}{vix\PYZus{}close}\PY{o}{.}\PY{n}{sort\PYZus{}index}\PY{p}{(}\PY{n}{inplace}\PY{o}{=}\PY{k+kc}{True}\PY{p}{)}
\PY{n}{dax\PYZus{}close}\PY{o}{.}\PY{n}{sort\PYZus{}index}\PY{p}{(}\PY{n}{inplace}\PY{o}{=}\PY{k+kc}{True}\PY{p}{)}
\PY{n}{eur\PYZus{}usd\PYZus{}close}\PY{o}{.}\PY{n}{sort\PYZus{}index}\PY{p}{(}\PY{n}{inplace}\PY{o}{=}\PY{k+kc}{True}\PY{p}{)}
\end{Verbatim}
\end{tcolorbox}

    \begin{tcolorbox}[breakable, size=fbox, boxrule=1pt, pad at break*=1mm,colback=cellbackground, colframe=cellborder]
\prompt{In}{incolor}{27}{\boxspacing}
\begin{Verbatim}[commandchars=\\\{\}]
\PY{n}{xgb\PYZus{}merged\PYZus{}data}\PY{p}{[}\PY{l+s+s1}{\PYZsq{}}\PY{l+s+s1}{date}\PY{l+s+s1}{\PYZsq{}}\PY{p}{]} \PY{o}{=} \PY{n}{xgb\PYZus{}merged\PYZus{}data}\PY{p}{[}\PY{l+s+s1}{\PYZsq{}}\PY{l+s+s1}{datetime}\PY{l+s+s1}{\PYZsq{}}\PY{p}{]}\PY{o}{.}\PY{n}{dt}\PY{o}{.}\PY{n}{date}

\PY{n}{vix\PYZus{}close}\PY{p}{[}\PY{l+s+s1}{\PYZsq{}}\PY{l+s+s1}{date}\PY{l+s+s1}{\PYZsq{}}\PY{p}{]} \PY{o}{=} \PY{n}{vix\PYZus{}close}\PY{o}{.}\PY{n}{index}\PY{o}{.}\PY{n}{date}
\PY{n}{dax\PYZus{}close}\PY{p}{[}\PY{l+s+s1}{\PYZsq{}}\PY{l+s+s1}{date}\PY{l+s+s1}{\PYZsq{}}\PY{p}{]} \PY{o}{=} \PY{n}{dax\PYZus{}close}\PY{o}{.}\PY{n}{index}\PY{o}{.}\PY{n}{date}
\PY{n}{eur\PYZus{}usd\PYZus{}close}\PY{p}{[}\PY{l+s+s1}{\PYZsq{}}\PY{l+s+s1}{date}\PY{l+s+s1}{\PYZsq{}}\PY{p}{]} \PY{o}{=} \PY{n}{eur\PYZus{}usd\PYZus{}close}\PY{o}{.}\PY{n}{index}\PY{o}{.}\PY{n}{date}

\PY{n}{xgb\PYZus{}merged\PYZus{}data} \PY{o}{=} \PY{n}{xgb\PYZus{}merged\PYZus{}data}\PY{o}{.}\PY{n}{merge}\PY{p}{(}
                                \PY{n}{vix\PYZus{}close}\PY{p}{[}\PY{p}{[}\PY{l+s+s1}{\PYZsq{}}\PY{l+s+s1}{date}\PY{l+s+s1}{\PYZsq{}}\PY{p}{,} \PY{l+s+s1}{\PYZsq{}}\PY{l+s+s1}{\PYZca{}VIX}\PY{l+s+s1}{\PYZsq{}}\PY{p}{]}\PY{p}{]}\PY{p}{,} 
                                \PY{n}{on}\PY{o}{=}\PY{l+s+s1}{\PYZsq{}}\PY{l+s+s1}{date}\PY{l+s+s1}{\PYZsq{}}\PY{p}{,} 
                                \PY{n}{how}\PY{o}{=}\PY{l+s+s1}{\PYZsq{}}\PY{l+s+s1}{left}\PY{l+s+s1}{\PYZsq{}}
                                \PY{p}{)}

\PY{n}{xgb\PYZus{}merged\PYZus{}data} \PY{o}{=} \PY{n}{xgb\PYZus{}merged\PYZus{}data}\PY{o}{.}\PY{n}{merge}\PY{p}{(}
                                \PY{n}{dax\PYZus{}close}\PY{p}{[}\PY{p}{[}\PY{l+s+s1}{\PYZsq{}}\PY{l+s+s1}{date}\PY{l+s+s1}{\PYZsq{}}\PY{p}{,} \PY{l+s+s1}{\PYZsq{}}\PY{l+s+s1}{\PYZca{}GDAXI}\PY{l+s+s1}{\PYZsq{}}\PY{p}{]}\PY{p}{]}\PY{p}{,} 
                                \PY{n}{on}\PY{o}{=}\PY{l+s+s1}{\PYZsq{}}\PY{l+s+s1}{date}\PY{l+s+s1}{\PYZsq{}}\PY{p}{,} 
                                \PY{n}{how}\PY{o}{=}\PY{l+s+s1}{\PYZsq{}}\PY{l+s+s1}{left}\PY{l+s+s1}{\PYZsq{}}
                                \PY{p}{)}

\PY{n}{xgb\PYZus{}merged\PYZus{}data} \PY{o}{=} \PY{n}{xgb\PYZus{}merged\PYZus{}data}\PY{o}{.}\PY{n}{merge}\PY{p}{(}
                                \PY{n}{eur\PYZus{}usd\PYZus{}close}\PY{p}{[}\PY{p}{[}\PY{l+s+s1}{\PYZsq{}}\PY{l+s+s1}{date}\PY{l+s+s1}{\PYZsq{}}\PY{p}{,} \PY{l+s+s1}{\PYZsq{}}\PY{l+s+s1}{EURUSD=X}\PY{l+s+s1}{\PYZsq{}}\PY{p}{]}\PY{p}{]}\PY{p}{,} 
                                \PY{n}{on}\PY{o}{=}\PY{l+s+s1}{\PYZsq{}}\PY{l+s+s1}{date}\PY{l+s+s1}{\PYZsq{}}\PY{p}{,} 
                                \PY{n}{how}\PY{o}{=}\PY{l+s+s1}{\PYZsq{}}\PY{l+s+s1}{left}\PY{l+s+s1}{\PYZsq{}}
                                \PY{p}{)}

\PY{n}{xgb\PYZus{}merged\PYZus{}data}\PY{o}{.}\PY{n}{set\PYZus{}index}\PY{p}{(}\PY{n}{xgb\PYZus{}merged\PYZus{}data}\PY{p}{[}\PY{l+s+s1}{\PYZsq{}}\PY{l+s+s1}{date}\PY{l+s+s1}{\PYZsq{}}\PY{p}{]}\PY{p}{,} \PY{n}{inplace}\PY{o}{=}\PY{k+kc}{True}\PY{p}{)}

\PY{n}{xgb\PYZus{}merged\PYZus{}data} \PY{o}{=} \PY{n}{xgb\PYZus{}merged\PYZus{}data}\PY{o}{.}\PY{n}{rename}\PY{p}{(}
                                        \PY{n}{columns}\PY{o}{=}\PY{p}{\PYZob{}}\PY{l+s+s1}{\PYZsq{}}\PY{l+s+s1}{\PYZca{}VIX}\PY{l+s+s1}{\PYZsq{}}\PY{p}{:} \PY{l+s+s1}{\PYZsq{}}\PY{l+s+s1}{VIX}\PY{l+s+s1}{\PYZsq{}}\PY{p}{,}
                                        \PY{l+s+s1}{\PYZsq{}}\PY{l+s+s1}{\PYZca{}GDAXI}\PY{l+s+s1}{\PYZsq{}}\PY{p}{:} \PY{l+s+s1}{\PYZsq{}}\PY{l+s+s1}{DAX}\PY{l+s+s1}{\PYZsq{}}\PY{p}{,}
                                        \PY{l+s+s1}{\PYZsq{}}\PY{l+s+s1}{EURUSD=X}\PY{l+s+s1}{\PYZsq{}}\PY{p}{:} \PY{l+s+s1}{\PYZsq{}}\PY{l+s+s1}{EURUSD}\PY{l+s+s1}{\PYZsq{}}\PY{p}{\PYZcb{}}
                                        \PY{p}{)}


\PY{n}{xgb\PYZus{}merged\PYZus{}data}\PY{o}{.}\PY{n}{drop}\PY{p}{(}\PY{p}{[}\PY{l+s+s1}{\PYZsq{}}\PY{l+s+s1}{date}\PY{l+s+s1}{\PYZsq{}}\PY{p}{]}\PY{p}{,} \PY{n}{axis}\PY{o}{=}\PY{l+m+mi}{1}\PY{p}{,} \PY{n}{inplace}\PY{o}{=}\PY{k+kc}{True}\PY{p}{)}
\PY{n}{xgb\PYZus{}merged\PYZus{}data} \PY{o}{=} \PY{n}{xgb\PYZus{}merged\PYZus{}data}\PY{o}{.}\PY{n}{set\PYZus{}index}\PY{p}{(}\PY{n}{xgb\PYZus{}merged\PYZus{}data}\PY{p}{[}\PY{l+s+s1}{\PYZsq{}}\PY{l+s+s1}{datetime}\PY{l+s+s1}{\PYZsq{}}\PY{p}{]}\PY{p}{)}
\PY{n}{xgb\PYZus{}merged\PYZus{}data}
\end{Verbatim}
\end{tcolorbox}

            \begin{tcolorbox}[breakable, size=fbox, boxrule=.5pt, pad at break*=1mm, opacityfill=0]
\prompt{Out}{outcolor}{27}{\boxspacing}
\begin{Verbatim}[commandchars=\\\{\}]
                      price  is\_weekend            datetime  hour  dayofweek  \textbackslash{}
datetime
2022-01-01 00:00:00   50.05        True 2022-01-01 00:00:00     0          5
2022-01-01 01:00:00   41.33        True 2022-01-01 01:00:00     1          5
2022-01-01 02:00:00   43.22        True 2022-01-01 02:00:00     2          5
2022-01-01 03:00:00   45.46        True 2022-01-01 03:00:00     3          5
2022-01-01 04:00:00   37.67        True 2022-01-01 04:00:00     4          5
{\ldots}                     {\ldots}         {\ldots}                 {\ldots}   {\ldots}        {\ldots}
2022-06-30 19:00:00  479.00       False 2022-06-30 19:00:00    19          3
2022-06-30 20:00:00  450.00       False 2022-06-30 20:00:00    20          3
2022-06-30 21:00:00  394.21       False 2022-06-30 21:00:00    21          3
2022-06-30 22:00:00  355.17       False 2022-06-30 22:00:00    22          3
2022-06-30 23:00:00  258.08       False 2022-06-30 23:00:00    23          3

                     quarter  year  dayofyear  dayofmonth  month  weekofyear  \textbackslash{}
datetime
2022-01-01 00:00:00        1  2022          1           1      1          52
2022-01-01 01:00:00        1  2022          1           1      1          52
2022-01-01 02:00:00        1  2022          1           1      1          52
2022-01-01 03:00:00        1  2022          1           1      1          52
2022-01-01 04:00:00        1  2022          1           1      1          52
{\ldots}                      {\ldots}   {\ldots}        {\ldots}         {\ldots}    {\ldots}         {\ldots}
2022-06-30 19:00:00        2  2022        181          30      6          26
2022-06-30 20:00:00        2  2022        181          30      6          26
2022-06-30 21:00:00        2  2022        181          30      6          26
2022-06-30 22:00:00        2  2022        181          30      6          26
2022-06-30 23:00:00        2  2022        181          30      6          26

                     is\_holiday  temp  tsun    VIX           DAX    EURUSD
datetime
2022-01-01 00:00:00       False  11.9   0.0    NaN           NaN       NaN
2022-01-01 01:00:00       False  11.9   0.0    NaN           NaN       NaN
2022-01-01 02:00:00       False  11.9   0.0    NaN           NaN       NaN
2022-01-01 03:00:00       False  11.9   0.0    NaN           NaN       NaN
2022-01-01 04:00:00       False  11.8   0.0    NaN           NaN       NaN
{\ldots}                         {\ldots}   {\ldots}   {\ldots}    {\ldots}           {\ldots}       {\ldots}
2022-06-30 19:00:00       False  25.7  49.0  28.16  13003.349609  1.052355
2022-06-30 20:00:00       False  24.4   0.0  28.16  13003.349609  1.052355
2022-06-30 21:00:00       False  23.4   0.0  28.16  13003.349609  1.052355
2022-06-30 22:00:00       False  22.1   0.0  28.16  13003.349609  1.052355
2022-06-30 23:00:00       False  21.5   0.0  28.16  13003.349609  1.052355

[4344 rows x 17 columns]
\end{Verbatim}
\end{tcolorbox}
        
    \begin{tcolorbox}[breakable, size=fbox, boxrule=1pt, pad at break*=1mm,colback=cellbackground, colframe=cellborder]
\prompt{In}{incolor}{28}{\boxspacing}
\begin{Verbatim}[commandchars=\\\{\}]
\PY{n}{plotters}\PY{o}{.}\PY{n}{plot\PYZus{}compare\PYZus{}two\PYZus{}col}\PY{p}{(}\PY{n}{xgb\PYZus{}merged\PYZus{}data}\PY{p}{,} \PY{l+s+s1}{\PYZsq{}}\PY{l+s+s1}{price}\PY{l+s+s1}{\PYZsq{}}\PY{p}{,} \PY{l+s+s1}{\PYZsq{}}\PY{l+s+s1}{VIX}\PY{l+s+s1}{\PYZsq{}}\PY{p}{,} \PY{l+s+s1}{\PYZsq{}}\PY{l+s+s1}{Day\PYZhy{}ahead Price [EUR/MWh] vs VIX}\PY{l+s+s1}{\PYZsq{}}\PY{p}{)}
\end{Verbatim}
\end{tcolorbox}

    
    
    \begin{tcolorbox}[breakable, size=fbox, boxrule=1pt, pad at break*=1mm,colback=cellbackground, colframe=cellborder]
\prompt{In}{incolor}{29}{\boxspacing}
\begin{Verbatim}[commandchars=\\\{\}]
\PY{n}{plotters}\PY{o}{.}\PY{n}{plot\PYZus{}compare\PYZus{}two\PYZus{}col}\PY{p}{(}\PY{n}{xgb\PYZus{}merged\PYZus{}data}\PY{p}{,} \PY{l+s+s1}{\PYZsq{}}\PY{l+s+s1}{price}\PY{l+s+s1}{\PYZsq{}}\PY{p}{,} \PY{l+s+s1}{\PYZsq{}}\PY{l+s+s1}{DAX}\PY{l+s+s1}{\PYZsq{}}\PY{p}{,} \PY{l+s+s1}{\PYZsq{}}\PY{l+s+s1}{Day\PYZhy{}ahead Price [EUR/MWh] vs DAX}\PY{l+s+s1}{\PYZsq{}}\PY{p}{)}
\end{Verbatim}
\end{tcolorbox}

    
    
    \begin{tcolorbox}[breakable, size=fbox, boxrule=1pt, pad at break*=1mm,colback=cellbackground, colframe=cellborder]
\prompt{In}{incolor}{30}{\boxspacing}
\begin{Verbatim}[commandchars=\\\{\}]
\PY{n}{plotters}\PY{o}{.}\PY{n}{plot\PYZus{}compare\PYZus{}two\PYZus{}col}\PY{p}{(}\PY{n}{xgb\PYZus{}merged\PYZus{}data}\PY{p}{,} \PY{l+s+s1}{\PYZsq{}}\PY{l+s+s1}{price}\PY{l+s+s1}{\PYZsq{}}\PY{p}{,} \PY{l+s+s1}{\PYZsq{}}\PY{l+s+s1}{EURUSD}\PY{l+s+s1}{\PYZsq{}}\PY{p}{,} \PY{l+s+s1}{\PYZsq{}}\PY{l+s+s1}{Day\PYZhy{}ahead Price [EUR/MWh] vs VIX}\PY{l+s+s1}{\PYZsq{}}\PY{p}{)}
\end{Verbatim}
\end{tcolorbox}

    
    
    \subsubsection{\texorpdfstring{{\textbf{Energy
Market}}}{Energy Market}}\label{energy-market}

Another feature to consider adding is incorporating additional data from
the DE/LU market. Specifically, we can include variables such as grid
load, residual load, and day-ahead generation.

\begin{itemize}
\tightlist
\item
  Grid load: total electricity demand on the power grid at any given
  time, reflecting how much energy consumers are using.
\item
  Residual load: difference between the total demand and the electricity
  supplied by renewable sources like wind and solar; it indicates the
  amount of electricity that must be provided by conventional power
  plants or other sources to maintain grid stability.
\item
  Day-ahead generation: forecasted electricity production scheduled for
  the next day, based on market bids and planned generation. This
  variable is chosen instead of actual generation because we
  {\textbf{assume}} actual realized demand and supply is already priced
  in.
\item
  energy balance: measures the gap between supply and demand on the
  grid.
\end{itemize}

Note here that, like in financial indicies, we {\textbf{assume}} that
only the data of one hour prior given a time point is available to us,
i.e.~at 1pm start time, we only know the data up to 12pm-1pm. Therefore,
we need to take 1-lag of the data.

The data are obtained from SMARD.

    \begin{tcolorbox}[breakable, size=fbox, boxrule=1pt, pad at break*=1mm,colback=cellbackground, colframe=cellborder]
\prompt{In}{incolor}{31}{\boxspacing}
\begin{Verbatim}[commandchars=\\\{\}]
\PY{n}{df\PYZus{}load} \PY{o}{=} \PY{n}{pd}\PY{o}{.}\PY{n}{read\PYZus{}csv}\PY{p}{(}\PY{l+s+s1}{\PYZsq{}}\PY{l+s+s1}{./data/Actual\PYZus{}consumption\PYZus{}202112310000\PYZus{}202207010000\PYZus{}Hour.csv}\PY{l+s+s1}{\PYZsq{}}\PY{p}{,}
                        \PY{n}{delimiter}\PY{o}{=}\PY{l+s+s1}{\PYZsq{}}\PY{l+s+s1}{;}\PY{l+s+s1}{\PYZsq{}}\PY{p}{,} \PY{n}{parse\PYZus{}dates}\PY{o}{=}\PY{p}{[}\PY{l+s+s1}{\PYZsq{}}\PY{l+s+s1}{Start date}\PY{l+s+s1}{\PYZsq{}}\PY{p}{,} \PY{l+s+s1}{\PYZsq{}}\PY{l+s+s1}{End date}\PY{l+s+s1}{\PYZsq{}}\PY{p}{]}\PY{p}{)}

\PY{n}{df\PYZus{}load} \PY{o}{=} \PY{n}{df\PYZus{}load}\PY{o}{.}\PY{n}{drop}\PY{p}{(}\PY{n}{columns}\PY{o}{=}\PY{p}{[}\PY{l+s+s1}{\PYZsq{}}\PY{l+s+s1}{End date}\PY{l+s+s1}{\PYZsq{}}\PY{p}{]}\PY{p}{)}
\PY{n}{df\PYZus{}load} \PY{o}{=} \PY{n}{df\PYZus{}load}\PY{o}{.}\PY{n}{rename}\PY{p}{(}\PY{n}{columns}\PY{o}{=}\PY{p}{\PYZob{}}\PY{l+s+s1}{\PYZsq{}}\PY{l+s+s1}{grid load [MWh] Calculated resolutions}\PY{l+s+s1}{\PYZsq{}}\PY{p}{:} \PY{l+s+s1}{\PYZsq{}}\PY{l+s+s1}{grid\PYZus{}load}\PY{l+s+s1}{\PYZsq{}}\PY{p}{,}
                                    \PY{l+s+s1}{\PYZsq{}}\PY{l+s+s1}{Residual load [MWh] Calculated resolutions}\PY{l+s+s1}{\PYZsq{}}\PY{p}{:} \PY{l+s+s1}{\PYZsq{}}\PY{l+s+s1}{residual\PYZus{}load}\PY{l+s+s1}{\PYZsq{}}\PY{p}{,}\PY{p}{\PYZcb{}}\PY{p}{)}

\PY{c+c1}{\PYZsh{} obtain grid load}
\PY{n}{df\PYZus{}load}\PY{p}{[}\PY{l+s+s1}{\PYZsq{}}\PY{l+s+s1}{grid\PYZus{}load}\PY{l+s+s1}{\PYZsq{}}\PY{p}{]}\PY{o}{.}\PY{n}{loc}\PY{p}{[}\PY{n}{df\PYZus{}load}\PY{p}{[}\PY{l+s+s1}{\PYZsq{}}\PY{l+s+s1}{grid\PYZus{}load}\PY{l+s+s1}{\PYZsq{}}\PY{p}{]}\PY{o}{==}\PY{l+s+s1}{\PYZsq{}}\PY{l+s+s1}{\PYZhy{}}\PY{l+s+s1}{\PYZsq{}}\PY{p}{]} \PY{o}{=} \PY{n}{np}\PY{o}{.}\PY{n}{nan}
\PY{n}{df\PYZus{}load}\PY{p}{[}\PY{l+s+s1}{\PYZsq{}}\PY{l+s+s1}{grid\PYZus{}load}\PY{l+s+s1}{\PYZsq{}}\PY{p}{]} \PY{o}{=} \PY{n}{df\PYZus{}load}\PY{p}{[}\PY{l+s+s1}{\PYZsq{}}\PY{l+s+s1}{residual\PYZus{}load}\PY{l+s+s1}{\PYZsq{}}\PY{p}{]}\PY{o}{.}\PY{n}{shift}\PY{p}{(}\PY{l+m+mi}{1}\PY{p}{)}
\PY{n}{df\PYZus{}load}\PY{p}{[}\PY{l+s+s1}{\PYZsq{}}\PY{l+s+s1}{grid\PYZus{}load}\PY{l+s+s1}{\PYZsq{}}\PY{p}{]} \PY{o}{=} \PY{n}{df\PYZus{}load}\PY{p}{[}\PY{l+s+s1}{\PYZsq{}}\PY{l+s+s1}{grid\PYZus{}load}\PY{l+s+s1}{\PYZsq{}}\PY{p}{]}\PY{o}{.}\PY{n}{str}\PY{o}{.}\PY{n}{replace}\PY{p}{(}\PY{l+s+s1}{\PYZsq{}}\PY{l+s+s1}{,}\PY{l+s+s1}{\PYZsq{}}\PY{p}{,} \PY{l+s+s1}{\PYZsq{}}\PY{l+s+s1}{\PYZsq{}}\PY{p}{)}\PY{o}{.}\PY{n}{astype}\PY{p}{(}\PY{n+nb}{float}\PY{p}{)}


\PY{c+c1}{\PYZsh{} obtain residual load}
\PY{n}{df\PYZus{}load}\PY{p}{[}\PY{l+s+s1}{\PYZsq{}}\PY{l+s+s1}{residual\PYZus{}load}\PY{l+s+s1}{\PYZsq{}}\PY{p}{]}\PY{o}{.}\PY{n}{loc}\PY{p}{[}\PY{n}{df\PYZus{}load}\PY{p}{[}\PY{l+s+s1}{\PYZsq{}}\PY{l+s+s1}{residual\PYZus{}load}\PY{l+s+s1}{\PYZsq{}}\PY{p}{]}\PY{o}{==}\PY{l+s+s1}{\PYZsq{}}\PY{l+s+s1}{\PYZhy{}}\PY{l+s+s1}{\PYZsq{}}\PY{p}{]} \PY{o}{=} \PY{n}{np}\PY{o}{.}\PY{n}{nan} \PY{c+c1}{\PYZsh{} replace missing values properly}
\PY{n}{df\PYZus{}load}\PY{p}{[}\PY{l+s+s1}{\PYZsq{}}\PY{l+s+s1}{residual\PYZus{}load}\PY{l+s+s1}{\PYZsq{}}\PY{p}{]} \PY{o}{=} \PY{n}{df\PYZus{}load}\PY{p}{[}\PY{l+s+s1}{\PYZsq{}}\PY{l+s+s1}{residual\PYZus{}load}\PY{l+s+s1}{\PYZsq{}}\PY{p}{]}\PY{o}{.}\PY{n}{shift}\PY{p}{(}\PY{l+m+mi}{1}\PY{p}{)}
\PY{n}{df\PYZus{}load}\PY{p}{[}\PY{l+s+s1}{\PYZsq{}}\PY{l+s+s1}{residual\PYZus{}load}\PY{l+s+s1}{\PYZsq{}}\PY{p}{]} \PY{o}{=} \PY{n}{df\PYZus{}load}\PY{p}{[}\PY{l+s+s1}{\PYZsq{}}\PY{l+s+s1}{residual\PYZus{}load}\PY{l+s+s1}{\PYZsq{}}\PY{p}{]}\PY{o}{.}\PY{n}{str}\PY{o}{.}\PY{n}{replace}\PY{p}{(}\PY{l+s+s1}{\PYZsq{}}\PY{l+s+s1}{,}\PY{l+s+s1}{\PYZsq{}}\PY{p}{,} \PY{l+s+s1}{\PYZsq{}}\PY{l+s+s1}{\PYZsq{}}\PY{p}{)}\PY{o}{.}\PY{n}{astype}\PY{p}{(}\PY{n+nb}{float}\PY{p}{)}

\PY{n}{df\PYZus{}load} \PY{o}{=} \PY{n}{df\PYZus{}load}\PY{o}{.}\PY{n}{set\PYZus{}index}\PY{p}{(}\PY{l+s+s1}{\PYZsq{}}\PY{l+s+s1}{Start date}\PY{l+s+s1}{\PYZsq{}}\PY{p}{)}
\PY{n}{df\PYZus{}load}\PY{o}{.}\PY{n}{index}\PY{o}{.}\PY{n}{name} \PY{o}{=} \PY{l+s+s1}{\PYZsq{}}\PY{l+s+s1}{datetime}\PY{l+s+s1}{\PYZsq{}}
\PY{n}{df\PYZus{}load} \PY{o}{=} \PY{n}{df\PYZus{}load}\PY{o}{.}\PY{n}{loc}\PY{p}{[}\PY{n}{df\PYZus{}load}\PY{o}{.}\PY{n}{index} \PY{o}{\PYZgt{}}\PY{o}{=} \PY{n}{datetime}\PY{p}{(}\PY{l+m+mi}{2022}\PY{p}{,} \PY{l+m+mi}{1}\PY{p}{,} \PY{l+m+mi}{1}\PY{p}{)}\PY{p}{]}
\PY{n}{df\PYZus{}load}\PY{o}{.}\PY{n}{sort\PYZus{}index}\PY{p}{(}\PY{n}{inplace}\PY{o}{=}\PY{k+kc}{True}\PY{p}{)}
\end{Verbatim}
\end{tcolorbox}

    \begin{Verbatim}[commandchars=\\\{\}]
/var/folders/bj/vqft4btx31x4b208qdl7q1nm0000gn/T/ipykernel\_1173/2514074313.py:1:
UserWarning:

Could not infer format, so each element will be parsed individually, falling
back to `dateutil`. To ensure parsing is consistent and as-expected, please
specify a format.

/var/folders/bj/vqft4btx31x4b208qdl7q1nm0000gn/T/ipykernel\_1173/2514074313.py:9:
SettingWithCopyWarning:


A value is trying to be set on a copy of a slice from a DataFrame

See the caveats in the documentation: https://pandas.pydata.org/pandas-
docs/stable/user\_guide/indexing.html\#returning-a-view-versus-a-copy

/var/folders/bj/vqft4btx31x4b208qdl7q1nm0000gn/T/ipykernel\_1173/2514074313.py:15
: SettingWithCopyWarning:


A value is trying to be set on a copy of a slice from a DataFrame

See the caveats in the documentation: https://pandas.pydata.org/pandas-
docs/stable/user\_guide/indexing.html\#returning-a-view-versus-a-copy

    \end{Verbatim}

    \begin{tcolorbox}[breakable, size=fbox, boxrule=1pt, pad at break*=1mm,colback=cellbackground, colframe=cellborder]
\prompt{In}{incolor}{32}{\boxspacing}
\begin{Verbatim}[commandchars=\\\{\}]
\PY{n}{df\PYZus{}DH\PYZus{}generation} \PY{o}{=} \PY{n}{pd}\PY{o}{.}\PY{n}{read\PYZus{}csv}\PY{p}{(}\PY{l+s+s1}{\PYZsq{}}\PY{l+s+s1}{./data/Forecasted\PYZus{}generation\PYZus{}Day\PYZhy{}Ahead\PYZus{}202112310000\PYZus{}202207010000\PYZus{}Hour.csv}\PY{l+s+s1}{\PYZsq{}}\PY{p}{,}
                        \PY{n}{delimiter}\PY{o}{=}\PY{l+s+s1}{\PYZsq{}}\PY{l+s+s1}{;}\PY{l+s+s1}{\PYZsq{}}\PY{p}{,} \PY{n}{parse\PYZus{}dates}\PY{o}{=}\PY{p}{[}\PY{l+s+s1}{\PYZsq{}}\PY{l+s+s1}{Start date}\PY{l+s+s1}{\PYZsq{}}\PY{p}{,} \PY{l+s+s1}{\PYZsq{}}\PY{l+s+s1}{End date}\PY{l+s+s1}{\PYZsq{}}\PY{p}{]}\PY{p}{)}

\PY{n}{df\PYZus{}DH\PYZus{}generation} \PY{o}{=} \PY{n}{df\PYZus{}DH\PYZus{}generation}\PY{o}{.}\PY{n}{set\PYZus{}index}\PY{p}{(}\PY{l+s+s1}{\PYZsq{}}\PY{l+s+s1}{Start date}\PY{l+s+s1}{\PYZsq{}}\PY{p}{)}
\PY{n}{df\PYZus{}DH\PYZus{}generation}\PY{o}{.}\PY{n}{index}\PY{o}{.}\PY{n}{name} \PY{o}{=} \PY{l+s+s1}{\PYZsq{}}\PY{l+s+s1}{datetime}\PY{l+s+s1}{\PYZsq{}}

\PY{n}{df\PYZus{}DH\PYZus{}generation}\PY{o}{=} \PY{n}{df\PYZus{}DH\PYZus{}generation}\PY{p}{[}\PY{p}{[}\PY{l+s+s1}{\PYZsq{}}\PY{l+s+s1}{Total [MWh] Original resolutions}\PY{l+s+s1}{\PYZsq{}}\PY{p}{]}\PY{p}{]}
\PY{n}{df\PYZus{}DH\PYZus{}generation} \PY{o}{=} \PY{n}{df\PYZus{}DH\PYZus{}generation}\PY{o}{.}\PY{n}{rename}\PY{p}{(}\PY{n}{columns}\PY{o}{=}\PY{p}{\PYZob{}}\PY{l+s+s1}{\PYZsq{}}\PY{l+s+s1}{Total [MWh] Original resolutions}\PY{l+s+s1}{\PYZsq{}}\PY{p}{:} \PY{l+s+s1}{\PYZsq{}}\PY{l+s+s1}{total\PYZus{}DH\PYZus{}generation}\PY{l+s+s1}{\PYZsq{}}\PY{p}{\PYZcb{}}\PY{p}{)}
\PY{n}{df\PYZus{}DH\PYZus{}generation}\PY{p}{[}\PY{l+s+s1}{\PYZsq{}}\PY{l+s+s1}{total\PYZus{}DH\PYZus{}generation}\PY{l+s+s1}{\PYZsq{}}\PY{p}{]} \PY{o}{=} \PY{n}{df\PYZus{}DH\PYZus{}generation}\PY{p}{[}\PY{l+s+s1}{\PYZsq{}}\PY{l+s+s1}{total\PYZus{}DH\PYZus{}generation}\PY{l+s+s1}{\PYZsq{}}\PY{p}{]}\PY{o}{.}\PY{n}{shift}\PY{p}{(}\PY{l+m+mi}{1}\PY{p}{)}
\PY{n}{df\PYZus{}DH\PYZus{}generation}\PY{p}{[}\PY{l+s+s1}{\PYZsq{}}\PY{l+s+s1}{total\PYZus{}DH\PYZus{}generation}\PY{l+s+s1}{\PYZsq{}}\PY{p}{]} \PY{o}{=} \PY{n}{df\PYZus{}DH\PYZus{}generation}\PY{p}{[}\PY{l+s+s1}{\PYZsq{}}\PY{l+s+s1}{total\PYZus{}DH\PYZus{}generation}\PY{l+s+s1}{\PYZsq{}}\PY{p}{]}\PYZbs{}
                                            \PY{o}{.}\PY{n}{str}\PY{o}{.}\PY{n}{replace}\PY{p}{(}\PY{l+s+s1}{\PYZsq{}}\PY{l+s+s1}{,}\PY{l+s+s1}{\PYZsq{}}\PY{p}{,} \PY{l+s+s1}{\PYZsq{}}\PY{l+s+s1}{\PYZsq{}}\PY{p}{)}\PY{o}{.}\PY{n}{astype}\PY{p}{(}\PY{n+nb}{float}\PY{p}{)}

\PY{c+c1}{\PYZsh{} drops the previous day data}
\PY{n}{df\PYZus{}DH\PYZus{}generation} \PY{o}{=} \PY{n}{df\PYZus{}DH\PYZus{}generation}\PY{o}{.}\PY{n}{loc}\PY{p}{[}\PY{n}{df\PYZus{}DH\PYZus{}generation}\PY{o}{.}\PY{n}{index} \PY{o}{\PYZgt{}}\PY{o}{=} \PY{n}{datetime}\PY{p}{(}\PY{l+m+mi}{2022}\PY{p}{,} \PY{l+m+mi}{1}\PY{p}{,} \PY{l+m+mi}{1}\PY{p}{)}\PY{p}{]}
\PY{n}{df\PYZus{}DH\PYZus{}generation}\PY{o}{.}\PY{n}{sort\PYZus{}index}\PY{p}{(}\PY{n}{inplace}\PY{o}{=}\PY{k+kc}{True}\PY{p}{)}
\end{Verbatim}
\end{tcolorbox}

    \begin{Verbatim}[commandchars=\\\{\}]
/var/folders/bj/vqft4btx31x4b208qdl7q1nm0000gn/T/ipykernel\_1173/2546593148.py:1:
UserWarning:

Could not infer format, so each element will be parsed individually, falling
back to `dateutil`. To ensure parsing is consistent and as-expected, please
specify a format.

    \end{Verbatim}

    \begin{tcolorbox}[breakable, size=fbox, boxrule=1pt, pad at break*=1mm,colback=cellbackground, colframe=cellborder]
\prompt{In}{incolor}{33}{\boxspacing}
\begin{Verbatim}[commandchars=\\\{\}]
\PY{c+c1}{\PYZsh{} join all the energery market data}
\PY{n}{xgb\PYZus{}merged\PYZus{}data} \PY{o}{=} \PY{n}{xgb\PYZus{}merged\PYZus{}data}\PY{o}{.}\PY{n}{join}\PY{p}{(}\PY{n}{df\PYZus{}load}\PY{p}{[}\PY{p}{[}\PY{l+s+s1}{\PYZsq{}}\PY{l+s+s1}{grid\PYZus{}load}\PY{l+s+s1}{\PYZsq{}}\PY{p}{,} \PY{l+s+s1}{\PYZsq{}}\PY{l+s+s1}{residual\PYZus{}load}\PY{l+s+s1}{\PYZsq{}}\PY{p}{]}\PY{p}{]}\PY{p}{,} \PY{n}{how}\PY{o}{=}\PY{l+s+s1}{\PYZsq{}}\PY{l+s+s1}{left}\PY{l+s+s1}{\PYZsq{}}\PY{p}{)}\PYZbs{}
                                    \PY{o}{.}\PY{n}{join}\PY{p}{(}\PY{n}{df\PYZus{}DH\PYZus{}generation}\PY{p}{[}\PY{p}{[}\PY{l+s+s1}{\PYZsq{}}\PY{l+s+s1}{total\PYZus{}DH\PYZus{}generation}\PY{l+s+s1}{\PYZsq{}}\PY{p}{]}\PY{p}{]}\PY{p}{,} \PY{n}{how}\PY{o}{=}\PY{l+s+s1}{\PYZsq{}}\PY{l+s+s1}{left}\PY{l+s+s1}{\PYZsq{}}\PY{p}{)}
\end{Verbatim}
\end{tcolorbox}

    \begin{tcolorbox}[breakable, size=fbox, boxrule=1pt, pad at break*=1mm,colback=cellbackground, colframe=cellborder]
\prompt{In}{incolor}{34}{\boxspacing}
\begin{Verbatim}[commandchars=\\\{\}]
\PY{n}{xgb\PYZus{}merged\PYZus{}data}\PY{p}{[}\PY{l+s+s1}{\PYZsq{}}\PY{l+s+s1}{energy\PYZus{}balance}\PY{l+s+s1}{\PYZsq{}}\PY{p}{]} \PY{o}{=} \PY{n}{xgb\PYZus{}merged\PYZus{}data}\PY{p}{[}\PY{l+s+s1}{\PYZsq{}}\PY{l+s+s1}{total\PYZus{}DH\PYZus{}generation}\PY{l+s+s1}{\PYZsq{}}\PY{p}{]} \PY{o}{\PYZhy{}} \PY{n}{xgb\PYZus{}merged\PYZus{}data}\PY{p}{[}\PY{l+s+s1}{\PYZsq{}}\PY{l+s+s1}{grid\PYZus{}load}\PY{l+s+s1}{\PYZsq{}}\PY{p}{]}
\end{Verbatim}
\end{tcolorbox}

    \subsubsection{Covariance Matrices}\label{covariance-matrices}

In the following, we will generate a covariance matrix to examine the
relationships between features. This helps us identify whether certain
features move together (positive or negative correlation), which can
provide insight into multicollinearity or potential feature
redundancies.

    \begin{tcolorbox}[breakable, size=fbox, boxrule=1pt, pad at break*=1mm,colback=cellbackground, colframe=cellborder]
\prompt{In}{incolor}{35}{\boxspacing}
\begin{Verbatim}[commandchars=\\\{\}]
\PY{n}{plt}\PY{o}{.}\PY{n}{figure}\PY{p}{(}\PY{n}{figsize}\PY{o}{=}\PY{p}{(}\PY{l+m+mi}{12}\PY{p}{,} \PY{l+m+mi}{10}\PY{p}{)}\PY{p}{)}
\PY{n}{sns}\PY{o}{.}\PY{n}{heatmap}\PY{p}{(}
            \PY{n}{xgb\PYZus{}merged\PYZus{}data}\PY{o}{.}\PY{n}{reset\PYZus{}index}\PY{p}{(}\PY{n}{drop}\PY{o}{=}\PY{k+kc}{True}\PY{p}{)}\PY{o}{.}\PY{n}{drop}\PY{p}{(}
                \PY{p}{[}\PY{l+s+s1}{\PYZsq{}}\PY{l+s+s1}{price}\PY{l+s+s1}{\PYZsq{}}\PY{p}{,} \PY{l+s+s1}{\PYZsq{}}\PY{l+s+s1}{datetime}\PY{l+s+s1}{\PYZsq{}}\PY{p}{]}\PY{p}{,}
                \PY{n}{axis}\PY{o}{=}\PY{l+m+mi}{1}
                \PY{p}{)}\PY{o}{.}\PY{n}{cov}\PY{p}{(}\PY{p}{)}\PY{p}{,}
            \PY{n}{annot}\PY{o}{=}\PY{k+kc}{True}\PY{p}{,}
            \PY{n}{cmap}\PY{o}{=}\PY{l+s+s1}{\PYZsq{}}\PY{l+s+s1}{coolwarm}\PY{l+s+s1}{\PYZsq{}}\PY{p}{,}
            \PY{n}{linewidths}\PY{o}{=}\PY{l+m+mf}{0.5}
            \PY{p}{)}
\PY{n}{plt}\PY{o}{.}\PY{n}{title}\PY{p}{(}\PY{l+s+s1}{\PYZsq{}}\PY{l+s+s1}{Covariance Matrix Heatmap}\PY{l+s+s1}{\PYZsq{}}\PY{p}{)}
\PY{n}{plt}\PY{o}{.}\PY{n}{show}\PY{p}{(}\PY{p}{)}
\end{Verbatim}
\end{tcolorbox}

    \begin{center}
    \adjustimage{max size={0.9\linewidth}{0.9\paperheight}}{submission_files/submission_54_0.png}
    \end{center}
    { \hspace*{\fill} \\}
    
    From the table above, we can see that energy balance while having high
variance has also non-trivial covariance with features related to the
energy load. This was to be expected as it was merely the difference
between DA generation and grid load. Between grid load and residual
load, it also espected they have some correlation by definition.

We could use PCA analysis to run, but as the feature is not managable,
we will instead use XGBoost feature later to see which features are the
most important

    \begin{tcolorbox}[breakable, size=fbox, boxrule=1pt, pad at break*=1mm,colback=cellbackground, colframe=cellborder]
\prompt{In}{incolor}{36}{\boxspacing}
\begin{Verbatim}[commandchars=\\\{\}]
\PY{n}{drop\PYZus{}col} \PY{o}{=} \PY{p}{[}
            \PY{c+c1}{\PYZsh{} \PYZsq{}price\PYZus{}ma\PYZus{}3d\PYZsq{},}
            \PY{c+c1}{\PYZsh{} \PYZsq{}temp\PYZsq{},}
            \PY{c+c1}{\PYZsh{} \PYZsq{}EURUSD\PYZsq{},}
            \PY{l+s+s1}{\PYZsq{}}\PY{l+s+s1}{quarter}\PY{l+s+s1}{\PYZsq{}}\PY{p}{,}
            \PY{c+c1}{\PYZsh{} \PYZsq{}is\PYZus{}weekend\PYZsq{},}
            \PY{c+c1}{\PYZsh{} \PYZsq{}is\PYZus{}holiday\PYZsq{},}
            \PY{p}{]}

\PY{n}{cleaned\PYZus{}merged\PYZus{}data} \PY{o}{=} \PY{n}{xgb\PYZus{}merged\PYZus{}data}\PY{o}{.}\PY{n}{copy}\PY{p}{(}\PY{p}{)}\PY{o}{.}\PY{n}{drop}\PY{p}{(}\PY{n}{drop\PYZus{}col}\PY{p}{,} \PY{n}{axis}\PY{o}{=}\PY{l+m+mi}{1}\PY{p}{)}
\end{Verbatim}
\end{tcolorbox}

    \paragraph{\texorpdfstring{\textbf{Run
XGBoost}}{Run XGBoost}}\label{run-xgboost}

    \begin{tcolorbox}[breakable, size=fbox, boxrule=1pt, pad at break*=1mm,colback=cellbackground, colframe=cellborder]
\prompt{In}{incolor}{37}{\boxspacing}
\begin{Verbatim}[commandchars=\\\{\}]
\PY{n}{split\PYZus{}date} \PY{o}{=} \PY{n}{datetime}\PY{p}{(}\PY{l+m+mi}{2022}\PY{p}{,} \PY{l+m+mi}{5}\PY{p}{,} \PY{l+m+mi}{1}\PY{p}{)}
\PY{n}{xgb\PYZus{}obj} \PY{o}{=} \PY{n}{XGBWrapper}\PY{p}{(}\PY{n}{cleaned\PYZus{}merged\PYZus{}data}\PY{o}{.}\PY{n}{copy}\PY{p}{(}\PY{p}{)}\PY{p}{,} \PY{n}{split\PYZus{}point}\PY{o}{=}\PY{n}{split\PYZus{}date}\PY{p}{)}
\end{Verbatim}
\end{tcolorbox}

    \begin{Verbatim}[commandchars=\\\{\}]
Train-Test split at 2022-05-01 00:00:00
    \end{Verbatim}

    \subsubsection{\texorpdfstring{{\textbf{Lagged target variables as
features}}}{Lagged target variables as features}}\label{lagged-target-variables-as-features}

Time series data often has autocorrelation, meaning past values can
provide useful information about future values. So we want to also
include the previous lagged price data into our feature. However, to
prevent leakage, we will also add this column after train-text data has
been split.

Here we will use the following lagged variable and rolling mean:

    \begin{tcolorbox}[breakable, size=fbox, boxrule=1pt, pad at break*=1mm,colback=cellbackground, colframe=cellborder]
\prompt{In}{incolor}{38}{\boxspacing}
\begin{Verbatim}[commandchars=\\\{\}]
\PY{n}{xgb\PYZus{}obj}\PY{o}{.}\PY{n}{add\PYZus{}lagged\PYZus{}MA\PYZus{}price}\PY{p}{(}\PY{n}{hours}\PY{o}{=}\PY{l+m+mi}{4}\PY{p}{,} \PY{n}{days}\PY{o}{=}\PY{l+m+mi}{3}\PY{p}{)}
\end{Verbatim}
\end{tcolorbox}

    \begin{Verbatim}[commandchars=\\\{\}]
Adding n-lagged hour of price to X\_train and X\_test
Adding mean of previous 1 hours of price to X\_train and X\_test
Adding mean of previous 2 hours of price to X\_train and X\_test
Adding mean of previous 3 hours of price to X\_train and X\_test
Adding mean of previous 4 hours of price to X\_train and X\_test
Adding mean of previous 24 hours of price to X\_train and X\_test
Adding mean of previous 48 hours of price to X\_train and X\_test
Adding mean of previous 72 hours of price to X\_train and X\_test
    \end{Verbatim}

    \begin{tcolorbox}[breakable, size=fbox, boxrule=1pt, pad at break*=1mm,colback=cellbackground, colframe=cellborder]
\prompt{In}{incolor}{39}{\boxspacing}
\begin{Verbatim}[commandchars=\\\{\}]
\PY{n}{xgb\PYZus{}obj}\PY{o}{.}\PY{n}{run\PYZus{}xgb}\PY{p}{(}\PY{p}{)}
\PY{n}{xgb\PYZus{}obj}\PY{o}{.}\PY{n}{plot\PYZus{}importance}
\end{Verbatim}
\end{tcolorbox}

    \begin{Verbatim}[commandchars=\\\{\}]
XGBoost model initialized with seed 11 and None trees.
XGBoost model fitted.
    \end{Verbatim}

    \begin{center}
    \adjustimage{max size={0.9\linewidth}{0.9\paperheight}}{submission_files/submission_61_1.png}
    \end{center}
    { \hspace*{\fill} \\}
    
    \begin{tcolorbox}[breakable, size=fbox, boxrule=1pt, pad at break*=1mm,colback=cellbackground, colframe=cellborder]
\prompt{In}{incolor}{40}{\boxspacing}
\begin{Verbatim}[commandchars=\\\{\}]
\PY{n}{xgb\PYZus{}obj}\PY{o}{.}\PY{n}{get\PYZus{}forecast}\PY{p}{(}\PY{p}{)}
\PY{n}{xgb\PYZus{}obj}\PY{o}{.}\PY{n}{plot\PYZus{}forecast}\PY{p}{(}\PY{p}{)}
\PY{n}{forecast\PYZus{}metric\PYZus{}long}\PY{p}{[}\PY{l+s+s1}{\PYZsq{}}\PY{l+s+s1}{XGBoost}\PY{l+s+s1}{\PYZsq{}}\PY{p}{]} \PY{o}{=} \PY{n+nb}{list}\PY{p}{(}\PY{n}{xgb\PYZus{}obj}\PY{o}{.}\PY{n}{get\PYZus{}rmse\PYZus{}mae}\PY{p}{(}\PY{p}{)}\PY{p}{)}
\end{Verbatim}
\end{tcolorbox}

    
    
    \begin{Verbatim}[commandchars=\\\{\}]
RSME for XGBoost: 22.639
MAE for XGBoost: 15.408
    \end{Verbatim}

    \paragraph{\texorpdfstring{\textbf{Short forecast window:
24}}{Short forecast window: 24}}\label{short-forecast-window-24}

Training data = 60 days from the split datetime

    \begin{tcolorbox}[breakable, size=fbox, boxrule=1pt, pad at break*=1mm,colback=cellbackground, colframe=cellborder]
\prompt{In}{incolor}{41}{\boxspacing}
\begin{Verbatim}[commandchars=\\\{\}]
\PY{n}{start\PYZus{}slice} \PY{o}{=} \PY{n}{short\PYZus{}split\PYZus{}date} \PY{o}{\PYZhy{}} \PY{n}{timedelta}\PY{p}{(}\PY{n}{days}\PY{o}{=}\PY{l+m+mi}{60}\PY{p}{)} 
\PY{n}{short\PYZus{}xgb\PYZus{}obj} \PY{o}{=} \PY{n}{XGBWrapper}\PY{p}{(}
                \PY{n}{cleaned\PYZus{}merged\PYZus{}data}\PY{o}{.}\PY{n}{copy}\PY{p}{(}\PY{p}{)}\PY{p}{,}
                \PY{n}{short\PYZus{}split\PYZus{}date}\PY{p}{,}
                \PY{n}{start\PYZus{}date\PYZus{}slice}\PY{o}{=}\PY{n}{start\PYZus{}slice}
                \PY{p}{)}
\PY{n}{short\PYZus{}xgb\PYZus{}obj}\PY{o}{.}\PY{n}{add\PYZus{}lagged\PYZus{}MA\PYZus{}price}\PY{p}{(}\PY{n}{hours}\PY{o}{=}\PY{l+m+mi}{4}\PY{p}{,} \PY{n}{days}\PY{o}{=}\PY{l+m+mi}{3}\PY{p}{)}
\PY{n}{short\PYZus{}xgb\PYZus{}obj}\PY{o}{.}\PY{n}{run\PYZus{}xgb}\PY{p}{(}\PY{p}{)}
\PY{n}{short\PYZus{}xgb\PYZus{}obj}\PY{o}{.}\PY{n}{plot\PYZus{}importance}
\PY{n}{short\PYZus{}xgb\PYZus{}obj}\PY{o}{.}\PY{n}{get\PYZus{}forecast}\PY{p}{(}\PY{p}{)}
\PY{n}{short\PYZus{}xgb\PYZus{}obj}\PY{o}{.}\PY{n}{plot\PYZus{}forecast}\PY{p}{(}\PY{p}{)}
\PY{n}{forecast\PYZus{}metric\PYZus{}short}\PY{p}{[}\PY{l+s+s1}{\PYZsq{}}\PY{l+s+s1}{XGBoost}\PY{l+s+s1}{\PYZsq{}}\PY{p}{]} \PY{o}{=} \PY{n+nb}{list}\PY{p}{(}\PY{n}{short\PYZus{}xgb\PYZus{}obj}\PY{o}{.}\PY{n}{get\PYZus{}rmse\PYZus{}mae}\PY{p}{(}\PY{p}{)}\PY{p}{)}
\PY{n}{xgb\PYZus{}rsme\PYZus{}24H}\PY{p}{,} \PY{n}{xgb\PYZus{}mae\PYZus{}24H} \PY{o}{=} \PY{n}{du}\PY{o}{.}\PY{n}{get\PYZus{}rmse\PYZus{}mae}\PY{p}{(}
                                \PY{n}{short\PYZus{}xgb\PYZus{}obj}\PY{o}{.}\PY{n}{y\PYZus{}test}\PY{p}{[}\PY{o}{\PYZhy{}}\PY{l+m+mi}{24}\PY{p}{:}\PY{p}{]}\PY{p}{,}
                                \PY{n}{short\PYZus{}xgb\PYZus{}obj}\PY{o}{.}\PY{n}{get\PYZus{}forecast\PYZus{}df}\PY{p}{[}\PY{l+s+s1}{\PYZsq{}}\PY{l+s+s1}{price\PYZus{}prediction}\PY{l+s+s1}{\PYZsq{}}\PY{p}{]}\PY{p}{[}\PY{o}{\PYZhy{}}\PY{l+m+mi}{24}\PY{p}{:}\PY{p}{]}
                                \PY{p}{)}

\PY{n}{forecast\PYZus{}metric\PYZus{}short}\PY{p}{[}\PY{l+s+s1}{\PYZsq{}}\PY{l+s+s1}{XGBoost}\PY{l+s+s1}{\PYZsq{}}\PY{p}{]} \PY{o}{=} \PY{p}{[}\PY{n}{xgb\PYZus{}rsme\PYZus{}24H}\PY{p}{,} \PY{n}{xgb\PYZus{}mae\PYZus{}24H}\PY{p}{]}

\PY{n+nb}{print}\PY{p}{(}\PY{l+s+sa}{f}\PY{l+s+s2}{\PYZdq{}}\PY{l+s+s2}{RMSE for XGBoost last 24 H: }\PY{l+s+si}{\PYZob{}}\PY{n}{xgb\PYZus{}rsme\PYZus{}24H}\PY{l+s+si}{\PYZcb{}}\PY{l+s+s2}{\PYZdq{}}\PY{p}{)}
\PY{n+nb}{print}\PY{p}{(}\PY{l+s+sa}{f}\PY{l+s+s2}{\PYZdq{}}\PY{l+s+s2}{MAE for XGBoost last 24 H: }\PY{l+s+si}{\PYZob{}}\PY{n}{xgb\PYZus{}mae\PYZus{}24H}\PY{l+s+si}{\PYZcb{}}\PY{l+s+s2}{\PYZdq{}}\PY{p}{)}
\end{Verbatim}
\end{tcolorbox}

    \begin{Verbatim}[commandchars=\\\{\}]
Train-Test split at 2022-06-29 23:00:00
Adding n-lagged hour of price to X\_train and X\_test
Adding mean of previous 1 hours of price to X\_train and X\_test
Adding mean of previous 2 hours of price to X\_train and X\_test
Adding mean of previous 3 hours of price to X\_train and X\_test
Adding mean of previous 4 hours of price to X\_train and X\_test
Adding mean of previous 24 hours of price to X\_train and X\_test
Adding mean of previous 48 hours of price to X\_train and X\_test
Adding mean of previous 72 hours of price to X\_train and X\_test
XGBoost model initialized with seed 11 and None trees.
XGBoost model fitted.
    \end{Verbatim}

    \begin{center}
    \adjustimage{max size={0.9\linewidth}{0.9\paperheight}}{submission_files/submission_64_1.png}
    \end{center}
    { \hspace*{\fill} \\}
    
    
    
    \begin{Verbatim}[commandchars=\\\{\}]
RSME for XGBoost: 42.685
MAE for XGBoost: 34.993
RMSE for XGBoost last 24 H: 42.685
MAE for XGBoost last 24 H: 34.993
    \end{Verbatim}

    Backtesting: Forecast Models

    Backtesting for time series is the process of evaluating a forecasting
model by simulating how it would perform on historical data. This is
also known as \textbf{Walk Forward Validation}

The data is split into training and testing periods: the model is
trained on past observations (e.g., the previous 45 days) and then used
to predict future values (e.g., the next 24 hours), which are compared
to the actual outcomes.

This process is repeated across multiple time windows to assess the
model's accuracy and generalization over time. Backtesting helps
identify overfitting, validate model robustness, and select the best
forecasting approach for real-world deployment.

    \begin{tcolorbox}[breakable, size=fbox, boxrule=1pt, pad at break*=1mm,colback=cellbackground, colframe=cellborder]
\prompt{In}{incolor}{42}{\boxspacing}
\begin{Verbatim}[commandchars=\\\{\}]
\PY{c+c1}{\PYZsh{} Get a list of split dates for backtesting}
\PY{n}{split\PYZus{}dates} \PY{o}{=} \PY{p}{[}\PY{p}{]}
\PY{n}{forecast\PYZus{}window} \PY{o}{=} \PY{l+m+mi}{24}
\PY{n}{train\PYZus{}data\PYZus{}days} \PY{o}{=} \PY{l+m+mi}{45}
\PY{c+c1}{\PYZsh{} Example: if 2022\PYZhy{}04\PYZhy{}02 00:00:00 is chosen as split date,}
\PY{c+c1}{\PYZsh{} train data starts at 2022\PYZhy{}02\PYZhy{}16 00:00:00 and }
\PY{c+c1}{\PYZsh{} test data ends at 2022\PYZhy{}04\PYZhy{}03 00:00:00}

\PY{n}{back\PYZus{}test\PYZus{}df} \PY{o}{=} \PY{n}{day\PYZus{}price\PYZus{}df}\PY{o}{.}\PY{n}{copy}\PY{p}{(}\PY{p}{)}

\PY{k}{for} \PY{n}{n} \PY{o+ow}{in} \PY{n+nb}{range}\PY{p}{(}\PY{n}{train\PYZus{}data\PYZus{}days}\PY{o}{*}\PY{l+m+mi}{24} \PY{o}{+} \PY{n}{forecast\PYZus{}window}\PY{p}{,} \PY{n+nb}{len}\PY{p}{(}\PY{n}{back\PYZus{}test\PYZus{}df}\PY{p}{)}\PY{p}{,} \PY{n}{train\PYZus{}data\PYZus{}days}\PY{o}{*}\PY{l+m+mi}{24} \PY{o}{+} \PY{n}{forecast\PYZus{}window}\PY{p}{)}\PY{p}{:}
    \PY{n}{m} \PY{o}{=} \PY{n}{n} \PY{o}{\PYZhy{}} \PY{n}{forecast\PYZus{}window}
    \PY{n}{split\PYZus{}datetime} \PY{o}{=} \PY{n}{pd}\PY{o}{.}\PY{n}{to\PYZus{}datetime}\PY{p}{(}\PY{n}{back\PYZus{}test\PYZus{}df}\PY{o}{.}\PY{n}{iloc}\PY{p}{[}\PY{n}{m}\PY{p}{]}\PY{p}{[}\PY{l+s+s1}{\PYZsq{}}\PY{l+s+s1}{start\PYZus{}time}\PY{l+s+s1}{\PYZsq{}}\PY{p}{]}\PY{p}{)}
    \PY{n}{split\PYZus{}dates}\PY{o}{.}\PY{n}{append}\PY{p}{(}\PY{n}{split\PYZus{}datetime}\PY{p}{)}
\end{Verbatim}
\end{tcolorbox}

    \begin{tcolorbox}[breakable, size=fbox, boxrule=1pt, pad at break*=1mm,colback=cellbackground, colframe=cellborder]
\prompt{In}{incolor}{43}{\boxspacing}
\begin{Verbatim}[commandchars=\\\{\}]
\PY{n}{split\PYZus{}dates}
\end{Verbatim}
\end{tcolorbox}

            \begin{tcolorbox}[breakable, size=fbox, boxrule=.5pt, pad at break*=1mm, opacityfill=0]
\prompt{Out}{outcolor}{43}{\boxspacing}
\begin{Verbatim}[commandchars=\\\{\}]
[Timestamp('2022-02-15 00:00:00'),
 Timestamp('2022-04-02 00:00:00'),
 Timestamp('2022-05-18 00:00:00')]
\end{Verbatim}
\end{tcolorbox}
        
    \begin{tcolorbox}[breakable, size=fbox, boxrule=1pt, pad at break*=1mm,colback=cellbackground, colframe=cellborder]
\prompt{In}{incolor}{44}{\boxspacing}
\begin{Verbatim}[commandchars=\\\{\}]
\PY{o}{\PYZpc{}\PYZpc{}capture}
\PY{c+c1}{\PYZsh{} supresses ouput}
\PY{c+c1}{\PYZsh{} Backtest XGBoost}

\PY{n}{window\PYZus{}xgb\PYZus{}rsme} \PY{o}{=} \PY{p}{[}\PY{p}{]}
\PY{n}{window\PYZus{}xgb\PYZus{}mae} \PY{o}{=} \PY{p}{[}\PY{p}{]}
\PY{n}{window\PYZus{}xgb\PYZus{}forecast}\PY{p}{:} \PY{n+nb}{list}\PY{p}{[}\PY{n}{pd}\PY{o}{.}\PY{n}{DataFrame}\PY{p}{]} \PY{o}{=} \PY{p}{[}\PY{p}{]}

\PY{k}{for} \PY{n}{split} \PY{o+ow}{in} \PY{n}{split\PYZus{}dates}\PY{p}{:}
    \PY{n}{backtest\PYZus{}xgb} \PY{o}{=} \PY{n}{BacktestXGB}\PY{p}{(}\PY{n}{cleaned\PYZus{}merged\PYZus{}data}\PY{p}{,}
                \PY{n}{split}\PY{p}{,}
                \PY{n}{n\PYZus{}backward}\PY{o}{=}\PY{n}{train\PYZus{}data\PYZus{}days}\PY{p}{,}
                \PY{n}{forecast\PYZus{}window}\PY{o}{=}\PY{n}{forecast\PYZus{}window}\PY{p}{)}
    \PY{n}{rsme}\PY{p}{,} \PY{n}{mae} \PY{o}{=}\PY{n}{backtest\PYZus{}xgb}\PY{o}{.}\PY{n}{get\PYZus{}windowed\PYZus{}rsme\PYZus{}mae}\PY{p}{(}\PY{p}{)}
    \PY{n}{window\PYZus{}xgb\PYZus{}rsme}\PY{o}{.}\PY{n}{append}\PY{p}{(}\PY{n}{rsme}\PY{p}{)}
    \PY{n}{window\PYZus{}xgb\PYZus{}mae}\PY{o}{.}\PY{n}{append}\PY{p}{(}\PY{n}{mae}\PY{p}{)}

    \PY{n}{window\PYZus{}xgb\PYZus{}forecast}\PY{o}{.}\PY{n}{append}\PY{p}{(}\PY{n}{backtest\PYZus{}xgb}\PY{o}{.}\PY{n}{get\PYZus{}windowed\PYZus{}forecast}\PY{p}{(}\PY{p}{)}\PY{p}{)}
\end{Verbatim}
\end{tcolorbox}

    \begin{tcolorbox}[breakable, size=fbox, boxrule=1pt, pad at break*=1mm,colback=cellbackground, colframe=cellborder]
\prompt{In}{incolor}{45}{\boxspacing}
\begin{Verbatim}[commandchars=\\\{\}]
\PY{o}{\PYZpc{}\PYZpc{}capture}
\PY{c+c1}{\PYZsh{} supresses ouput}
\PY{c+c1}{\PYZsh{} Backtest XGBoost}

\PY{n}{window\PYZus{}prophet\PYZus{}rsme} \PY{o}{=} \PY{p}{[}\PY{p}{]}
\PY{n}{window\PYZus{}prophet\PYZus{}mae} \PY{o}{=} \PY{p}{[}\PY{p}{]}

\PY{k}{for} \PY{n}{split} \PY{o+ow}{in} \PY{n}{split\PYZus{}dates}\PY{p}{:}
    \PY{n}{backtest\PYZus{}prophet} \PY{o}{=} \PY{n}{BacktestProphet}\PY{p}{(}\PY{n}{cleaned\PYZus{}merged\PYZus{}data}\PY{p}{,}
                \PY{n}{split}\PY{p}{,}
                \PY{n}{n\PYZus{}backward}\PY{o}{=}\PY{n}{train\PYZus{}data\PYZus{}days}\PY{p}{,}
                \PY{n}{forecast\PYZus{}window}\PY{o}{=}\PY{n}{forecast\PYZus{}window}\PY{p}{)}
    \PY{n}{rsme}\PY{p}{,} \PY{n}{mae} \PY{o}{=}\PY{n}{backtest\PYZus{}prophet}\PY{o}{.}\PY{n}{get\PYZus{}windowed\PYZus{}rsme\PYZus{}mae}\PY{p}{(}\PY{p}{)}
    \PY{n}{window\PYZus{}prophet\PYZus{}rsme}\PY{o}{.}\PY{n}{append}\PY{p}{(}\PY{n}{rsme}\PY{p}{)}
    \PY{n}{window\PYZus{}prophet\PYZus{}mae}\PY{o}{.}\PY{n}{append}\PY{p}{(}\PY{n}{mae}\PY{p}{)}
\end{Verbatim}
\end{tcolorbox}

    \begin{Verbatim}[commandchars=\\\{\}]
00:01:50 - cmdstanpy - INFO - Chain [1] start processing
00:01:50 - cmdstanpy - INFO - Chain [1] done processing
00:01:50 - cmdstanpy - INFO - Chain [1] start processing
00:01:50 - cmdstanpy - INFO - Chain [1] done processing
00:01:50 - cmdstanpy - INFO - Chain [1] start processing
00:01:50 - cmdstanpy - INFO - Chain [1] done processing
    \end{Verbatim}

    \begin{tcolorbox}[breakable, size=fbox, boxrule=1pt, pad at break*=1mm,colback=cellbackground, colframe=cellborder]
\prompt{In}{incolor}{46}{\boxspacing}
\begin{Verbatim}[commandchars=\\\{\}]
\PY{o}{\PYZpc{}\PYZpc{}capture}
\PY{c+c1}{\PYZsh{} Backtest ARIMA}

\PY{n}{window\PYZus{}ARIMA\PYZus{}rsme} \PY{o}{=} \PY{p}{[}\PY{p}{]}
\PY{n}{window\PYZus{}ARIMA\PYZus{}mae} \PY{o}{=} \PY{p}{[}\PY{p}{]}

\PY{k}{for} \PY{n}{split} \PY{o+ow}{in} \PY{n}{split\PYZus{}dates}\PY{p}{:}
    \PY{n}{backtest\PYZus{}ARIMA} \PY{o}{=} \PY{n}{BacktestARIMA}\PY{p}{(}\PY{n}{day\PYZus{}price\PYZus{}ts}\PY{p}{,}
                \PY{n}{split}\PY{p}{,}
                \PY{n}{n\PYZus{}backward}\PY{o}{=}\PY{n}{train\PYZus{}data\PYZus{}days}\PY{p}{,}
                \PY{n}{forecast\PYZus{}window}\PY{o}{=}\PY{n}{forecast\PYZus{}window}\PY{p}{)}
    \PY{n}{rsme}\PY{p}{,} \PY{n}{mae} \PY{o}{=}\PY{n}{backtest\PYZus{}ARIMA}\PY{o}{.}\PY{n}{get\PYZus{}windowed\PYZus{}rsme\PYZus{}mae}\PY{p}{(}\PY{p}{)}
    \PY{n}{window\PYZus{}ARIMA\PYZus{}rsme}\PY{o}{.}\PY{n}{append}\PY{p}{(}\PY{n}{rsme}\PY{p}{)}
    \PY{n}{window\PYZus{}ARIMA\PYZus{}mae}\PY{o}{.}\PY{n}{append}\PY{p}{(}\PY{n}{mae}\PY{p}{)}
\end{Verbatim}
\end{tcolorbox}

    \begin{tcolorbox}[breakable, size=fbox, boxrule=1pt, pad at break*=1mm,colback=cellbackground, colframe=cellborder]
\prompt{In}{incolor}{47}{\boxspacing}
\begin{Verbatim}[commandchars=\\\{\}]
\PY{c+c1}{\PYZsh{} aggregate results and the mean for the RSME and MAE from the backtesting}

\PY{n}{forecast\PYZus{}metric\PYZus{}window} \PY{o}{=} \PY{p}{\PYZob{}}\PY{p}{\PYZcb{}}

\PY{n}{forecast\PYZus{}metric\PYZus{}window}\PY{p}{[}\PY{l+s+s1}{\PYZsq{}}\PY{l+s+s1}{ARIMA}\PY{l+s+s1}{\PYZsq{}}\PY{p}{]} \PY{o}{=} \PY{p}{[}\PY{n}{np}\PY{o}{.}\PY{n}{mean}\PY{p}{(}\PY{n}{window\PYZus{}ARIMA\PYZus{}rsme}\PY{p}{)}\PY{p}{,} \PY{n}{np}\PY{o}{.}\PY{n}{mean}\PY{p}{(}\PY{n}{window\PYZus{}ARIMA\PYZus{}mae}\PY{p}{)}\PY{p}{]}
\PY{n}{forecast\PYZus{}metric\PYZus{}window}\PY{p}{[}\PY{l+s+s1}{\PYZsq{}}\PY{l+s+s1}{Prophet}\PY{l+s+s1}{\PYZsq{}}\PY{p}{]} \PY{o}{=} \PY{p}{[}\PY{n}{np}\PY{o}{.}\PY{n}{mean}\PY{p}{(}\PY{n}{window\PYZus{}prophet\PYZus{}rsme}\PY{p}{)}\PY{p}{,} \PY{n}{np}\PY{o}{.}\PY{n}{mean}\PY{p}{(}\PY{n}{window\PYZus{}prophet\PYZus{}mae}\PY{p}{)}\PY{p}{]}
\PY{n}{forecast\PYZus{}metric\PYZus{}window}\PY{p}{[}\PY{l+s+s1}{\PYZsq{}}\PY{l+s+s1}{XGBoost}\PY{l+s+s1}{\PYZsq{}}\PY{p}{]} \PY{o}{=} \PY{p}{[}\PY{n}{np}\PY{o}{.}\PY{n}{mean}\PY{p}{(}\PY{n}{window\PYZus{}xgb\PYZus{}rsme}\PY{p}{)}\PY{p}{,} \PY{n}{np}\PY{o}{.}\PY{n}{mean}\PY{p}{(}\PY{n}{window\PYZus{}xgb\PYZus{}mae}\PY{p}{)}\PY{p}{]}
\end{Verbatim}
\end{tcolorbox}

    Q1 Analysis Conclusion

A common method to evaluate the performance of forecasting models is to
look at the error metrics, RMSE and MAE. Denote \(y_i\) and
\(\hat{y}_i\) as the target and predicated value, here it the DAA price
and predicted DAA price, where \(i \in \{1,\dots,N\}\). Then, RSME and
MAE are defined as:

\begin{itemize}
\tightlist
\item
  Root Mean Squared Error (RMSE) \[
  \text{RMSE} = \sqrt{ \frac{1}{N} \sum_{i=1}^{N} \left( y_i - \hat{y}_i \right)^2 }.
  \]
\end{itemize}

RMSE gives more weight to larger errors, so it's more sensitive to
outliers in the data.

\begin{itemize}
\tightlist
\item
  Mean Absolute Error (MAE) \[
  \text{MAE} = \frac{1}{N} \sum_{i=1}^{N} \left| y_i - \hat{y}_i \right|
  \] MAE is easy to interpret and treats all errors equally, regardless
  of their magnitude.
\end{itemize}

Rule of thumb is, we want the smaller values of RSME and/or MAE and
these values. As sanity check, these values should be lower than the
mean and std of the time series.

Now, let's look at the RSME and MAE of the models

    \begin{tcolorbox}[breakable, size=fbox, boxrule=1pt, pad at break*=1mm,colback=cellbackground, colframe=cellborder]
\prompt{In}{incolor}{48}{\boxspacing}
\begin{Verbatim}[commandchars=\\\{\}]
\PY{n}{du}\PY{o}{.}\PY{n}{print\PYZus{}rmse\PYZus{}mae\PYZus{}table}\PY{p}{(}\PY{n}{forecast\PYZus{}metric\PYZus{}long}\PY{p}{,} \PY{l+s+s2}{\PYZdq{}}\PY{l+s+s2}{RMSE and MAE table for Models \PYZhy{} Complete Evaluation, split at 2022\PYZhy{}05\PYZhy{}01}\PY{l+s+s2}{\PYZdq{}}\PY{p}{)}
\PY{n}{du}\PY{o}{.}\PY{n}{print\PYZus{}rmse\PYZus{}mae\PYZus{}table}\PY{p}{(}\PY{n}{forecast\PYZus{}metric\PYZus{}short}\PY{p}{,} \PY{l+s+s2}{\PYZdq{}}\PY{l+s+s2}{RMSE and MAE table for Models \PYZhy{} 24H Evaluation}\PY{l+s+s2}{\PYZdq{}}\PY{p}{)}
\end{Verbatim}
\end{tcolorbox}

    \begin{Verbatim}[commandchars=\\\{\}]
RMSE and MAE table for Models - Complete Evaluation, split at 2022-05-01
+---------+----------+----------+
| Model   |     RMSE |      MAE |
+=========+==========+==========+
| ARIMA   |  80.867  |  65.13   |
+---------+----------+----------+
| Prophet | 139.455  | 114.839  |
+---------+----------+----------+
| XGBoost |  22.6392 |  15.4079 |
+---------+----------+----------+
------------------------------------------------------------
RMSE and MAE table for Models - 24H Evaluation
+---------+--------+--------+
| Model   |   RMSE |    MAE |
+=========+========+========+
| ARIMA   | 68.445 | 57.871 |
+---------+--------+--------+
| Prophet | 39.688 | 30.529 |
+---------+--------+--------+
| XGBoost | 42.685 | 34.993 |
+---------+--------+--------+
------------------------------------------------------------
    \end{Verbatim}

    \begin{tcolorbox}[breakable, size=fbox, boxrule=1pt, pad at break*=1mm,colback=cellbackground, colframe=cellborder]
\prompt{In}{incolor}{49}{\boxspacing}
\begin{Verbatim}[commandchars=\\\{\}]
\PY{n}{du}\PY{o}{.}\PY{n}{print\PYZus{}rmse\PYZus{}mae\PYZus{}table}\PY{p}{(}\PY{n}{forecast\PYZus{}metric\PYZus{}window}\PY{p}{,} \PY{l+s+s2}{\PYZdq{}}\PY{l+s+s2}{RMSE and MAE table for Models \PYZhy{} Backtesting}\PY{l+s+s2}{\PYZdq{}}\PY{p}{)}
\end{Verbatim}
\end{tcolorbox}

    \begin{Verbatim}[commandchars=\\\{\}]
RMSE and MAE table for Models - Backtesting
+---------+---------+---------+
| Model   |    RMSE |     MAE |
+=========+=========+=========+
| ARIMA   | 47.075  | 38.739  |
+---------+---------+---------+
| Prophet | 34.7153 | 29.0017 |
+---------+---------+---------+
| XGBoost | 19.7091 | 15.0012 |
+---------+---------+---------+
------------------------------------------------------------
    \end{Verbatim}

    As shown in the table generated above, XGBoost demonstrates the lowest
overall RMSE and MAE, outperforming both ARIMA and Prophet, in longer
forecast window.

This is expected as XGBoost used more feature engineering and ARIMA \&
Prophet are univariate time series models. These traditional models
struggle with long-term forecasting, with Prophet performing worst
amongst the 3 models. However, when the training and testing window is
shortened, both ARIMA and Prophet show improved RMSE and MAE
values---though still not as low as those achieved by XGBoost.

In the short forecast window, however, we see Prophet shines in that it
was less RMSE and MAE than XGBoost and ARIMA.

As for forward rolling (backtesting), for a given window (61 days, 60
days train, 1 day forcast), we find that XGBoost performs better than
ARIMA and Prophet.

As for stability, XGBoost might be different from each run if seed is
not set. This changes is especially noticeable in the shorter forecast
window.

In terms of runtime, the difference seems to be insignificant given the
data we obtained. However, for our usecase, we did not need a large
n-estimator.However, a caveat for ARIMA is that the ACF plot does not
give a clear picture of what parameters to be picked in
ARIMA(\(p,d,q\)). Therefore, an optimization was carried out to find the
optimized \((p,d,q)\). This can be time consuming. For XGBoost, the
higher the n-estimators, the longer that it will take. For Prophet, its
runtime is negligible.

In consideration an weighting the performance of the models in terms of
backtesting metrics, we will {\textbf{choose XGBoost}} going forward to
Question 2 (Q2) as the the backtesting metrics suggests that it is more
resilient and robust.

    \subsubsection{\texorpdfstring{{\textbf{Limitations and
Improvements}}}{Limitations and Improvements}}\label{limitations-and-improvements}

One way to improve the analysis is by incorporating backtesting.
Backtesting simulates how a forecasting model would have performed in
the past by repeatedly training and testing it on different rolling
windows of historical data. This way we can get a more complete metrics
when evaluating the model.

This helps assess the model's stability and generalization across time.
Implementation typically includes splitting the dataset into multiple
train-test splits, sliding the window forward for time series, and keep
forecasting metrics at each step.

Another potential model that can be considered in the future is the use
of neural network such as Recurrent Neural Networks (RNNs) and Long
Short-Term Memory (LSTM) networks, which are designed to handle
sequential data by maintaining information over time steps. However,
they also have limitation in a way that the global minima might not be
reached or the runtime would take too long for larger data.

    \begin{tcolorbox}[breakable, size=fbox, boxrule=1pt, pad at break*=1mm,colback=cellbackground, colframe=cellborder]
\prompt{In}{incolor}{50}{\boxspacing}
\begin{Verbatim}[commandchars=\\\{\}]
\PY{c+c1}{\PYZsh{} \PYZsh{} pickle dump fitted XGBoost for forcasting short window (24H)}
\PY{c+c1}{\PYZsh{} with open(\PYZsq{}xgb\PYZus{}model\PYZus{}fitted.pkl\PYZsq{}, \PYZsq{}wb\PYZsq{}) as f:}
\PY{c+c1}{\PYZsh{}     pickle.dump(xgb\PYZus{}obj.model, f)}
\end{Verbatim}
\end{tcolorbox}

    \begin{tcolorbox}[breakable, size=fbox, boxrule=1pt, pad at break*=1mm,colback=cellbackground, colframe=cellborder]
\prompt{In}{incolor}{51}{\boxspacing}
\begin{Verbatim}[commandchars=\\\{\}]
\PY{c+c1}{\PYZsh{} xgb\PYZus{}obj.predicted\PYZus{}XGBoost.to\PYZus{}csv(\PYZsq{}./data/xgb\PYZus{}forecasted\PYZus{}price.csv\PYZsq{})}
\end{Verbatim}
\end{tcolorbox}

    Question 2

    Problem Formulation

So we have predicted the price in our previous question. Now, we want to
make use of the predictions in order to maximize profit by charging and
discharging our battery. We are given the following environment
variables:

\begin{itemize}
\tightlist
\item
  DAA Price (EUR/MWh) predicted by the XGBoost
\item
  A battery that can charge and discharge 1 MW per hour and has capacity
  of 1MWh
\item
  {\textbf{Aim}}: to make the strategy when to charge and discharge in
  time span of last 24H available in dataset (i.e.~30. June 2022)
\end{itemize}

{\textbf{Assumptions}}:

\begin{enumerate}
\def\labelenumi{\arabic{enumi}.}
\tightlist
\item
  battery starts at capacity of 0
\item
  battery charge and discharge is instantaneous and incurs no
  transaction cost
\item
  battery does not degrade overtime
\item
  Energy market is open and always have supply for the battery to charge
\item
  At the beginning of the day, the strategy of charge/discharge is
  decided and set for the day
\end{enumerate}

\textbf{Remark}: one can think about this as buying a stock in the stock
market where one person can buy or sell a homogenous stock given a
period of time. The act of charging, would be buying a stock from the
market and vice versa.

First, we define the denote the objective function by \[
    \sum_{t=0}^T p(t) [- a(t)],
    \] where \(p(t)\) is the price at time \(t\) and
\(a(t) \in [-1, 1, 0]\) is the action to charge/discharge at hourly time
\(t \in [0,\dots,T]\) (here \(T=23\)). In particular, the action space
is given by \[
a(t) =
\begin{cases}
1,\quad & \text{charge},\\
0, & \text{do nothing},\\
-1, & \text{discharge},
\end{cases}
\] for all \(t \in (1\dots,T)\)

\textbf{Note} that in the objective function, the negative sign on the
\(a(t)\) is because as one discharges, \(a(t) = -1\) and one earns
profit, therefore it has to be negated.

Recall our aim is \[
    \max_{\phi(a)} \sum_{t=0}^T p(t) [- a(t)],
    \] where \(\phi(a) = (a(t))_{t=0}^T\) is schedule of
charge/discharge.

Now we will formulate our constraints as follows:

\begin{enumerate}
\def\labelenumi{\arabic{enumi}.}
\item
  The battery cannot be charged or discharged (action) beyond its
  maximum rate, here 1 MHh:

  \[
  |a(t)| \leq 1.
  \]
\item
  Denote \(b(t)\) to be the charge level of the battery at time \(t\),
  its capacity can be represented by

  \[
  0 \leq b(t) \leq 1.
  \]
\item
  Recall we assumed we start with empty battery

  \[
   b(0) = 0.
   \]
\item
  To represent the physical fact that if we have an empty battery, we
  cannot discharge and that we must charged sometime earlier in order to
  discharge it later in time:

  \[
   \begin{aligned}
   b(t) & = b(t-1) + a(t-1)
   \\
   & = b(t-2) + a(t-2) + a(t-1)
   \\
   & \vdots
   \\
   & = b(0) + \sum_{i=0}^{t-1} a(i)
   \\
   & = \sum_{i=0}^{t-1} a(i).
   \end{aligned}
   \]
\end{enumerate}

\textbf{Well-posedness of the problem}: Finally, we show that solution
exists. Since our objective function is linear and, most importantly,
that all the variable parameters (convex cone) above are closed and
bounded (i.e.~compact). Then, by Weierstrass Theorem, there must exist a
unique maximum and minimum. Hence, the solution to the optimization
problem exists and unique, and therefore well-defined.

    Implementation: CVXPY

    \begin{tcolorbox}[breakable, size=fbox, boxrule=1pt, pad at break*=1mm,colback=cellbackground, colframe=cellborder]
\prompt{In}{incolor}{52}{\boxspacing}
\begin{Verbatim}[commandchars=\\\{\}]
\PY{c+c1}{\PYZsh{} load the forecasted price by XGBoost in Q1}
\PY{n}{forecasted\PYZus{}df} \PY{o}{=} \PY{n}{xgb\PYZus{}obj}\PY{o}{.}\PY{n}{predicted\PYZus{}XGBoost}\PY{o}{.}\PY{n}{copy}\PY{p}{(}\PY{p}{)}

\PY{n}{plotters}\PY{o}{.}\PY{n}{plotly\PYZus{}actual\PYZus{}predict}\PY{p}{(}
                        \PY{n}{forecasted\PYZus{}df}\PY{p}{,}
                        \PY{l+s+s1}{\PYZsq{}}\PY{l+s+s1}{price}\PY{l+s+s1}{\PYZsq{}}\PY{p}{,}
                        \PY{l+s+s1}{\PYZsq{}}\PY{l+s+s1}{price\PYZus{}prediction}\PY{l+s+s1}{\PYZsq{}}\PY{p}{,}
                        \PY{n}{title}\PY{o}{=}\PY{l+s+s1}{\PYZsq{}}\PY{l+s+s1}{Recall: Day\PYZhy{}ahead Price [EUR/MWh] Predicted with XGBoost}\PY{l+s+s1}{\PYZsq{}}
                        \PY{p}{)}
\end{Verbatim}
\end{tcolorbox}

    
    
    \begin{tcolorbox}[breakable, size=fbox, boxrule=1pt, pad at break*=1mm,colback=cellbackground, colframe=cellborder]
\prompt{In}{incolor}{53}{\boxspacing}
\begin{Verbatim}[commandchars=\\\{\}]
\PY{c+c1}{\PYZsh{} \PYZsh{} Uncomment if want to use Prophet}

\PY{c+c1}{\PYZsh{} prophet = short\PYZus{}prophet\PYZus{}obj.ts[[\PYZsq{}ds\PYZsq{}, \PYZsq{}y\PYZsq{}]].copy()}
\PY{c+c1}{\PYZsh{} prophet = prophet.merge(short\PYZus{}prophet\PYZus{}obj.forecasted\PYZus{}df, on=\PYZsq{}ds\PYZsq{}, how=\PYZsq{}left\PYZsq{})}
\PY{c+c1}{\PYZsh{} prophet.set\PYZus{}index(\PYZsq{}ds\PYZsq{}, inplace=True)}
\PY{c+c1}{\PYZsh{} forecasted\PYZus{}df = prophet.rename(columns=\PYZob{}\PYZsq{}y\PYZsq{}: \PYZsq{}price\PYZsq{}, \PYZsq{}yhat\PYZsq{}: \PYZsq{}price\PYZus{}prediction\PYZsq{}\PYZcb{})}
\PY{c+c1}{\PYZsh{} plotters.plotly\PYZus{}actual\PYZus{}predict(}
\PY{c+c1}{\PYZsh{}                         forecasted\PYZus{}df,}
\PY{c+c1}{\PYZsh{}                         \PYZsq{}price\PYZsq{},}
\PY{c+c1}{\PYZsh{}                         \PYZsq{}price\PYZus{}prediction\PYZsq{},}
\PY{c+c1}{\PYZsh{}                         title=\PYZsq{}Recall: Day\PYZhy{}ahead Price [EUR/MWh] Predicted with Prophet\PYZsq{}}
\PY{c+c1}{\PYZsh{}                         )}
\end{Verbatim}
\end{tcolorbox}

    \begin{tcolorbox}[breakable, size=fbox, boxrule=1pt, pad at break*=1mm,colback=cellbackground, colframe=cellborder]
\prompt{In}{incolor}{54}{\boxspacing}
\begin{Verbatim}[commandchars=\\\{\}]
\PY{c+c1}{\PYZsh{} limits the scope to last 24H}
\PY{n}{df\PYZus{}24H} \PY{o}{=} \PY{n}{forecasted\PYZus{}df}\PY{p}{[}\PY{l+s+s1}{\PYZsq{}}\PY{l+s+s1}{price\PYZus{}prediction}\PY{l+s+s1}{\PYZsq{}}\PY{p}{]}\PY{p}{[}\PY{o}{\PYZhy{}}\PY{l+m+mi}{24}\PY{p}{:}\PY{p}{]}
\PY{n}{predicted\PYZus{}array} \PY{o}{=} \PY{n}{df\PYZus{}24H}\PY{o}{.}\PY{n}{to\PYZus{}numpy}\PY{p}{(}\PY{p}{)} 
\end{Verbatim}
\end{tcolorbox}

    \begin{tcolorbox}[breakable, size=fbox, boxrule=1pt, pad at break*=1mm,colback=cellbackground, colframe=cellborder]
\prompt{In}{incolor}{55}{\boxspacing}
\begin{Verbatim}[commandchars=\\\{\}]
\PY{c+c1}{\PYZsh{} Env parameters}
\PY{k}{def}\PY{+w}{ }\PY{n+nf}{opt\PYZus{}battery\PYZus{}problem}\PY{p}{(}
        \PY{n}{predicted\PYZus{}array}\PY{p}{:} \PY{n}{np}\PY{o}{.}\PY{n}{ndarray}\PY{p}{,}
        \PY{n}{battery\PYZus{}capa}\PY{p}{:} \PY{n+nb}{int} \PY{o}{=} \PY{l+m+mi}{1}\PY{p}{,}
        \PY{n}{max\PYZus{}charge\PYZus{}discharge}\PY{p}{:} \PY{n+nb}{int} \PY{o}{=} \PY{l+m+mi}{1}
        \PY{p}{)} \PY{o}{\PYZhy{}}\PY{o}{\PYZgt{}} \PY{n+nb}{tuple}\PY{p}{[}\PY{n}{cp}\PY{o}{.}\PY{n}{Problem}\PY{p}{,} \PY{n}{np}\PY{o}{.}\PY{n}{ndarray}\PY{p}{,} \PY{n}{np}\PY{o}{.}\PY{n}{ndarray}\PY{p}{]}\PY{p}{:}
    
    \PY{n}{n} \PY{o}{=} \PY{n+nb}{len}\PY{p}{(}\PY{n}{predicted\PYZus{}array}\PY{p}{)}

    \PY{c+c1}{\PYZsh{} define the optimization variables}
    \PY{n}{charge\PYZus{}discharge} \PY{o}{=} \PY{n}{cp}\PY{o}{.}\PY{n}{Variable}\PY{p}{(}\PY{n}{n}\PY{p}{)}   \PY{c+c1}{\PYZsh{} a(t)}
    \PY{n}{state\PYZus{}of\PYZus{}charge} \PY{o}{=} \PY{n}{cp}\PY{o}{.}\PY{n}{Variable}\PY{p}{(}\PY{n}{n}\PY{p}{)}    \PY{c+c1}{\PYZsh{} s(t)}

    \PY{c+c1}{\PYZsh{} constraints}
    \PY{n}{constraints} \PY{o}{=} \PY{p}{[}
        \PY{n}{state\PYZus{}of\PYZus{}charge}\PY{p}{[}\PY{l+m+mi}{0}\PY{p}{]} \PY{o}{==} \PY{l+m+mi}{0}\PY{p}{,}                    \PY{c+c1}{\PYZsh{} Initial state of charge}
        \PY{n}{state\PYZus{}of\PYZus{}charge} \PY{o}{\PYZlt{}}\PY{o}{=} \PY{n}{battery\PYZus{}capa}\PY{p}{,}            \PY{c+c1}{\PYZsh{} Battery capacity limit}
        \PY{n}{charge\PYZus{}discharge} \PY{o}{\PYZgt{}}\PY{o}{=} \PY{o}{\PYZhy{}}\PY{n}{max\PYZus{}charge\PYZus{}discharge}\PY{p}{,}  \PY{c+c1}{\PYZsh{} Maximum discharge rate}
        \PY{n}{charge\PYZus{}discharge} \PY{o}{\PYZlt{}}\PY{o}{=} \PY{n}{max\PYZus{}charge\PYZus{}discharge}\PY{p}{,}   \PY{c+c1}{\PYZsh{} Maximum charge rate}
        \PY{n}{state\PYZus{}of\PYZus{}charge} \PY{o}{\PYZgt{}}\PY{o}{=} \PY{l+m+mi}{0}\PY{p}{,}                       \PY{c+c1}{\PYZsh{} Battery cannot be negative}
        \PY{n}{state\PYZus{}of\PYZus{}charge}\PY{p}{[}\PY{l+m+mi}{1}\PY{p}{:}\PY{p}{]} \PY{o}{==} \PY{n}{cp}\PY{o}{.}\PY{n}{cumsum}\PY{p}{(}\PY{n}{charge\PYZus{}discharge}\PY{p}{[}\PY{p}{:}\PY{o}{\PYZhy{}}\PY{l+m+mi}{1}\PY{p}{]}\PY{p}{)}\PY{p}{,}  \PY{c+c1}{\PYZsh{} State of charge evolution}
    \PY{p}{]}

    \PY{c+c1}{\PYZsh{} objective function}
    \PY{n}{objective} \PY{o}{=} \PY{n}{cp}\PY{o}{.}\PY{n}{Maximize}\PY{p}{(}\PY{n}{cp}\PY{o}{.}\PY{n}{sum}\PY{p}{(}\PY{n}{cp}\PY{o}{.}\PY{n}{multiply}\PY{p}{(}\PY{n}{predicted\PYZus{}array}\PY{p}{,} \PY{o}{\PYZhy{}}\PY{n}{charge\PYZus{}discharge}\PY{p}{)}\PY{p}{)}\PY{p}{)}
    \PY{k}{return} \PY{n}{cp}\PY{o}{.}\PY{n}{Problem}\PY{p}{(}\PY{n}{objective}\PY{p}{,} \PY{n}{constraints}\PY{p}{)}\PY{p}{,} \PY{n}{charge\PYZus{}discharge}\PY{p}{,} \PY{n}{state\PYZus{}of\PYZus{}charge}
\end{Verbatim}
\end{tcolorbox}

    \begin{tcolorbox}[breakable, size=fbox, boxrule=1pt, pad at break*=1mm,colback=cellbackground, colframe=cellborder]
\prompt{In}{incolor}{56}{\boxspacing}
\begin{Verbatim}[commandchars=\\\{\}]
\PY{c+c1}{\PYZsh{} solve the problem}
\PY{n}{problem\PYZus{}obj\PYZus{}24H}\PY{p}{,} \PY{n}{charge\PYZus{}discharge}\PY{p}{,} \PY{n}{state\PYZus{}of\PYZus{}charge} \PY{o}{=} \PY{n}{opt\PYZus{}battery\PYZus{}problem}\PY{p}{(}\PY{n}{predicted\PYZus{}array}\PY{p}{)}
\PY{n}{problem\PYZus{}obj\PYZus{}24H}\PY{o}{.}\PY{n}{solve}\PY{p}{(}\PY{p}{)}

\PY{n+nb}{print}\PY{p}{(}\PY{l+s+sa}{f}\PY{l+s+s2}{\PYZdq{}}\PY{l+s+s2}{Maximum predicted profit: }\PY{l+s+si}{\PYZob{}}\PY{n+nb}{round}\PY{p}{(}\PY{n}{problem\PYZus{}obj\PYZus{}24H}\PY{o}{.}\PY{n}{value}\PY{p}{,}\PY{l+m+mi}{2}\PY{p}{)}\PY{l+s+si}{\PYZcb{}}\PY{l+s+s2}{\PYZdq{}}\PY{p}{)}
\PY{n+nb}{print}\PY{p}{(}\PY{l+s+sa}{f}\PY{l+s+s2}{\PYZdq{}}\PY{l+s+s2}{Optimized charge/discharge schedule, phi(a): }\PY{l+s+si}{\PYZob{}}\PY{n}{np}\PY{o}{.}\PY{n}{round}\PY{p}{(}\PY{n}{charge\PYZus{}discharge}\PY{o}{.}\PY{n}{value}\PY{p}{)}\PY{l+s+si}{\PYZcb{}}\PY{l+s+s2}{\PYZdq{}}\PY{p}{)}
\PY{n+nb}{print}\PY{p}{(}\PY{l+s+sa}{f}\PY{l+s+s2}{\PYZdq{}}\PY{l+s+s2}{State of charge at each time step, s(t): }\PY{l+s+si}{\PYZob{}}\PY{n}{np}\PY{o}{.}\PY{n}{round}\PY{p}{(}\PY{n}{state\PYZus{}of\PYZus{}charge}\PY{o}{.}\PY{n}{value}\PY{p}{)}\PY{l+s+si}{\PYZcb{}}\PY{l+s+s2}{\PYZdq{}}\PY{p}{)}
\end{Verbatim}
\end{tcolorbox}

    \begin{Verbatim}[commandchars=\\\{\}]
Maximum predicted profit: 562.01
Optimized charge/discharge schedule, phi(a): [ 1.  0. -0.  0. -1.  1.  0. -1.
-0.  0. -0. -0.  1. -1.  1. -0.  0. -1.
 -0.  1. -0. -1. -0. -1.]
State of charge at each time step, s(t): [ 0.  1.  1.  1.  1.  0.  1.  1.  0.
0.  0.  0.  0.  1.  0.  1.  1.  1.
  0.  0.  1.  1.  0. -0.]
    \end{Verbatim}

    \begin{tcolorbox}[breakable, size=fbox, boxrule=1pt, pad at break*=1mm,colback=cellbackground, colframe=cellborder]
\prompt{In}{incolor}{57}{\boxspacing}
\begin{Verbatim}[commandchars=\\\{\}]
\PY{n}{px}\PY{o}{.}\PY{n}{line}\PY{p}{(}\PY{n}{x} \PY{o}{=} \PY{n}{df\PYZus{}24H}\PY{o}{.}\PY{n}{index}\PY{p}{,}
        \PY{n}{y}\PY{o}{=}\PY{n}{charge\PYZus{}discharge}\PY{o}{.}\PY{n}{value}\PY{p}{,}
        \PY{n}{title}\PY{o}{=}\PY{l+s+s2}{\PYZdq{}}\PY{l+s+s2}{Optimized schedule for charge/discharge, }\PY{l+s+se}{\PYZbs{}u03A6}\PY{l+s+s2}{(a(t))}\PY{l+s+s2}{\PYZdq{}}\PY{p}{)}
\end{Verbatim}
\end{tcolorbox}

    
    
    \begin{tcolorbox}[breakable, size=fbox, boxrule=1pt, pad at break*=1mm,colback=cellbackground, colframe=cellborder]
\prompt{In}{incolor}{58}{\boxspacing}
\begin{Verbatim}[commandchars=\\\{\}]
\PY{n}{px}\PY{o}{.}\PY{n}{line}\PY{p}{(}\PY{n}{x} \PY{o}{=} \PY{n}{df\PYZus{}24H}\PY{o}{.}\PY{n}{index}\PY{p}{,}
        \PY{n}{y}\PY{o}{=}\PY{n}{state\PYZus{}of\PYZus{}charge}\PY{o}{.}\PY{n}{value}\PY{p}{,}
        \PY{n}{title}\PY{o}{=}\PY{l+s+s2}{\PYZdq{}}\PY{l+s+s2}{Battery State, b(t)}\PY{l+s+s2}{\PYZdq{}}\PY{p}{)}
\end{Verbatim}
\end{tcolorbox}

    
    
    {\textbf{Note}}: Here we have obtained just the predicted profit
according to the predicted pirce XGBoost modeled. To find the actual
profit, we will need to compare with the real price with the strategy
\(\phi(a(t))\) we have optimized.

    \begin{tcolorbox}[breakable, size=fbox, boxrule=1pt, pad at break*=1mm,colback=cellbackground, colframe=cellborder]
\prompt{In}{incolor}{59}{\boxspacing}
\begin{Verbatim}[commandchars=\\\{\}]
\PY{c+c1}{\PYZsh{} load the real data}
\PY{n}{df\PYZus{}24H\PYZus{}actual} \PY{o}{=} \PY{n}{forecasted\PYZus{}df}\PY{p}{[}\PY{l+s+s1}{\PYZsq{}}\PY{l+s+s1}{price}\PY{l+s+s1}{\PYZsq{}}\PY{p}{]}\PY{p}{[}\PY{o}{\PYZhy{}}\PY{l+m+mi}{24}\PY{p}{:}\PY{p}{]}
\PY{n}{df\PYZus{}24H\PYZus{}actual} \PY{o}{=} \PY{n}{df\PYZus{}24H\PYZus{}actual}\PY{o}{.}\PY{n}{to\PYZus{}numpy}\PY{p}{(}\PY{p}{)}

\PY{n}{actual\PYZus{}profit} \PY{o}{=} \PY{n}{np}\PY{o}{.}\PY{n}{sum}\PY{p}{(}\PY{o}{\PYZhy{}} \PY{n}{df\PYZus{}24H\PYZus{}actual} \PY{o}{*} \PY{n}{charge\PYZus{}discharge}\PY{o}{.}\PY{n}{value}\PY{p}{)}
\PY{n+nb}{print}\PY{p}{(}\PY{l+s+sa}{f}\PY{l+s+s2}{\PYZdq{}}\PY{l+s+s2}{Actual profit: }\PY{l+s+si}{\PYZob{}}\PY{n+nb}{round}\PY{p}{(}\PY{n}{actual\PYZus{}profit}\PY{p}{,}\PY{l+m+mi}{2}\PY{p}{)}\PY{l+s+si}{\PYZcb{}}\PY{l+s+s2}{\PYZdq{}}\PY{p}{)}
\end{Verbatim}
\end{tcolorbox}

    \begin{Verbatim}[commandchars=\\\{\}]
Actual profit: 334.93
    \end{Verbatim}

    \paragraph{\texorpdfstring{{\textbf{Perfect
Foresight}}}{Perfect Foresight}}\label{perfect-foresight}

Let's first evaluate how much profit we would make if we knew the price
in perfect foresight.

    \begin{tcolorbox}[breakable, size=fbox, boxrule=1pt, pad at break*=1mm,colback=cellbackground, colframe=cellborder]
\prompt{In}{incolor}{60}{\boxspacing}
\begin{Verbatim}[commandchars=\\\{\}]
\PY{c+c1}{\PYZsh{} solve the problem}
\PY{n}{problem\PYZus{}obj\PYZus{}act}\PY{p}{,} \PY{n}{charge\PYZus{}discharge\PYZus{}act}\PY{p}{,} \PY{n}{state\PYZus{}of\PYZus{}charge\PYZus{}act} \PY{o}{=} \PY{n}{opt\PYZus{}battery\PYZus{}problem}\PY{p}{(}\PY{n}{df\PYZus{}24H\PYZus{}actual}\PY{p}{)}
\PY{n}{problem\PYZus{}obj\PYZus{}act}\PY{o}{.}\PY{n}{solve}\PY{p}{(}\PY{p}{)}

\PY{n+nb}{print}\PY{p}{(}\PY{l+s+sa}{f}\PY{l+s+s2}{\PYZdq{}}\PY{l+s+s2}{Perfect foresight: Maximum profit: }\PY{l+s+si}{\PYZob{}}\PY{n+nb}{round}\PY{p}{(}\PY{n}{problem\PYZus{}obj\PYZus{}act}\PY{o}{.}\PY{n}{value}\PY{p}{,}\PY{l+m+mi}{2}\PY{p}{)}\PY{l+s+si}{\PYZcb{}}\PY{l+s+s2}{\PYZdq{}}\PY{p}{)}
\PY{n+nb}{print}\PY{p}{(}\PY{l+s+sa}{f}\PY{l+s+s2}{\PYZdq{}}\PY{l+s+s2}{Perfect foresight: Optimized charge/discharge schedule, phi(a): }\PY{l+s+si}{\PYZob{}}\PY{n}{np}\PY{o}{.}\PY{n}{round}\PY{p}{(}\PY{n}{charge\PYZus{}discharge\PYZus{}act}\PY{o}{.}\PY{n}{value}\PY{p}{)}\PY{l+s+si}{\PYZcb{}}\PY{l+s+s2}{\PYZdq{}}\PY{p}{)}
\PY{n+nb}{print}\PY{p}{(}\PY{l+s+sa}{f}\PY{l+s+s2}{\PYZdq{}}\PY{l+s+s2}{Perfect foresight: State of charge at each time step, s(t): }\PY{l+s+si}{\PYZob{}}\PY{n}{np}\PY{o}{.}\PY{n}{round}\PY{p}{(}\PY{n}{state\PYZus{}of\PYZus{}charge\PYZus{}act}\PY{o}{.}\PY{n}{value}\PY{p}{)}\PY{l+s+si}{\PYZcb{}}\PY{l+s+s2}{\PYZdq{}}\PY{p}{)}
\end{Verbatim}
\end{tcolorbox}

    \begin{Verbatim}[commandchars=\\\{\}]
Perfect foresight: Maximum profit: 587.83
Perfect foresight: Optimized charge/discharge schedule, phi(a): [ 0.  1. -0. -1.
1.  0. -0. -0. -1. -0.  0.  1. -1.  1.  0.  0.  0. -0.
  0. -1. -0.  0. -0. -1.]
Perfect foresight: State of charge at each time step, s(t): [ 0.  0.  1.  1.  0.
1.  1.  1.  1.  0.  0.  0.  1.  0.  1.  1.  1.  1.
  1.  1.  0.  0.  0. -0.]
    \end{Verbatim}

    \paragraph{\texorpdfstring{{\textbf{How did XGBoost + CVXPY
do?}}}{How did XGBoost + CVXPY do?}}\label{how-did-xgboost-cvxpy-do}

    \begin{tcolorbox}[breakable, size=fbox, boxrule=1pt, pad at break*=1mm,colback=cellbackground, colframe=cellborder]
\prompt{In}{incolor}{61}{\boxspacing}
\begin{Verbatim}[commandchars=\\\{\}]
\PY{n}{act\PYZus{}profit} \PY{o}{=} \PY{n+nb}{round}\PY{p}{(}\PY{n}{actual\PYZus{}profit}\PY{p}{,}\PY{l+m+mi}{2}\PY{p}{)}
\PY{n}{perc\PYZus{}gain} \PY{o}{=} \PY{n+nb}{round}\PY{p}{(}\PY{p}{(}\PY{p}{(}\PY{n}{problem\PYZus{}obj\PYZus{}act}\PY{o}{.}\PY{n}{value} \PY{o}{\PYZhy{}} \PY{n}{actual\PYZus{}profit}\PY{p}{)} \PY{o}{/} \PY{n}{problem\PYZus{}obj\PYZus{}act}\PY{o}{.}\PY{n}{value}\PY{p}{)} \PY{o}{*} \PY{l+m+mi}{100}\PY{p}{,}\PY{l+m+mi}{2}\PY{p}{)}

\PY{n}{msg} \PY{o}{=} \PY{l+s+sa}{f}\PY{l+s+s2}{\PYZdq{}}\PY{l+s+s2}{As shown above, we were able to get (average) profit of }\PY{l+s+si}{\PYZob{}}\PY{n+nb}{round}\PY{p}{(}\PY{n}{actual\PYZus{}profit}\PY{p}{,}\PY{l+m+mi}{2}\PY{p}{)}\PY{l+s+si}{\PYZcb{}}\PY{l+s+s2}{ Euro with predicted price from XGBoost and optimization from CVXPY.}\PY{l+s+s2}{\PYZdq{}}
\PY{n}{msg} \PY{o}{+}\PY{o}{=} \PY{l+s+sa}{f}\PY{l+s+s2}{\PYZdq{}}\PY{l+s+se}{\PYZbs{}n}\PY{l+s+s2}{If we have perfect foresight, we would have earn about }\PY{l+s+si}{\PYZob{}}\PY{n}{perc\PYZus{}gain}\PY{l+s+si}{\PYZcb{}}\PY{l+s+s2}{\PYZpc{} more.}\PY{l+s+s2}{\PYZdq{}}

\PY{n+nb}{print}\PY{p}{(}\PY{n}{msg}\PY{p}{)}
\end{Verbatim}
\end{tcolorbox}

    \begin{Verbatim}[commandchars=\\\{\}]
As shown above, we were able to get (average) profit of 334.93 Euro with
predicted price from XGBoost and optimization from CVXPY.
If we have perfect foresight, we would have earn about 43.02\% more.
    \end{Verbatim}

    Backtesting: Optimization

    \begin{tcolorbox}[breakable, size=fbox, boxrule=1pt, pad at break*=1mm,colback=cellbackground, colframe=cellborder]
\prompt{In}{incolor}{62}{\boxspacing}
\begin{Verbatim}[commandchars=\\\{\}]
\PY{k+kn}{from}\PY{+w}{ }\PY{n+nn}{scipy}\PY{n+nn}{.}\PY{n+nn}{stats}\PY{+w}{ }\PY{k+kn}{import} \PY{n}{hmean}
\PY{n+nb}{print}\PY{p}{(}\PY{l+s+sa}{f}\PY{l+s+s2}{\PYZdq{}}\PY{l+s+s2}{we will use the following }\PY{l+s+si}{\PYZob{}}\PY{n+nb}{str}\PY{p}{(}\PY{n}{split\PYZus{}dates}\PY{p}{)}\PY{l+s+si}{\PYZcb{}}\PY{l+s+s2}{\PYZdq{}}\PY{p}{)}

\PY{n}{backtest\PYZus{}proft\PYZus{}act} \PY{o}{=} \PY{p}{[}\PY{p}{]}
\PY{n}{backtest\PYZus{}profit\PYZus{}foresight} \PY{o}{=} \PY{p}{[}\PY{p}{]}

\PY{k}{for} \PY{n}{df} \PY{o+ow}{in} \PY{n}{window\PYZus{}xgb\PYZus{}forecast}\PY{p}{:}

    \PY{n+nb}{print}\PY{p}{(}\PY{n}{start\PYZus{}date}\PY{p}{,} \PY{n}{end\PYZus{}date}\PY{p}{)}
    \PY{c+c1}{\PYZsh{} get charge discharge schedule from forecasted price}
    \PY{n}{predicted\PYZus{}array} \PY{o}{=} \PY{n}{df}\PY{p}{[}\PY{l+s+s1}{\PYZsq{}}\PY{l+s+s1}{price\PYZus{}prediction}\PY{l+s+s1}{\PYZsq{}}\PY{p}{]}\PY{o}{.}\PY{n}{to\PYZus{}numpy}\PY{p}{(}\PY{p}{)}
    \PY{n}{problem\PYZus{}obj\PYZus{}window}\PY{p}{,} \PY{n}{charge\PYZus{}discharge}\PY{p}{,} \PY{n}{state\PYZus{}of\PYZus{}charge} \PY{o}{=} \PY{n}{opt\PYZus{}battery\PYZus{}problem}\PY{p}{(}\PY{n}{predicted\PYZus{}array}\PY{p}{)}
    \PY{n}{problem\PYZus{}obj\PYZus{}window}\PY{o}{.}\PY{n}{solve}\PY{p}{(}\PY{p}{)}
    \PY{n}{backtest\PYZus{}proft\PYZus{}act}\PY{o}{.}\PY{n}{append}\PY{p}{(}\PY{n}{problem\PYZus{}obj\PYZus{}24H}\PY{o}{.}\PY{n}{value}\PY{p}{)}

    \PY{n}{actual\PYZus{}array} \PY{o}{=} \PY{n}{df}\PY{p}{[}\PY{l+s+s1}{\PYZsq{}}\PY{l+s+s1}{price}\PY{l+s+s1}{\PYZsq{}}\PY{p}{]}\PY{o}{.}\PY{n}{to\PYZus{}numpy}\PY{p}{(}\PY{p}{)}

    \PY{c+c1}{\PYZsh{} calculate profit from the schedule from forecasted price}
    \PY{n}{actual\PYZus{}profit\PYZus{}pred} \PY{o}{=} \PY{n}{np}\PY{o}{.}\PY{n}{sum}\PY{p}{(}\PY{o}{\PYZhy{}} \PY{n}{actual\PYZus{}array} \PY{o}{*} \PY{n}{charge\PYZus{}discharge}\PY{o}{.}\PY{n}{value}\PY{p}{)}

    \PY{c+c1}{\PYZsh{} get max profit from perfect foresight}
    \PY{n}{problem\PYZus{}obj\PYZus{}foresight}\PY{p}{,} \PY{n}{charge\PYZus{}discharge\PYZus{}foresight}\PY{p}{,} \PY{n}{state\PYZus{}of\PYZus{}charge\PYZus{}foresight} \PY{o}{=} \PY{n}{opt\PYZus{}battery\PYZus{}problem}\PY{p}{(}\PY{n}{actual\PYZus{}array}\PY{p}{)}
    \PY{n}{problem\PYZus{}obj\PYZus{}foresight}\PY{o}{.}\PY{n}{solve}\PY{p}{(}\PY{p}{)}
    \PY{n}{backtest\PYZus{}profit\PYZus{}foresight}\PY{o}{.}\PY{n}{append}\PY{p}{(}\PY{n}{problem\PYZus{}obj\PYZus{}foresight}\PY{o}{.}\PY{n}{value}\PY{p}{)}

\PY{c+c1}{\PYZsh{} calculate the average profit from the model and foresight}
\PY{n}{avg\PYZus{}profit\PYZus{}act} \PY{o}{=} \PY{n+nb}{round}\PY{p}{(}\PY{n}{np}\PY{o}{.}\PY{n}{mean}\PY{p}{(}\PY{n}{actual\PYZus{}profit\PYZus{}pred}\PY{p}{)}\PY{p}{,}\PY{l+m+mi}{3}\PY{p}{)}
\PY{n}{avg\PYZus{}profit\PYZus{}foresight} \PY{o}{=} \PY{n+nb}{round}\PY{p}{(}\PY{n}{np}\PY{o}{.}\PY{n}{mean}\PY{p}{(}\PY{n}{backtest\PYZus{}profit\PYZus{}foresight}\PY{p}{)}\PY{p}{,}\PY{l+m+mi}{3}\PY{p}{)}

\PY{n}{msg} \PY{o}{=} \PY{l+s+sa}{f}\PY{l+s+s2}{\PYZdq{}}\PY{l+s+s2}{As shown above, we were able to get (average) profit of }\PY{l+s+si}{\PYZob{}}\PY{n}{avg\PYZus{}profit\PYZus{}act}\PY{l+s+si}{\PYZcb{}}\PY{l+s+s2}{ Euro with predicted price from XGBoost and optimization from CVXPY.}\PY{l+s+s2}{\PYZdq{}}
\PY{n}{msg} \PY{o}{+}\PY{o}{=} \PY{l+s+sa}{f}\PY{l+s+s2}{\PYZdq{}}\PY{l+s+se}{\PYZbs{}n}\PY{l+s+s2}{If we have perfect foresight, we would have earn about (average) }\PY{l+s+si}{\PYZob{}}\PY{n}{avg\PYZus{}profit\PYZus{}foresight}\PY{l+s+si}{\PYZcb{}}\PY{l+s+s2}{ Euro.}\PY{l+s+s2}{\PYZdq{}}

\PY{n+nb}{print}\PY{p}{(}\PY{n}{msg}\PY{p}{)}
\end{Verbatim}
\end{tcolorbox}

    \begin{Verbatim}[commandchars=\\\{\}]
we will use the following [Timestamp('2022-02-15 00:00:00'),
Timestamp('2022-04-02 00:00:00'), Timestamp('2022-05-18 00:00:00')]
2022-03-26 12:00:00 2022-03-27 12:00:00
2022-03-26 12:00:00 2022-03-27 12:00:00
2022-03-26 12:00:00 2022-03-27 12:00:00
As shown above, we were able to get (average) profit of 367.33 Euro with
predicted price from XGBoost and optimization from CVXPY.
If we have perfect foresight, we would have earn about (average) 363.837 Euro.
    \end{Verbatim}

    Q2 Conclusion

In summary, due to good performance of XGBoost in predicting the price,
we were able to obtain quite a good simulated profit from using
optimization tool in CVXPY. However, one should be aware that the
assumption for this usecase might not hold. For example, the battery
might have stochastic degradation. There might be a delay, energy loss
or even transaction cost when charging and discharging.

The time window for optimization is also small in our exercise. If the
time window is larger, it might requires a more runtime or a more
efficient algoritms.

Usually, to tackle above, it could be done by redefining more
constraints or refine the objective function further. However, it could
be that environment unpredicted that it has no solution or failed to
converge. For example, the forecast layer, it could be that suddenly one
of the important features are not available to us due to various reason
(e.g.~meteostat didn't work due to server error) and we might not able
to sit around.

In adding more dimensionality, it might be worth use Reinforcement
Learning to train an Agent that interacts with the past enviroment in
order to come out with an optimal actions.

    Question 3

    \section{\texorpdfstring{{\textbf{Reinforcement
Learning}}}{Reinforcement Learning}}\label{reinforcement-learning}

RL offers another way to optimizes optimization problem posed in Q2. In
this method, an agent is exposed to the environment and make decesion
through interacting with the environment to achieve a target. Here, the
environment is the energy market and other real life environmental
variables that may be helpful with agent learning and the target is to
charge and discharge at given DAA price in order to maximize profit.

In this section, we will first set the environment by defining the state
space, action space and reward function. Then, we will briefly give
compare the pros and cons of RL algorithms such as Q-learning, Deep
Q-Networks (DQN) and Proximal Policy Optimization (PPO) and select a RL
algorithm that is suitable for our usecase (spoiler: it's Q-learning)
and present its implementation.

{\textbf{Assumptions}}: Note in the following, we will inherit the
assumptions we have made for XGBoost in Q1 and the optimization problem
in Q2.

    Set Up

To begin, we shall define an epsidoe a time window where the agent will
interact with the market, \(t \in [0,T]\). Following from Q2, we shall
assume

{\textbf{State Space}}

Denote state space as \(\mathcal{S}\) and state as
\(s \in \mathcal{S}\). It is the observerable environment. For this, can
consider the most important features we have used in XGBoost, as well as
other observable features for charging/discharging market we have used
in the constraints for the optimization problem in Q2:

\begin{itemize}
\tightlist
\item
  \(b(t)\) the battery level at time t,
\item
  the properties of datetime such as day of week, month of year,
  weekends, holidays,
\item
  weather parameters such as temperature, snow fall, sun light
\item
  financial market properties such as the votalitiy index, DOWJONES or
  DAX
\item
  parameters from the energy sectors such as grid load, planned/actual
  consumption, planned/actual generation
\end{itemize}

We may also use XGBoost predicted price as feature, however, if we want
to compare if RL does better than XGBoost and using optimizer, then we
should omitted it.

{\textbf{Action Space}}

Denote state space as \(\mathcal{A}\). It is available action for the
agent to do, here we will motify the action from the optimization
problem to incorporate the state of the battery \(b(t)\) \[
a(t) =
\begin{cases}
1,\quad & \text{charge and if } b(t-1) \text{ not at max},\\
-1, & \text{discharge and if } b(t-1) \text{ not empty},\\
0, & \text{else},\\
\end{cases}
\] for all \(t \in (1\dots,T)\).

{\textbf{Reward Function}}

Reward function \(r_t\) is a way to `motivate' the training agent to
make the right choice, sort of like a feedback. For example, it can be
the following

\[
r_t \equiv r (a(t), b(t)) = p(t) \cdot a(t) - \text{penalty} (a(t), b(t), t)
\],

where \(\text{penalty}(a(t), b(t))\) be a penalty depending on the
action taken or the state of the battery. This can acts like constraint
as in the optimization problem. For example, we want to discourage the
agent to charge or discharge a certain time we can increase the penalty
for that time. Or that we do not want the battery to charge and
discharge the battery too frequently due to potential degradation from
rapid change in battery level.

    We make the following definition and notations:

\begin{itemize}
\item
  \(\pi(a|s)\) denote as policy - probability of choosing action \(a\)
  given state \(s\). This means that for a(t) at any \(t\),
  \(a(t) \sim \pi(a|s)\)
\item
  \(\gamma \in [0,1]\) discount factor - designed to discount the reward
  as time passes by and episode not ended, e.g.~\(\gamma^t\). Note, only
  set \(\gamma = 1\) if horizon is finite.
\item
  \(G_t := \sum_{j=t}^{T-1} \gamma^{j-t} r_j\) be the reward-to-go at
  time \(t\). Note here \(G_t = r_t + \gamma G_{t+1}\)
\end{itemize}

\textbf{Value Iteration}

The value function (state-value function), the expected return when
starting at \(s\) and following policy \(\pi\), is defined as \[
 V_\pi (s) := E_\pi [G_0|s_0 = s] = E_\pi \left[\sum_{t=0}^{T} \gamma^{t} r_t\bigg|s_0=s\right]
\]

The Q-function (action-value function), the expected return if we start
by acting \(a\) at state \(s\) and then follow \(\pi\) to choose actions
afterwards, is defined as \[
 Q_\pi (s) := E_\pi [G_0|s_0 = s, a_0 = a] = E_\pi \left[\sum_{t=0}^{T} \gamma^{t} r_t\bigg|s_0=s, a_0 =a\right]
\]

    \textbf{Optimal value function and policy}:

Generally, we want to get the following optimal equations known as
{\textbf{Bellman's optimality equations}}: \[
\begin{aligned}
V_{\pi_*}(s) & = \max_a R(s,a) + \gamma E_{p(s'|s,a)} [V_{\pi_*}(s')]
\\
Q_{\pi_*}(s,a) & = R(s,a) + \gamma E_{p(s'|s,a)} [\max_{a'} Q_{\pi_*}(s',a')],
\end{aligned}
\] where \(p(s'|s,a)\) is the probability desity function of reaching
state \(s'\) given last action \(a\) state \(s\) (trajectory
probability) and \(\pi_*\) is the optimal policy, \[
\begin{aligned}
{\pi_*}(s) & = \text{argmax}_a Q_{\pi_*}(s,a)
\\
& =  \text{argmax}_a  \bigg(R(s,a) + \gamma E_{p(s'|s,a)} [V_{\pi_*}(s')]\bigg)
\end{aligned}
\]

Note here we have a chicken-egg problem where we need optimized value
function to calculate optimal policy vice versa. In fact, they generally
shoud feedback each other to reach optimality via `Evaluation' (E) and
`Policy Improvement' (I) via `Boostrapping': \[
\pi_0 \rightarrow^E V_{\pi_0}  \rightarrow^I \pi_1 \rightarrow^E \cdots \rightarrow^I \pi_* \rightarrow^E V_{\pi_*},
\] we may replace \(V\) with \(Q\) is we would like to use action-value
function as eventually, it is proven that \[
Q_\pi(s,\pi(s')) \geqslant V_\pi(s) \Rightarrow V_\pi(s') \geqslant V_\pi(s).
\]

    Training Strategy

    \paragraph{\texorpdfstring{{\textbf{Algorithm
Selection}}:}{Algorithm Selection:}}\label{algorithm-selection}

In this section, we will briefly summarize the pros and cons of the
Algorithm and selection an algorithm that is best suited to our used
case. Here we will mainly consider model-free algorithm as it is more

\begin{itemize}
\tightlist
\item
  flexible for high dimensional feature
\end{itemize}

In summary, the algorithm can be described and catergozied as follows:
\textbar{} Algorithm \textbar{} Policy Type\textbar{} Action Space
\textbar{} State Space \textbar{} Remarks \textbar{}
\textbar--\textbar--\textbar--\textbar---\textbar---\textbar{}
\textbar{} Q-learning\textbar{} Off-policy \textbar{} Discrete
\textbar{} Small \textbar{} Simple but less stable compared to
DQN\textbar{} \textbar{} DQN \textbar Off-policy \textbar{} Discrete
\textbar{} Large/high-dimensional \textbar{} More stable than
Q-learning, but more complex (computational power \&
Explanability)\textbar{} \textbar{} PPO \textbar{} On-policy \textbar{}
continuous/discrete \textbar{} Large/high-dimensional \textbar{} Stable
but more complex \textbar{}

* On policy here allure to the Evaluation and Improvement cycles as
mentioned before, there action taken from policy which affects value
function which, in turns, affects next policy.

Given the summary above, we find that the best algorithm suited for the
exercise is {\textbf{Q-learning}} for the following reasons:

\begin{itemize}
\item
  Both action space and state space are discrete and small,
  i.e.~\(\mathcal{A} \in \{\text{charge, discharge, do nothing} \}\) and
  \(\mathcal{S} = \{ \text{DAA Price, weather features, financial market features} \}\)
  are relatively small features.
\item
  Easier to implement than, for example DQN.
\end{itemize}

    \paragraph{\texorpdfstring{{\textbf{Q-learning}} (off-policy TD
Model):}{Q-learning (off-policy TD Model):}}\label{q-learning-off-policy-td-model}

To begin introducing Q-learning, we will first introduce what Time
Difference (TD) model is. TD model is a model-free RL is where the agent
faces the model and use Q-function to guide policy and bootstrapping to
achieve optimum. The idea is to incrementally reduce the
{\textbf{Bellman error}}: \[
r + \gamma \max_{a'} Q_(s',a') - Q_(s,a),
\] through the evaluation cycles.

Q-learning model is a special case of TD model. Off-policy here means
that the model is an algorithm that learns Q-function \(Q_*\) even if a
suboptimal policy is used to choose action, \[
a' = \text{argmax}_{b\in\mathcal{A}} Q(s',b),
\] here we can implement \(\varepsilon\)-greedy exploration \[
b =
\begin{cases}
\text{argmax}_b Q(s,b),& \quad \text{with probability }1-\varepsilon,
\\
\text{random exploration action from }\mathcal{A},&  \quad \text{with probability }\varepsilon.
\end{cases}
\]

Thus, the resulting the boostrapping function to be \[
Q(s,a) \leftarrow Q(s,a) + \alpha \left[ r + \gamma \max_{b} Q(s',b) - Q(s,a)\right]
\]

\subparagraph{\texorpdfstring{\textbf{Q-learning
pseudocode}}{Q-learning pseudocode}}\label{q-learning-pseudocode}

\begin{Shaded}
\begin{Highlighting}[]
\NormalTok{Initialize env:}
\NormalTok{    Set initial battery level to a 0}
\NormalTok{    Set initial market conditions, i.e. DAA Price}
\NormalTok{    Define the maximum battery capacity (1 MHh)}
\NormalTok{    Define the maximum charging rate, (1 MH per hou)}
\NormalTok{    Define the maximum discharging rate, ({-}1MH per hour)}
\NormalTok{    Define time step size (hourly)}
\NormalTok{    Set up initial external factors (e.g., weather features, market features)}
\NormalTok{    episode (user choice, int)}

\NormalTok{Reset env (to be called at the start of each episode): }
\NormalTok{    Reset battery level to its initial state}
\NormalTok{    Reset market conditions to their initial state}
\NormalTok{    Reset external factors to their initial state}
\NormalTok{    Return initial observation (state) to the agent}

\NormalTok{Step(action):}
\NormalTok{    Given action from the agent (charge/discharge/do nothing):}
\NormalTok{        Update battery level based on action and constraints stated above}
\NormalTok{        Update market conditions (e.g., DAA price and all the features defined)}
\NormalTok{        Calculate reward based on action taken and current market conditions}
\NormalTok{        Check if the episode has ended (end of horizon, e.g. 24H)}
\NormalTok{        Return new state and reward and done flag (indicating end of episode)}

\NormalTok{Initialize Q(s, a) for all state{-}action pairs:}
\NormalTok{    For each battery state s in the state space S (0 or 1 MW):}
\NormalTok{        For each market action a in the action space A (charge/discharge/do nothing):}
\NormalTok{            Q(s, a) \textless{}{-} arbitrary\_value (often initialized to 0)}

\NormalTok{{-}{-} Main {-}{-}}
\NormalTok{Repeat (for each episode):}
\NormalTok{    Sample starting state s for new episode}
\NormalTok{    Repeat (for each step of the episode):}
\NormalTok{        Sample action b defined above}
\NormalTok{        Take action b, observe reward r, and next state s\textquotesingle{}}
\NormalTok{        Bootstrap: Q(s, a) \textless{}{-} Q(s, a) + α [r + γ max(Q(s\textquotesingle{}, b)) {-} Q(s, a)]}
\NormalTok{        s \textless{}{-} s\textquotesingle{}}
\NormalTok{    until state s is terminal}
\NormalTok{until converged}
\end{Highlighting}
\end{Shaded}

    Q3 Conclusion

Given the relatively limited and straightforward setup of the exercises,
along with the simplifying assumptions made in Q1 and Q2, using a
forecasting model combined with a basic optimization approach seems
sufficient.

However, as noted in the Q2 conclusion, these assumptions may not hold
in more realistic settings.

On the forecasting side, the chosen features or constraints used to
build the model may lose relevance over time for reasons that are
difficult to explain. This introduces uncertainty into the forecasting
model itself. For example, the Volatility Index (VIX), often called the
``fear index'', spiked during Russia's invasion of Ukraine in March
2022---an event that coincided with high energy prices due to Russia's
role as a major energy exporter to Europe. Yet, a similar spike in the
VIX in May 2022, potentially driven by inflation concerns, did not
impact Day-Ahead Auction (DAA) prices in the same way. This highlights
the unpredictability and possible disconnect between such indicators and
actual market behavior.

On the optimization side, the challenge grows if we consider a network
of batteries, each potentially following its own ill-defined stochastic
behavior, incurring different costs, and facing unique transaction
constraints. As the dimensionality and complexity of the problem
increase, traditional optimization methods may become less practical.

As such, in high-dimensional, dynamic environments, complex situations,
it could be more effective to use RL learning agents. These agents can
adapt to evolving environments and learn optimal strategies over time
without the need to explicitly define every constraint or model
parameter.

    \begin{tcolorbox}[breakable, size=fbox, boxrule=1pt, pad at break*=1mm,colback=cellbackground, colframe=cellborder]
\prompt{In}{incolor}{ }{\boxspacing}
\begin{Verbatim}[commandchars=\\\{\}]

\end{Verbatim}
\end{tcolorbox}


    % Add a bibliography block to the postdoc
    
    
    
\end{document}
